\chapter{La force de la liberté d'expression}
\label{les:6}

\begin{chapquote}{Lewis Carroll, \textit{Alice au pays des merveilles}}
\enquote{Je te demande pardon !} dit la Souris très poliment, mais en fronçant
le sourcil. \enquote{Tu as dit quelque chose ? }
\end{chapquote}

Bitcoin est une idée. Une idée qui, dans sa forme actuelle, est la manifestation
de rouages purement alimentés par du texte. Tous les aspects de Bitcoin sont du
texte : le livre blanc, c'est du texte. Le logiciel qui s'exécute sur les nœuds,
c'est du texte. Le registre, c'est du texte. Les transactions, ce sont du texte.
Les clés publiques et privées, ce sont du texte. Tous les aspects de Bitcoin
sont du texte, en conséquence ils sont équivalents à de la parole.

\begin{quotation}\begin{samepage}
\enquote{Le Congrès n'adoptera aucune loi relative à l'établissement d'une
religion, ou à l'interdiction de son libre exercice ; ou pour limiter la liberté
d'expression, de la presse ou le droit des citoyens de se réunir pacifiquement
ou d'adresser au Gouvernement des pétitions pour obtenir réparations des torts
subis.}
\begin{flushright} -- Premier amendement de la Constitution des États-Unis
\end{flushright}\end{samepage}\end{quotation}

Bien que la bataille finale des Crypto Wars\footnote{Les \textit{Crypto Wars}
est le nom officieux des tentatives par les États-Unis et les gouvernements
alliés de saboter le chiffrement des
données.~\cite{eff-cryptowars}~\cite{wiki:cryptowars}} n'ait pas encore été
livrée, il sera extrêmement difficile de criminaliser une idée, qui plus est une
idée basée sur l'échange de messages écrits. Chaque fois qu'un gouvernement
tente d'interdire du texte ou de la parole, nous glissons sur le chemin de
l'absurdité qui mène inévitablement aux abominations telles que les nombres
illégaux\footnote{Un nombre illégal est un nombre qui représente une information
qu'il est interdit de posséder, prononcer, diffuser ou transmettre dans une
juridiction légale donnée.\cite{wiki:illegal-number}} et les nombres premiers
illégaux\footnote{Un nombre premier illégal est un nombre premier qui représente
une information dont la possession ou la distribution est interdite dans une
quelconque juridiction légale. L'un des premiers nombres premiers illégaux fut
découvert en 2001. Lorsqu'interprété d'une façon particulière, il décrit un
programme informatique qui permet de contourner le système de gestion des
droits numériques sur les DVD. La distribution d'un tel programme aux États-Unis
est illégale selon le Digital Millennium Copyright Act. Un nombre premier
illégal est une sorte de nombre illégal.\cite{wiki:illegal-prime}}.

Tant que quelque part dans le monde, l'expression restera libre comme dans
\textit{liberté}, Bitcoin sera inarrêtable.

\begin{quotation}\begin{samepage}
\enquote{Il n'y a aucun moment lors d'une transaction où Bitcoin ne cesse d'être
du \textit{texte}. Ce n'est \textit{que du texte}, tout le temps. [...] Bitcoin,
c'est du \textit{texte}. Bitcoin, c'est de la \textit{parole}. Il ne peut pas
être réglementé dans un pays libre tel que les États-Unis qui possède des droits
inaliénables garantis et un premier amendement qui retire explicitement le droit
de publier du contrôle du gouvernement.}
\begin{flushright} -- Beautyon\footnote{Beautyon, \textit{Pourquoi l'Amérique ne
peut réglementer Bitcoin} \cite{america-regulate-bitcoin}}
\end{flushright}\end{samepage}\end{quotation}

\paragraph{Bitcoin m'a appris que dans une société libre, la liberté
d'expression et le logiciel libre étaient inarrêtables.}

% ---
%
% #### Through the Looking-Glass
%
% - [The Magic Dust of Cryptography: How digital information is changing our society][a magic spell]
%
% #### Down the Rabbit Hole
%
% - [Why America can't regulate Bitcoin][Beautyon] by Beautyon
% - [First Amendment to the United States Constitution][1st Amendment], [Crypto Wars], [illegal numbers], [illegal primes] on Wikipedia
%
% <!-- Through the Looking-Glass -->
% [a magic spell]: 
%
% <!-- Down the Rabbit Hole -->
% [1st Amendment]: https://en.wikipedia.org/wiki/First_Amendment_to_the_United_States_Constitution
% [Crypto Wars]: https://en.wikipedia.org/wiki/Crypto_Wars
% [illegal numbers]: https://en.wikipedia.org/wiki/Illegal_number
% [illegal primes]: https://en.wikipedia.org/wiki/Illegal_prime
% [Beautyon]: https://hackernoon.com/why-america-cant-regulate-bitcoin-8c77cee8d794
%
% <!-- Wikipedia -->
% [alice]: https://en.wikipedia.org/wiki/Alice%27s_Adventures_in_Wonderland
% [carroll]: https://en.wikipedia.org/wiki/Lewis_Carroll
