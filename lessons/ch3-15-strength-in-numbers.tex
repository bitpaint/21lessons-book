\chapter{La force dans les nombres}
\label{les:15}

\begin{chapquote}{Lewis Carroll, \textit{Alice au pays des merveilles}}
\enquote{Voyons un peu : quatre fois cinq font douze, quatre fois six font
treize, et quatre fois sept font… Oh ! mon Dieu ! jamais je n’arriverai jusqu’à
vingt à cette allure !}
\end{chapquote}

Les nombres sont indispensables à notre vie quotidienne. Les grands nombres, en
revanche, sont peu familiers à la plupart d'entre nous. Les plus grands nombres
que l'on rencontrera généralement au jour le jour seront de l'ordre des
millions, des milliards voire des billions\footnote{NdT : \textit{billion} est
la traduction de l'anglais \textit{trillion}.}. On pourra par exemple parler de
millions de gens dans la misère, de milliards de dollars dépensés à renflouer
les banques ou encore de billions de dette publique. Malgré la difficulté à se
représenter ce genre de gros titres, nous sommes relativement à l'aise avec
l'ordre de grandeur de ces nombres.

Même si nous sommes familiers avec les milliards et les billions, notre
intuition fait déjà défaut face à leur magnitude. Avez-vous une idée du temps
qu'il faut pour qu'un million, milliard ou billion de secondes ne s'écoulent ?
Si vous êtes comme moi, vous êtes perdu si vous ne faites pas les calculs.

Creusons cet exemple : la différence entre chaque représente trois ordres de
magnitude ; $10^6$, $10^9$, $10^{12}$. Ce n'est pas très utile de penser en
secondes, alors transformons-les afin de pouvoir les appréhender :

\begin{itemize}
  \item $10^6$: Un million de secondes représentent la dernière semaine et demie.
  \item $10^9$: Un milliard de secondes représentent les 32 dernières années.
  \item $10^{12}$: Il y a un billion de secondes, Manhattan était couverte d'une
  épaisse couche de glace\footnote{Un billion de secondes ($10^{12}$)
  représentent les $31710$ dernières années. Le Dernier Maximum Glaciaire a eu
  lieu il y a $33,000$ ans.~\cite{wiki:LGM}}.
\end{itemize}

\begin{figure}
  \includegraphics{assets/images/xkcd-1225.png}
  \caption{Il y a environ 1 billion de secondes. Source : xkcd n°1225}
  \label{fig:xkcd-1225}
\end{figure}

Au moment où nous entrons dans les échelles astronomiques de la cryptographie
moderne, notre instinct échoue dans les grandes largeurs. Bitcoin est bâti
autour des grands nombres et de l'impossibilité virtuelle à les deviner. Ces
nombres sont bien, bien plus grands que ceux qu'on peut rencontrer au quotidien.
Plus grands de beaucoup d'ordres de magnitude. Il est indispensable
d'appréhender à quel point ces nombres sont réellement grands pour pouvoir
comprendre Bitcoin dans son ensemble.

Prenons comme exemple concret SHA-256\footnote{SHA-256 fait partie de la famille
des fonctions de hachage cryptographique SHA-2 développée par la
NSA.~\cite{wiki:sha2}}, l'une des fonctions de hachage\footnote{Bitcoin utilise
SHA-256 dans son algorithme de hachage de bloc.~\cite{btcwiki:block-hashing}}
utilisée par Bitcoin. Il est naturel de se dire que 256 bits ne sont que
\enquote{deux cent cinquante-six}, ce qui n'est pas du tout un grand nombre.
Pourtant, le nombre dans SHA-256 représente des ordres de magnitude -- une chose
pour laquelle nos cerveaux ne sont pas équipés.

Bien que le nombre de bits soit une métrique appropriée, le vrai sens d'une
sécurité 256-bit se perd dans l'interprétation. À l'image des millions ($10^6$)
et des milliards ($10^9$) ci-dessus, le nombre dans SHA-256 parle d'ordres
de magnitude ($2^{256}$).

Mais alors à quel point SHA-256 est-il solide, au juste ?

\begin{quotation}\begin{samepage}
\enquote{SHA-256 est très solide. Ce n'est pas incrémental comme passer de MD5 à
SHA1. Ça pourrait prendre plusieurs décennies avant une attaque massive
révolutionnaire.}
\begin{flushright} -- Satoshi Nakamoto\footnote{Satoshi Nakamoto, dans une
réponse aux questions sur les collisions SHA-256. \cite{satoshi-sha256}}
\end{flushright}\end{samepage}\end{quotation}

Appelons un chat un chat. $2^{256}$ représente ce nombre :

\begin{quotation}\begin{samepage}
115 duodecilliards 792 duodecillions 89 undecilliards 237 undecillions 316
decilliards 195 decillions 423 nonilliards 570 nonillions 985 octilliards 8
octillions 687 septilliards 907 septillions 853 sextilliards 269 sextillions 984
quintilliards 665 quintillions 640 quadrilliards 564 quadrillions 39 trilliards
457 trillions 584 billiards 7 billions 913 milliards 129 millions 639 mille 936.
\end{samepage}\end{quotation}

Ça fait un paquet de quintillions ! Se faire une idée de ce nombre est somme
toute impossible. Il n'existe rien dans l'univers physique à quoi le comparer.
Ça représente bien plus que le nombre d'atomes de l'univers observable. Le
cerveau humain n'est tout simplement pas fait pour le comprendre.

\newpage

On trouve l'une des meilleures représentations de la vraie force de SHA-256 dans
une vidéo de Grant Sanderson. Judicieusement nommée \textit{\enquote{À quel
point la sécurité 256-bit est-elle sûre ?}}\footnote{Regardez la vidéo sur
\url{https://youtu.be/S9JGmA5_unY}}, elle montre parfaitement l'immensité d'un
espace de cette taille. Rendez-vous service et prenez cinq minutes pour la
regarder. Comme toutes les vidéos de \textit{3Blue1Brown} celle-ci est tout
autant fascinante qu'exceptionnellement bien faite. Mais attention : vous
pourriez bien tomber dans un terrier de lapin mathématique.

\begin{figure}
  \includegraphics{assets/images/youtube-vid-inverted.png}
  \caption{Illustration de la sécurité dans SHA-256. Schéma original de Grant
  Sanderson alias 3Blue1Brown.}
  \label{fig:youtube-vid-inverted}
\end{figure}

Bruce Schneier~\cite{web:schneier} s'est servi des limites physiques de
l'informatique afin de relativiser ce nombre : même si nous parvenions à
construire un ordinateur optimal, qui utiliserait sans perte l'énergie fournie
pour manipuler les bits~\cite{wiki:landauer}, que nous construisions une sphère
de Dyson\footnote{Une sphère de Dyson est une mégastructure hypothétique qui
entoure totalement une étoile et capture un grand pourcentage de son
énergie.~\cite{wiki:dyson}} autour du Soleil et que nous les laissions tourner
pendant 100 milliards de milliards d'années, nous n'aurions que $25\%$ de
chances de trouver une aiguille dans une botte de 256 bits.

\begin{quotation}\begin{samepage}
\enquote{Ces nombres n'ont rien à voir avec la technologie des appareils ; ce
sont les maximums autorisés par la thermodynamique. Et ils suggèrent fortement
que les attaques par force brute contre des clés de 256 bits ne seront pas
envisageables avant que les ordinateurs ne soient faits d'autre chose que la
matière et occupent autre chose que l'espace.}
\begin{flushright} -- Bruce Schneier\footnote{Bruce Schneier,
\textit{Cryptographie appliquée} \cite{bruce-schneier}}
\end{flushright}\end{samepage}\end{quotation}

On ne peut surestimer la profondeur de ces mots. Une cryptographie solide
renverse l'équilibre des pouvoirs du monde physique auquel nous sommes habitués.
Dans notre réalité, les choses incassables n'existent pas. Si vous y mettez
assez de force, vous pourrez ouvrir n'importe quelle porte, boîte ou coffre au
trésor.

Le coffre au trésor de Bitcoin est très différent. Il est protégé par une
cryptographie solide, qui ne laisse pas la place à la force brute. Tant que les
hypothèses mathématiques sous-jacentes s'appliqueront, nous n'aurons que cette
force brute à notre disposition. Bon, d'accord, il y a aussi l'option d'une
attaque globale à la clé à molette à 5\$ (Figure~\ref{fig:xkcd-538}). Mais la
torture ne fonctionnera pas pour toutes les adresses bitcoin et les remparts
cryptographiques de bitcoin mettront en échec les attaques par force brute.
Même si vous attaquez avec la puissance d'un millier d'étoiles. Littéralement.

\begin{figure}
  \centering
  \includegraphics[width=8cm]{assets/images/xkcd-538.png}
  \caption{Attaque à la clé à molette à 5\$. Source : xkcd n°538}
  \label{fig:xkcd-538}
\end{figure}

Ce fait et ses répercussions ont été résumés de façon bouleversante dans l'appel
aux armes cryptographiques : \textit{\enquote{Aucune force coercitive ne
résoudra jamais un problème de maths.}}

\begin{quotation}\begin{samepage}
\enquote{Ça n'avait rien d'évident, que le monde finirait par fonctionner comme
ça. Mais d'une façon ou d'une autre, l'univers sourit au chiffrement.}
\begin{flushright} -- Julian Assange\footnote{Julian Assange, \textit{Un appel
aux armes cryptographiques} \cite{call-to-cryptographic-arms}}
\end{flushright}\end{samepage}\end{quotation}

Personne ne sait encore avec certitude si le sourire de l'univers est
authentique. Il est possible que notre hypothèse des asymétries mathématiques
soit fausse et que nous découvrions qu'en réalité P est égal à NP
\cite{wiki:pnp} ou que nous trouvions étonnamment rapidement une solution à des
problèmes spécifiques \cite{wiki:discrete-log} que nous estimons actuellement
complexes. Si cela devait arriver, la cryptographie telle que nous la
connaissons disparaîtrait et les conséquences changeraient sans doute
radicalement la face du monde.

\begin{quotation}\begin{samepage}
\enquote{Vires in Numeris} = \enquote{Les forces dans les
nombres}\footnote{\textit{Vires in Numeris} à été proposé comme devise de
Bitcoin pour la première fois par l'utilisateur de bitcointalk
\textit{epii}~\cite{epii}}
\end{samepage}\end{quotation}

\textit{Vires in numeris} n'est pas qu'un slogan accrocheur pour les bitcoiners.
La prise de conscience d'une force inimaginable présente dans les nombres est
intense. En faire l'expérience et comprendre l'inversion dans l'équilibre des
pouvoirs existants qui en découle a changé ma façon de voir le monde et l'avenir
qui nous attend.

Un effet direct de cela, c'est que vous n'avez pas à demander la permission à
quiconque pour participer à Bitcoin. Il n'y a pas de page d'inscription, pas
d'entreprise qui en est responsable, pas d'agence publique à qui envoyer les
formulaires. Générez simplement un grand nombre et vous êtes à peu près paré à y
aller. L'autorité centrale sur la création des comptes, ce sont les
mathématiques. Et Dieu seul sait qui en est responsable.

\begin{figure}
  \includegraphics{assets/images/elliptic-curve-examples.png}
  \caption{Exemples de courbes elliptiques. Crédit schéma CC-BY-SA Emmanuel
  Boutet.}
  \label{fig:elliptic-curve-examples}
\end{figure}

Bitcoin est bâti sur notre meilleure compréhension de la réalité. Certes, il
reste bien des problèmes non-résolus en physique, en informatique et en
mathématiques, mais il y a des choses dont nous sommes plutôt sûrs. Qu'il y ait
une asymétrie entre trouver des solutions et valider la justesse de ces
solutions en est une. Que les calculs requièrent de l'énergie en est une autre.
En d'autres termes : trouver une aiguille dans une botte de foin est plus
difficile que de vérifier si le truc pointu dans votre main est bien une
aiguille. Et trouver l'aiguille prend du temps.

L'immensité de l'espace d'adressage de Bitcoin est vraiment ahurissante. Le
nombre de clés privées l'est encore plus. C'est fascinant de se rendre compte à
quel point notre monde moderne se résume à l'improbabilité de trouver une
aiguille dans une botte de foin incommensurable. J'en suis conscient plus que
jamais, dorénavant.

\paragraph{Bitcoin m'a appris que les nombres renfermaient de la puissance.}

% ---
%
% #### Down the Rabbit Hole
%
% - [How secure is 256 bit security?]["How secure is 256 bit security?"] by 3Blue1Brown
% - [Block Hashing Algorithm][hash functions] on the Bitcoin Wiki
% - [Last Glacial Maximum][thick layer of ice], [SHA-2][SHA-256], [Dyson Sphere][Dyson sphere], [Landauer's Principle][flip bits perfectly] [P versus NP][P actually equals NP], [Discrete Logarithm][specific problems] on Wikipedia
%
% [thick layer of ice]: https://en.wikipedia.org/wiki/Last_Glacial_Maximum
% [xkcd \#1125]: https://xkcd.com/1225/
% [SHA-256]: https://en.wikipedia.org/wiki/SHA-2
% [hash functions]: https://en.bitcoin.it/wiki/Block_hashing_algorithm
% ["How secure is 256 bit security?"]: https://www.youtube.com/watch?v=S9JGmA5_unY
% [Bruce Schneier]: https://www.schneier.com/
% [flip bits perfectly]: https://en.wikipedia.org/wiki/Landauer%27s_principle#Equation
% [Dyson sphere]: https://en.wikipedia.org/wiki/Dyson_sphere
% [2]: https://books.google.com/books?id=Ok0nDwAAQBAJ&pg=PT316&dq=%22These+numbers+have+nothing+to+do+with+the+technology+of+the+devices;%22&hl=en&sa=X&ved=0ahUKEwjXttWl8YLhAhUphOAKHZZOCcsQ6AEIKjAA#v=onepage&q&f=false
% [wrench attack]: https://xkcd.com/538/
% [call to cryptographic arms]: https://cryptome.org/2012/12/assange-crypto-arms.htm
% [P actually equals NP]: https://en.wikipedia.org/wiki/P_versus_NP_problem#P_=_NP
% [specific problems]: https://en.wikipedia.org/wiki/Discrete_logarithm#Cryptography
% [3Blue1Brown]: https://twitter.com/3blue1brown
%
% <!-- Wikipedia -->
% [alice]: https://en.wikipedia.org/wiki/Alice%27s_Adventures_in_Wonderland
% [carroll]: https://en.wikipedia.org/wiki/Lewis_Carroll
