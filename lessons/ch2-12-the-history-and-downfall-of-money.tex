\chapter{L'histoire et le déclin de la monnaie}
\label{les:12}

\begin{chapquote}{Lewis Carroll, \textit{Alice au pays des merveilles}}
\enquote{\ldots ils avaient refusé de se rappeler les simples règles de conduite
que leurs amis leur avaient enseignées : par exemple, qu’un tisonnier chauffé au
rouge vous brûle si vous le tenez trop longtemps, ou que, si vous vous faites au
doigt une coupure très profonde avec un couteau, votre doigt, d’ordinaire, se
met à saigner ; et Alice n’avait jamais oublié que si l’on boit une bonne partie
du contenu d’une bouteille portant l’étiquette : poison, cela ne manque presque
jamais, tôt ou tard, de vous causer des ennuis.}
\end{chapquote}

Beaucoup de personnes pensent que la monnaie est soutenue par de l'or, qui
serait enfermé dans de grandes chambres fortes, protégées par d'épais murs. Ce
n'est plus vrai depuis plusieurs décennies. Je ne suis pas certain de ce que
j'en ai pensé quand je l'ai appris, puisque j'étais encore plus embêté, n'ayant
potentiellement aucune compréhension de l'or, de l'argent papier ou même de
pourquoi il faudrait le soutenir avec quelque chose, pour commencer.

Un des aspects de l'étude de Bitcoin est l'étude de la monnaie fiduciaire : ce
que ça signifie, comment c'est apparu et pourquoi c'est peut-être pas la
meilleure idée que nous ayons eu. Alors, qu'est-ce que la monnaie fiduciaire
exactement ? Et comment en est-on arrivés à l'utiliser ?

Lorsque quelque chose est imposé par \textit{décret}, ça veut simplement dire
c'est imposé par une autorisation ou une proposition officielle. Par conséquent,
la monnaie fiduciaire est monnaie simplement parce que \textit{quelqu'un} le
décide. De nos jours, comme tous les gouvernements utilisent de la monnaie
fiduciaire, ce quelqu'un c'est \textit{votre} gouvernement. Malheureusement,
vous n'êtes pas \textit{libre} d'être en désaccord avec cette proposition de
valeur. Vous vous apercevrez rapidement qu'elle est tout sauf non-violente. Si
vous refusez d'utiliser cette monnaie papier pour mener vos affaires et payer
vos impôts, les seules personnes avec qui vous pourrez parler économie seront
vos compagnons de cellule.

La valeur de la monnaie fiduciaire ne découle pas de ses attributs intrinsèques.
La qualité d'une catégorie donnée de monnaie fiduciaire n'est corrélée qu'à 
l'(in)stabilité politique et fiscale de ceux qui la font passer du rêve à la
réalité. Sa valeur est décrétée, arbitrairement.

\begin{figure}
  \centering
  \includegraphics[width=8cm]{assets/images/fiat-definition.png}
  \caption{fiat --- `Qu'il en soit ainsi'}
  \label{fig:fiat-definition}
\end{figure}

\paragraph{}
Jusqu'à récemment, deux sortes de monnaie étaient d'usage : la \textbf{monnaie
de commodité}, faite de \textit{choses} précieuses ; et la \textbf{monnaie
représentative}, qui \textit{représente} uniquement la chose précieuse,
généralement par des jeux d'écriture.

\paragraph{}
Nous avons déjà abordé la monnaie de commodité plus haut. Les gens utilisaient
des os, des coquillages et des métaux précieux comme monnaie. Plus tard, ce sont
principalement des pièces faites de ces métaux comme l'or ou l'argent qui
furent utilisées. La plus vieille pièce qu'on ait trouvée jusqu'à aujourd'hui
est faite d'un alliage naturel d'or et d'argent et a été frappée il y a plus de
2700 ans\footnote{D'après l'historien grec Hérodote, contemporain du Ve siècle
avant J.-C., les Lydiens furent le premier peuple à utiliser des pièces d'or et
d'argent. \cite{coinage-origins}}. S'il y a de l'innovation dans Bitcoin, ce
n'est pas le concept de pièce.

\begin{figure}
  \centering
  \includegraphics[width=5cm]{assets/images/lydian-coin-stater.png}
  \caption{Pièce lydienne en électrum. Crédit photo CC-BY-SA Classical
  Numismatic Group, Inc.}
  \label{fig:lydian-coin-stater}
\end{figure}

\paragraph{}
Il s'avère que la thésaurisation, ou hodling, pour reprendre le jargon moderne,
est presque aussi vieille que les pièces elles-mêmes. Le plus ancien hodler de
pièces était une personne qui a mis une centaine de celles-ci dans une jarre et
l'a enterrée sous les fondations d'un temple, pour qu'on ne les retrouve qu'au
bout de 2500 ans. Plutôt un bon stockage à froid si vous voulez mon avis.

L'un des inconvénients d'utiliser des pièces faites de métaux précieux est
qu'elles peuvent être rognées, dépréciant de fait leur valeur. De nouvelles
pièces peuvent être frappées à partir des copeaux, faisant gonfler l'offre de
monnaie au fil du temps, dévaluant chaque autre pièce au passage. Les gens
rabotaient littéralement tout ce qu'ils pouvaient de leurs dollars d'argent. Je
me demande quel genre de pubs \textit{Dollar Shave Club} faisait à l'époque.

Puisque la seule inflation que les gouvernements tolèrent est celle dont ils
sont responsables, des efforts furent entrepris pour mettre un terme à cette
guérilla de la dépréciation. Au jeu traditionnel du gendarme et du voleur, les
rogneurs de pièces ont fait preuve d'imagination dans leurs procédés, forçant
les \enquote{Maîtres de la Monnaie} à faire preuve d'encore plus d'imagination
dans leurs contre-mesures. Isaac Newton, le physicien mondialement reconnu,
auteur de \textit{Principia Mathematica}, était l'un de ces maîtres. C'est à lui
qu'on attribue l'ajout des stries sur la tranche des pièces que l'on peut
toujours observer aujourd'hui. L'époque du rognage facile était révolue.

\begin{figure}
  \includegraphics{assets/images/clipped-coins.png}
  \caption{Pièces d'argent rognées à divers degrés.}
  \label{fig:clipped-coins}
\end{figure}

Même en gardant un œil sur les procédés de dépréciation des pièces\footnote{En
plus du rognage, le grippage (secouer les pièces dans un sac et récupérer la
poussière de métal qui s'est détachée) ainsi que le tamponnage (faire un trou au
centre de la pièce et l'aplatir au marteau jusqu'à combler le trou) étaient les
techniques de dépréciation les plus répandues. \cite{wiki:coin-debasement}},
celles-ci faisaient encore face à d'autres problèmes. Elles sont encombrantes et
pas très pratiques à transporter, surtout en cas de grands transferts de valeur.
C'est pas vraiment faisable d'arriver avec un gros sac de dollars en argent
chaque fois que vous voulez acheter une Mercedes.

En parlant de trucs allemands : l'origine du nom du \textit{dollar} américain
est une autre histoire intéressante. Le mot \enquote{dollar} est dérivé du mot
allemand \textit{Thaler}, l'abréviation de
\textit{Joachimsthaler}~\cite{wiki:thaler}. Un Joachimsthaler était une pièce
frappée dans la ville de \textit{Sankt Joachimsthal}. Thaler est simplement le
raccourci pour désigner quelqu'un (ou quelque chose) qui vient de la vallée. Et
puisque Joachimsthal était \textit{la} vallée ou l'on produisait les pièces
d'argent, les gens appelaient naturellement ces pièces des \textit{Thaler}.
Thaler (en allemand) a glissé vers daalders (en hollandais) puis finalement vers
dollars (en anglais).

\begin{figure}
  \centering
  \includegraphics[width=5cm]{assets/images/joachimsthaler.png}
  \caption{Le `dollar' originel. Saint Joachim est représenté avec sa robe et
  son chapeau de mage. Crédit photo CC-BY-SA Wikipedia utilisateur
  Berlin-George}
  \label{fig:joachimsthaler}
\end{figure}

L'introduction de la monnaie représentative sonna le glas de la monnaie dure.
Les certificats sur l'or furent introduits en 1863 et quinze ans après, le
dollar d'argent fut lui aussi lentement mais sûrement remplacé par un
intermédiaire de papier : le certificat sur l'argent.
\cite{wiki:silver-certificate}

Il aura ensuite fallu environ un demi-siècle après l'arrivée des certificats
pour que ces morceaux de papier se changent en ce que nous connaissons de nos
jours comme le dollar américain.

\begin{figure}
  \centering
  \includegraphics{assets/images/us-silver-dollar-note-smaller.png}
  \caption{Un dollar d'argent américain de 1928. `Payable au porteur sur
  demande.' Crédit photo CC-BY-SA Collection numismatique nationale de
  l'institut Smithsonian}
  \label{fig:us-silver-dollar-note-smaller}
\end{figure}

Notez que le dollar d'argent américain de 1928 dans la
figure~\ref{fig:us-silver-dollar-note-smaller} porte toujours le nom de
\textit{certificat sur l'argent}, ce qui indique qu'il s'agit bien d'un document
stipulant que l'on doit au porteur de ce bout de papier un peu de métal argenté.
Il est intéressant de remarquer que le texte l'indiquant s'est vu rétrécir au
fil du temps. La trace du mot \textit{certificat} a fini par totalement
disparaître, remplacée par la déclaration rassurante qu'il s'agit de billets de
la réserve fédérale.

Nous l'avons évoqué plus haut, il s'est passé la même chose avec l'or. Dans sa
majorité, le monde fonctionnait sur un étalon
bimétallique~\cite{wiki:bimetallism}, ce qui signifie que les pièces étaient
principalement composées d'or et d'argent. C'était à n'en pas douter une avancée
technologique d'avoir des certificats sur l'or, échangeables contre des pièces
de ce même métal. Le papier est plus pratique, plus léger et puisqu'il est
divisible arbitrairement en inscrivant simplement dessus un nombre plus petit,
il est facile de le réduire en plus petites unités.

Afin de rappeler aux porteurs (utilisateurs) que ces certificats représentaient
de l'or ou de l'argent bien réels, ils revêtaient une couleur évocatrice et
l'indiquaient clairement en toutes lettres. Vous pouvez facilement lire ce
message de haut en bas :

\begin{quotation}\begin{samepage}
\enquote{Il est certifié par la présente que cent dollars en pièces d'or,
payables au porteur sur demande, ont été déposés au trésor des États-Unis
d'Amérique.}
\end{samepage}\end{quotation}

\begin{figure}
  \centering
  \includegraphics{assets/images/us-gold-cert-100-smaller.png}
  \caption{Un certificat américain sur l'or de 100\$ de 1928. Crédit photo
  CC-BY-SA Collection numismatique nationale, Musée National de l'Histoire
  américaine.}
  \label{fig:us-gold-cert-100-smaller}
\end{figure}

En 1963, les mots \enquote{PAYABLES AU PORTEUR SUR DEMANDE} furent retirés de
tous les nouveaux billets. Cinq ans plus tard, il fut mis fin à la
convertibilité des billets en or ou en argent.

Les mots qui rappelaient l'origine de la monnaie papier et l'idée qui
l'accompagne furent supprimés. La couleur dorée disparut. Il ne restait plus que
le papier et avec lui, pour le gouvernement, la possibilité d'en imprimer autant
qu'il le voulait.

Ce tour de passe-passe long de plus d'un siècle fut achevé en 1971 par
l'abolition de l'étalon-or. L'argent devint l'illusion que nous partageons tous
dorénavant : la monnaie fiduciaire. Elle a de la valeur car quelqu'un, qui
commande une armée et gère des prisons, a dit qu'elle en avait. On peut très
clairement le lire sur chaque dollar en circulation aujourd'hui, \enquote{CE
BILLET A COURS LÉGAL}. Autrement dit : il vaut quelque chose parce que c'est
écrit dessus.

\begin{figure}
  \centering
  \includegraphics{assets/images/us-dollar-2004.jpg}
  \caption{Un billet américain de vingt dollars de 2004 utilisé de nos jours.
  `CE BILLET A COURS LÉGAL'}
  \label{fig:us-dollar-2004}
\end{figure}

À ce propos, il y a une autre leçon intéressante sur les billets modernes qui se
cache sous nos yeux. La deuxième ligne indique que le cours légal concerne
\enquote{TOUTES LES DETTES, PUBLIQUES ET PRIVÉES}. J'ai été surpris par ce qui
peut paraître une évidence pour les économistes : tout l'argent consiste en de
la dette. J'en ai encore des migraines, je laisserai donc au lecteur la tâche
d'étudier le lien entre l'argent et la dette.

\paragraph{}
Nous l'avons vu, l'or et l'argent ont été utilisés comme monnaie pendant des
millénaires. Au fil du temps, les pièces d'or et d'argent furent remplacées par
du papier. Celui-ci a lentement été accepté comme moyen de paiement. Cette
adoption créa une illusion --- l'illusion que le papier lui-même a de la valeur.
Le coup de grâce fut de totalement rompre le lien entre la représentation et le
réel : abolir l'étalon-or en convainquant tout le monde que c'est le papier qui
est précieux.

\paragraph{Bitcoin m'a appris l'Histoire de la monnaie et le plus important tour
de passe-passe dans l'Histoire de l'économie : la monnaie fiduciaire.}

% ---
%
% #### Down the Rabbit Hole
%
% - [Shelling Out: The Origins of Money] by Nick Szabo
% - [Methods of Coin Debasement][coin debasement], [Thaler], [U.S. Silver Certificate][silver certificates], [Bimetallism][bimetallic standard] on Wikipedia
%
% [oldest coin]: https://www.britishmuseum.org/explore/themes/money/the_origins_of_coinage.aspx
% [coin debasement]: https://en.wikipedia.org/wiki/Methods_of_coin_debasement
% [Thaler]: https://en.wikipedia.org/wiki/Thaler
% [Berlin-George]: https://en.wikipedia.org/wiki/File:Bohemia,_Joachimsthaler_1525_Electrotype_Copy._VF._Obverse..jpg
% [silver certificates]: https://en.wikipedia.org/wiki/Silver_certificate_%28United_States%29
% [bimetallic standard]: https://en.wikipedia.org/wiki/Bimetallism
% [Shelling Out: The Origins of Money]: https://nakamotoinstitute.org/shelling-out/
%
% <!-- Wikipedia -->
% [alice]: https://en.wikipedia.org/wiki/Alice%27s_Adventures_in_Wonderland
% [carroll]: https://en.wikipedia.org/wiki/Lewis_Carroll
