\chapter{Immutabilité et changement}
\label{les:1}

\begin{chapquote}{Alice}
\enquote{Je me demande si on m’a changée pendant la nuit ? Voyons,
réfléchissons : est-ce que j’étais bien la même quand je me suis levée ce
matin ? Je crois me rappeler que je me suis sentie un peu différente. Mais, si
je ne suis pas la même, la question qui se pose est la suivante : Qui diable
puis-je bien être ? Ah, c’est là le grand problème !}
\end{chapquote}

Bitcoin est fondamentalement difficile à décrire. C'est un \textit{truc
nouveau}, donc chaque tentative de comparaison à des concepts antérieurs -- y
compris l'appeler or numérique ou Internet de l'argent -- est condamnée à ne pas
pouvoir rendre compte de son entièreté. Quelle que soit votre analogie favorite,
il y a deux aspects de Bitcoin réellement essentiels : la décentralisation et
l'immutabilité.

\paragraph{}
On peut voir Bitcoin comme un contrat social automatisé\footnote{Hasu, Unpacking
Bitcoin's Social Contract~\cite{social-contract}}. Le logiciel est juste une des
pièces du puzzle, de sorte que vouloir changer Bitcoin en changeant le logiciel
est absolument futile. Il faudrait pour cela convaincre l'ensemble du réseau
d'adopter les changements, ce qui tient plus de l'effort psychologique que de
l'effort d'ingénierie.

\paragraph{}
Ce qui suit peut sembler absurde au départ, comme tant d'autres choses dans ce
domaine, mais je crois fermement en la vérité de cette maxime : ce n'est pas
vous qui changerez Bitcoin, c'est Bitcoin qui vous changera.

\begin{quotation}\begin{samepage}
\enquote{Bitcoin nous changera plus que nous ne le changerons.}
\begin{flushright} -- Marty Bent\footnote{Tales From the Crypt~\cite{tftc21}}
\end{flushright}\end{samepage}\end{quotation}

Ça m'a pris un bon moment pour comprendre la profondeur de cette phrase. Après
tout, comme Bitcoin est juste un logiciel et qu'il est entièrement libre, on
peut simplement changer les choses à volonté, non ? Faux. \textit{Totalement}
faux. Sans surprise, l'inventeur de Bitcoin le savait parfaitement.

\begin{quotation}\begin{samepage}
\enquote{La nature de Bitcoin est telle qu'une fois sortie la version 0.1, les
concepts essentiels étaient gravés dans le marbre pour le restant de ses jours.}
\begin{flushright} -- Satoshi Nakamoto\footnote{Message du forum BitcoinTalk :
`Re: Transactions and Scripts\ldots'~\cite{satoshi-set-in-stone}}
\end{flushright}\end{samepage}\end{quotation}

De nombreuses personnes ont tenté de modifier la nature de Bitcoin. Elles ont
toutes échoué jusqu'à présent. Bien qu'il existe une étendue infinie de forks et
d'altcoins, le réseau Bitcoin continue sa route, exactement comme à la mise en
ligne du prenier nœud. Les altcoins n'importent pas sur le long terme. Les forks
finiront par mourir de faim. Ce qui importe c'est Bitcoin. Tant que notre
compréhension fondamentale des mathématiques et/ou de la physique ne change pas,
le ratel Bitcoin continuera de s'en moquer.

\begin{quotation}\begin{samepage}
\enquote{Bitcoin est le premier exemple d'une nouvelle forme de vie. Il vit et
respire sur internet. Il vit car il est capable de payer des gens pour le
maintenir en vie. [\ldots] Il ne peut être changé. On ne peut le contredire. On
ne peut l'altérer. On ne peut le corrompre. On ne peut l'arrêter. [\ldots] Si
une guerre nucléaire détruisait la moitié de la planète, il continuerait à
vivre, intact.}
\begin{flushright} -- Ralph Merkle\footnote{DAOs, Democracy and
Governance,~\cite{merkle-dao}}
\end{flushright}\end{samepage}\end{quotation}

Le cœur de Bitcoin battra plus longtemps que tous les nôtres.

~

Comprendre tout ça m'a fait changer bien plus que ne le feront les précédents
blocs de Bitcoin. Ma préférence temporelle a été modifiée, ma compréhension de
l'économie, mes opinions politiques et bien plus encore. Mince, ça change même
le régime alimentaire des gens\footnote{Inside the World of the Bitcoin
Carnivores,~\cite{carnivores}}. Si tout ça vous semble dingue, vous êtes au bon
endroit. Tout ceci est dingue en effet ; et c'est pourtant ce qui se passe.

~

\paragraph{Bitcoin m'a appris qu'il ne changerait pas. C'est moi qui changerai.}

% ---
%
% #### Through the Looking-Glass
%
% - [Bitcoin's Gravity: How idea-value feedback loops are pulling people in][gravity]
% - [Lesson 18: Move slowly and don't break things][lesson18]
%
% #### Down the Rabbit Hole
%
% - [Unpacking Bitcoin's Social Contract][automated social contract]: A framework for skeptics by Hasu
% - [DAOs, Democracy and Governance][Ralph Merkle] by Ralph C. Merkle
% - [Marty's Bent][bent]: A daily newsletter highlighting signal in Bitcoin by Marty Bent
% - [Technical Discussion on Bitcoin's Transactions and Scripts][Satoshi Nakamoto] by Satoshi Nakamoto, Gavin Andresen, and others
% - [Inside the World of the Bitcoin Carnivores][carnivores]: Why a small community of Bitcoin users is eating meat exclusively by Jordan Pearson
% - [Tales From the Crypt][tftc] hosted by Marty Bent
%
% <!-- Internal -->
% [gravity]: 
% [lesson18]: {{ 'bitcoin/lessons/ch3-18-move-slowly-and-dont-break-things' | absolute_url }}
%
% <!-- Further Reading -->
% [automated social contract]: https://medium.com/@hasufly/bitcoins-social-contract-1f8b05ee24a9
% [carnivores]: https://motherboard.vice.com/en_us/article/ne74nw/inside-the-world-of-the-bitcoin-carnivores
% [tftc]: https://tftc.io/tales-from-the-crypt/
% [bent]: https://tftc.io/martys-bent/
%
% <!-- Quotes -->
% [Ralph Merkle]: http://merkle.com/papers/DAOdemocracyDraft.pdf
% [Satoshi Nakamoto]: https://bitcointalk.org/index.php?topic=195.msg1611#msg1611
%
% <!-- Twitter People -->
% [Marty Bent]: https://twitter.com/martybent
%
% <!-- Wikipedia -->
% [alice]: https://en.wikipedia.org/wiki/Alice%27s_Adventures_in_Wonderland
% [carroll]: https://en.wikipedia.org/wiki/Lewis_Carroll
