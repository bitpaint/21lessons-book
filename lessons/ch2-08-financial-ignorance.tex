\chapter{La méconnaissance financière}
\label{les:8}

\begin{chapquote}{Lewis Carroll, \textit{Alice au pays des merveilles}}
\enquote{Et la dame pensera que je suis une petite fille ignorante ! Non, il
vaudra mieux ne rien demander ; peut-être que je verrai le nom écrit quelque
part.}
\end{chapquote}

L'une des choses qui m'a le plus frappé, c'est la quantité de finance,
d'économie et de psychologie nécessaire à la compréhension de ce qui semble, à
première vue, un système purement \textit{technique} -- un réseau informatique.
Pour paraphraser un petit gars aux pieds poilus : \enquote{Il est fort
dangereux, Frodon, d'étudier Bitcoin. On lit le livre blanc et si l'on ne
regarde pas où l'on met les pieds, on ne sait pas jusqu'où cela peut nous
mener.}

Pour comprendre un nouveau système monétaire, il faut apprendre à connaître
l'ancien. J'ai très vite réalisé que la quantité d'éducation financière
reçue au sein du système scolaire était à peu près égale à \textit{zéro}.

\paragraph{}
Comme un enfant, j'ai commencé à me poser bon nombre de questions : comment
fonctionne le système bancaire ? Comment fonctionnent les marchés financiers ?
Qu'est-ce que la monnaie fiduciaire ? Qu'est-ce que la monnaie
\textit{normale} ? Pourquoi existe-t-il autant de
dette ?\footnote{\url{https://www.usdebtclock.org/}} Combien de monnaie est
véritablement émise, et qui décide de ça ?

\newpage

Après une légère panique face à l'ampleur de mon ignorance, je me suis senti
rassuré lorsque j'ai constaté que j'étais en bonne compagnie.

\begin{quotation}\begin{samepage}
\enquote{C'est pas ironique que Bitcoin m'en ait appris plus sur la
monnaie que toutes ces années à travailler pour des institutions financières ?
\ldots y compris en ayant commencé dans une banque centrale}
\begin{flushright} -- Aaron\footnote{Aaron (\texttt{@aarontaycc},
\texttt{@fiatminimalist}), tweet du 12 déc. 2018~\cite{aarontaycc-tweet}}
\end{flushright}\end{samepage}\end{quotation}

\begin{quotation}\begin{samepage}
\enquote{J'ai plus appris sur la finance, l'économie, la technologie, la
cryptographie, la psychologie, la politique, la théorie des jeux, la législation
et moi-même pendant les trois derniers mois dans la crypto que mes trois années
et demie d'université}
\begin{flushright} -- Dunny\footnote{Dunny (\texttt{@BitcoinDunny}), tweet du 28
nov. 2017~\cite{bitcoindunny-tweet}}
\end{flushright}\end{samepage}\end{quotation}

Deux exemples des nombreuses confessions que l'on trouve un peu partout sur
Twitter\footnote{Voir \url{http://bit.ly/btc-learned} pour plus de confessions
Twitter.}. Bitcoin, comme nous l'avons vu dans la Leçon \ref{les:1}, est un
organisme vivant. Mises prétendait que l'économie aussi est vivante. Et nous le
savons tous par expérience, le vivant est difficile à appréhender par nature.

\begin{quotation}\begin{samepage}
\enquote{Un système scientifique est simplement une étape atteinte dans la
recherche indéfiniment continuée de la connaissance. Il est forcément affecté
par l'imperfection inhérente à tout effort humain. Mais reconnaître ces faits ne
signifie pas que la science économique de notre temps soit arriérée. Cela veut
dire seulement qu'elle est chose vivante, et vivre implique à la fois
imperfection et changement.}
\begin{flushright} -- Ludwig von Mises\footnote{Ludwig von Mises, \textit{L'Action Humaine}
\cite{human-action}}
\end{flushright}\end{samepage}\end{quotation}

\newpage

On voit tous passer des articles sur diverses crises financières, en se
demandant comment ces grands sauvetages fonctionnent et nous restons perplexes
face à l'absence de responsable de ces milliers de milliards de dégâts. Je reste
perplexe, mais au moins je commence à entrevoir ce qui se passe dans le monde de
la finance.

Certaines personnes vont jusqu'à attribuer la méconnaissance générale de ces
sujets à une méconnaissance plus systémique et délibérée. Tandis que l'histoire,
la physique, la biologie, les maths et les langues font toutes partie de notre
cursus, l'univers monétaire et financier n'est étonnamment abordé qu'en surface,
voire pas du tout. Je me demande si les gens continueraient d'accroître la dette
autant qu'ils le font si nous étions tous éduqués sur la gestion personnelle et
les rouages de la monnaie et du crédit. Puis je réfléchis à combien de couches
d'aluminium feraient un bon chapeau. Trois, probablement.

\begin{quotation}\begin{samepage}
\enquote{Ces crashs, ces sauvetages, ce ne sont pas des accidents. Et ce n'est 
pas non plus par accident qu'il n'y a pas d'éducation financière à l'école.
[...] C'est prémédité. Comme avant la Guerre Civile où il était illégal
d'instruire un esclave, nous n'avons pas le droit d'étudier la monnaie à
l'école.}
\begin{flushright} -- Robert Kiyosaki\footnote{Robert Kiyosaki, \textit{Pourquoi
les riches deviennent encore plus riches}\cite{robert-kiyosaki}}
\end{flushright}\end{samepage}\end{quotation}

Comme dans Le Magicien d'Oz, on nous demande de ne pas prêter attention à
l'homme derrière le rideau. Contrairement au Magicien d'Oz, nous possédons
maintenant une véritable
sorcellerie\footnote{\url{http://bit.ly/btc-wizardry}} : un réseau de transfert
de valeur, résistant à la censure, ouvert et sans frontières. Il n'y a pas de
rideau et chacun peut en apprécier la
magie\footnote{\url{https://github.com/bitcoin/bitcoin}}.

\paragraph{Bitcoin m'a appris à regarder derrière le rideau et à surmonter ma
méconnaissance financière.}

% ---
%
% #### Down the Rabbit Hole
%
% - [Human Action][Ludwig von Mises] by Ludwig von Mises
% - [Why the Rich are Getting Richer][Robert Kiyosaki] by Robert Kiyosaki
%
% [real wizardry]: https://external-preview.redd.it/8d03MWWOf2HIyKrT8ThBGO4WFv-u25JaYqhbEO9b1Sk.jpg?width=683&auto=webp&s=dc5922d84717c6a94527bafc0189fd4ca02a24bb
% [visible to anyone]: https://github.com/bitcoin/bitcoin
%
% <!-- Wikipedia -->
% [alice]: https://en.wikipedia.org/wiki/Alice%27s_Adventures_in_Wonderland
% [carroll]: https://en.wikipedia.org/wiki/Lewis_Carroll
