\chapter{La valeur}
\label{les:10}

\begin{chapquote}{Lewis Carroll, \textit{Alice au pays des merveilles}}
\enquote{C’était le Lapin Blanc qui revenait en trottant lentement et en jetant
autour de lui des regards inquiets comme s’il avait perdu quelque chose\ldots}
\end{chapquote}

La valeur est en quelque sorte paradoxale et il existe de multiples
théories\footnote{Voir \textit{Théorie de la valeur (économie)} sur Wikipédia
\cite{wiki:theory-of-value}} qui tentent d'expliquer pourquoi nous donnons de la
valeur à certaines choses plutôt qu'à d'autres. Les gens sont conscients de ce
paradoxe depuis des millénaires. Comme Platon l'écrivait dans son dialogue avec
Euthydème, nous estimons certaines choses parce qu'elles sont rares, pas
seulement sur leur nécessité à notre survie.

\begin{quotation}\begin{samepage}
\enquote{Même, pour bien faire, vous avertiriez vos écoliers d'en user de la
sorte, et de n'en parler qu'entre eux ou avec vous ; car la rareté, Euthydème,
met le prix aux choses, et l'eau, comme dit Pindare, se vend à vil prix
quoiqu'elle soit ce qu'il y a de plus précieux.}
\begin{flushright} -- Platon\footnote{Platon, \textit{Euthydème}
\cite{euthydemus}}
\end{flushright}\end{samepage}\end{quotation}

Ce paradoxe de la valeur\footnote{Voir \textit{Paradoxe de l'eau et du diamant}
sur Wikipedia\cite{wiki:paradox-of-value}} montre une chose intéressante à
propos de nous autres, humains : il semblerait que nous estimions les choses sur
une base subjective\footnote{Voir \textit{Conception subjective de la valeur}
sur Wikipedia \cite{wiki:subjective-theory-of-value}}, tout en observant
certains critères raisonnés. Une chose peut nous être \textit{précieuse} pour de
nombreuses raisons, mais celles auxquelles nous accordons de la valeur partagent
certains traits. Si cette chose se copie facilement ou qu'elle est naturellement
abondante, nous ne l'estimons pas.

Apparemment, nous accordons de la valeur à quelque chose par sa rareté (l'or,
les diamants, le temps), sa complexité ou sa quantité de travail nécessaire, son
irremplaçabilité (une vieille photo d'un être cher), son utilité à permettre des
choses autrement impossibles ou encore une combinaison de tout ça, comme les
grandes œuvres d'art.

Bitcoin est tout ça à la fois : il est extrêmement rare (21 millions), de plus
en plus dur à produire (la réduction des récompenses), impossible à remplacer
(une clé privée perdue l'est à jamais) et nous permet de faire des choses plutôt
utiles. C'est vraisemblablement le meilleur outil pour transférer de la valeur
au-delà des frontières, il est virtuellement résistant à la censure et à la
saisie, ce qui permet à n'importe qui de stocker sa valeur sans l'aval des
banques et du gouvernement, pour ne citer qu'eux.

\paragraph{Bitcoin m'a appris que la valeur était subjective, mais pas
arbitraire.}

% ---
%
% #### Down the Rabbit Hole
%
% - [Euthydemus] by Plato
% - [Theory of Value][multiple theories], [Paradox of Value][paradox of value], [Subjective Theory of Value][subjective] on Wikipedia
%
% [Euthydemus]: http://www.perseus.tufts.edu/hopper/text?doc=Perseus:text:1999.01.0178:text=Euthyd.
% [Plato]: http://www.perseus.tufts.edu/hopper/text?doc=plat.+euthyd.+304b
%
% <!-- Wikipedia -->
% [multiple theories]: https://en.wikipedia.org/wiki/Theory_of_value_%28economics%29
% [paradox of value]: https://en.wikipedia.org/wiki/Paradox_of_value
% [subjective]: https://en.wikipedia.org/wiki/Subjective_theory_of_value
% [alice]: https://en.wikipedia.org/wiki/Alice%27s_Adventures_in_Wonderland
% [carroll]: https://en.wikipedia.org/wiki/Lewis_Carroll
