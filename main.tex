% !TEX root = main.tex
% \documentclass{tufte-book}%[a4paper,twoside]
% See https://github.com/Tufte-LaTeX/tufte-latex/blob/master/sample-book.tex for details

% --- AMAZON BEGIN ---
% WITHOUT BLEED
% US Trade => 6x9
\documentclass[paper=6in:9in,pagesize=pdftex,
               headinclude=on,footinclude=on,12pt]{scrbook}
%
% Paper width
% W = 6in
% Paper height
% H = 9in
% Paper gutter
% BCOR = 0.5in
% Margin (0.5in imposed on lulu, recommended on createspace)
% m = 0.5in
% Text height
% h = H - 2m = 8in
% Text width
% w = W - 2m - BCOR = 4.5in
\areaset[0.50in]{4.5in}{8in}
% --- AMAZON END ---

%encoding
%--------------------------------------
\usepackage[T1]{fontenc}
\usepackage[utf8]{inputenc}
%--------------------------------------

% Copyright with title BEGIN
\usepackage{fancyhdr}
\def\secondpage{\clearpage\null\vfill
\pagestyle{empty}
\begin{minipage}[b]{0.9\textwidth}
\normalsize 21 Leçons \newline
\footnotesize Enseignements tirés de ma chute dans le terrier du lapin Bitcoin \par

Seconde édition. Version 0.3.12, git commit \texttt{8e70090}.

\footnotesize\raggedright
\setlength{\parskip}{0.5\baselineskip}
Copyright \copyright 2018--\the\year\ Gigi / \href{https://twitter.com/dergigi}{@dergigi} / \href{https://dergigi.com}{dergigi.com} \par

\footnotesize\raggedright
\setlength{\parskip}{0.5\baselineskip}
Traduit de l'anglais par Antho / \href{https://twitter.com/PrplSknk}{@PrplSknk} \par

\includegraphics[width=2cm]{assets/images/cc-by-sa.pdf}

Ce livre et sa version en ligne sont distribués sous les termes de la licence
Creative Commons Attribution-ShareAlike 4.0. Une copie de référence de cette
licence se trouve sur le site officiel de Creative
Commons.\footnote{\url{https://creativecommons.org/licenses/by-sa/4.0}}

\end{minipage}
\vspace*{2\baselineskip}
\cleardoublepage
\rfoot{\thepage}}

\makeatletter
\g@addto@macro{\maketitle}{\secondpage}
\makeatother
% Copyright with title END

% Use serif font for chapters and parts
\setkomafont{disposition}{\bfseries}
\KOMAoptions{headings=small}

% Packages
\usepackage{setspace}
\usepackage{booktabs}
\usepackage{graphicx}
\setkeys{Gin}{width=\linewidth,totalheight=\textheight,keepaspectratio}
\graphicspath{{graphics/}}

%%
% For Quotes
\usepackage{csquotes}
\renewcommand\mkbegdispquote[2]{\makebox[0pt][r]{\textquotedblleft\,}}
\renewcommand\mkenddispquote[2]{\,\textquotedblright#2}

%%
% Just some sample text
\usepackage{lipsum}

%%
% For nicely typeset tabular material
\usepackage{booktabs}

%%
% Bibliography stuff: Biber, BibTex, BibLatex
%\usepackage[autostyle]{csquotes}
% \usepackage[
    % backend=biber,
    % style=authoryear-icomp,
    % sortlocale=de_DE,
    % natbib=true,
    % url=false,
    % doi=true,
    % eprint=false
% ]{biblatex}
% \usepackage[backend=biber]{biblatex}
\usepackage{url}
\usepackage{natbib}
\bibliographystyle{plain}

%French-specific commands
%--------------------------------------
\usepackage[french]{babel}
%--------------------------------------

%%
% Hyperlinks
\usepackage[hidelinks]{hyperref}

%%
% For graphics / images
\usepackage{caption}
\usepackage{graphicx}
\setkeys{Gin}{width=\linewidth,totalheight=\textheight,keepaspectratio}
\graphicspath{{graphics/}}

% The fancyvrb package lets us customize the formatting of verbatim
% environments.  We use a slightly smaller font.
\usepackage{fancyvrb}
\fvset{fontsize=\normalsize}

%%
% Prints argument within hanging parentheses (i.e., parentheses that take
% up no horizontal space).  Useful in tabular environments.
\newcommand{\hangp}[1]{\makebox[0pt][r]{(}#1\makebox[0pt][l]{)}}

%%
% Prints an asterisk that takes up no horizontal space.
% Useful in tabular environments.
\newcommand{\hangstar}{\makebox[0pt][l]{*}}

%%
% Prints a trailing space in a smart way.
\usepackage{xspace}

% Prints the month name (e.g., January) and the year (e.g., 2008)
\newcommand{\monthyear}{%
  \ifcase\month\or January\or February\or March\or April\or May\or June\or
  July\or August\or September\or October\or November\or
  December\fi\space\number\year
}


% Prints an epigraph and speaker in sans serif, all-caps type.
\newcommand{\openepigraph}[2]{%
  %\sffamily\fontsize{14}{16}\selectfont
  \begin{fullwidth}
  \sffamily\large
  \begin{doublespace}
  \noindent\allcaps{#1}\\% epigraph
  \noindent\allcaps{#2}% author
  \end{doublespace}
  \end{fullwidth}
}

% Inserts a blank page
\newcommand{\blankpage}{\newpage\hbox{}\thispagestyle{empty}\newpage}

\usepackage{units}

% Typesets the font size, leading, and measure in the form of 10/12x26 pc.
\newcommand{\measure}[3]{#1/#2$\times$\unit[#3]{pc}}

% Macros for typesetting the documentation
\newcommand{\hlred}[1]{\textcolor{Maroon}{#1}}% prints in red
\newcommand{\hangleft}[1]{\makebox[0pt][r]{#1}}
\newcommand{\hairsp}{\hspace{1pt}}% hair space
\newcommand{\hquad}{\hskip0.5em\relax}% half quad space
\newcommand{\TODO}{\textcolor{red}{\bf TODO!}\xspace}
\newcommand{\na}{\quad--}% used in tables for N/A cells
\providecommand{\XeLaTeX}{X\lower.5ex\hbox{\kern-0.15em\reflectbox{E}}\kern-0.1em\LaTeX}
\newcommand{\tXeLaTeX}{\XeLaTeX\index{XeLaTeX@\protect\XeLaTeX}}
% \index{\texttt{\textbackslash xyz}@\hangleft{\texttt{\textbackslash}}\texttt{xyz}}
\newcommand{\tuftebs}{\symbol{'134}}% a backslash in tt type in OT1/T1
\newcommand{\doccmdnoindex}[2][]{\texttt{\tuftebs#2}}% command name -- adds backslash automatically (and doesn't add cmd to the index)
\newcommand{\doccmddef}[2][]{%
  \hlred{\texttt{\tuftebs#2}}\label{cmd:#2}%
  \ifthenelse{\isempty{#1}}%
    {% add the command to the index
      \index{#2 command@\protect\hangleft{\texttt{\tuftebs}}\texttt{#2}}% command name
    }%
    {% add the command and package to the index
      \index{#2 command@\protect\hangleft{\texttt{\tuftebs}}\texttt{#2} (\texttt{#1} package)}% command name
      \index{#1 package@\texttt{#1} package}\index{packages!#1@\texttt{#1}}% package name
    }%
}% command name -- adds backslash automatically
\newcommand{\doccmd}[2][]{%
  \texttt{\tuftebs#2}%
  \ifthenelse{\isempty{#1}}%
    {% add the command to the index
      \index{#2 command@\protect\hangleft{\texttt{\tuftebs}}\texttt{#2}}% command name
    }%
    {% add the command and package to the index
      \index{#2 command@\protect\hangleft{\texttt{\tuftebs}}\texttt{#2} (\texttt{#1} package)}% command name
      \index{#1 package@\texttt{#1} package}\index{packages!#1@\texttt{#1}}% package name
    }%
}% command name -- adds backslash automatically
\newcommand{\docopt}[1]{\ensuremath{\langle}\textrm{\textit{#1}}\ensuremath{\rangle}}% optional command argument
\newcommand{\docarg}[1]{\textrm{\textit{#1}}}% (required) command argument
\newenvironment{docspec}{\begin{quotation}\begin{samepage}\ttfamily\parskip0pt\parindent0pt\ignorespaces}{\end{flushright}\end{samepage}\end{quotation}}% command specification environment
\newcommand{\docenv}[1]{\texttt{#1}\index{#1 environment@\texttt{#1} environment}\index{environments!#1@\texttt{#1}}}% environment name
\newcommand{\docenvdef}[1]{\hlred{\texttt{#1}}\label{env:#1}\index{#1 environment@\texttt{#1} environment}\index{environments!#1@\texttt{#1}}}% environment name
\newcommand{\docpkg}[1]{\texttt{#1}\index{#1 package@\texttt{#1} package}\index{packages!#1@\texttt{#1}}}% package name
\newcommand{\doccls}[1]{\texttt{#1}}% document class name
\newcommand{\docclsopt}[1]{\texttt{#1}\index{#1 class option@\texttt{#1} class option}\index{class options!#1@\texttt{#1}}}% document class option name
\newcommand{\docclsoptdef}[1]{\hlred{\texttt{#1}}\label{clsopt:#1}\index{#1 class option@\texttt{#1} class option}\index{class options!#1@\texttt{#1}}}% document class option name defined
\newcommand{\docmsg}[2]{\bigskip\begin{fullwidth}\noindent\ttfamily#1\end{fullwidth}\medskip\par\noindent#2}
\newcommand{\docfilehook}[2]{\texttt{#1}\index{file hooks!#2}\index{#1@\texttt{#1}}}
\newcommand{\doccounter}[1]{\texttt{#1}\index{#1 counter@\texttt{#1} counter}}

% Generates the index
\usepackage{makeidx}
\makeindex

%%
% Chapter/Lesson Quotes
\makeatletter
\renewcommand{\@chapapp}{}% Not necessary...
\newenvironment{chapquote}[2][4em]
  {\setlength{\@tempdima}{#1}%
   \def\chapquote@author{#2}%
   \parshape 1 \@tempdima \dimexpr\textwidth-2\@tempdima\relax%
   \itshape}
  {\par\normalfont\hfill--\ \chapquote@author\hspace*{\@tempdima}\par\bigskip}
\makeatother

%%%%%%%%%%%%%%%%%%%%%%%%%%%%%%%%%%%%%%%%%%%%%%%%%%%%%%%%%%%%%%%%%%%%%%%%%%%%%%%%
%                                   DOCUMENT
%%%%%%%%%%%%%%%%%%%%%%%%%%%%%%%%%%%%%%%%%%%%%%%%%%%%%%%%%%%%%%%%%%%%%%%%%%%%%%%%

\begin{document}

\frontmatter

\title{21 Leçons}
\subtitle{Enseignements tirés de ma chute dans le terrier du lapin Bitcoin}
\author{Gigi}
\date{}

\maketitle

\cleardoublepage


\newpage \vspace*{8cm}
% Sets a PDF bookmark for the dedication
\pdfbookmark{Dedication}{dedication}
\thispagestyle{empty}
\begin{center}
  \Large \emph{
  Dédié à ma femme, à mon enfant, ainsi qu'à tous les enfants du monde. Puisse
  Bitcoin vous soutenir et vous apporter la vision d'un futur qui vaille la
  peine de s'engager.
  }
\end{center}

\chapter*{Avant-propos}
\pdfbookmark{Foreword}{foreword}

Certains appellent ça une expérience mystique. D'autres appellent ça Bitcoin.

J'ai rencontré Gigi pour la première fois dans un de mes foyers spirituels --
Riga, Lettonie -- patrie de la conférence \textit{The Baltic Honeybadger}, où
les plus fervents fidèles de Bitcoin accomplissent un pèlerinage annuel. Après
une profonde conversation autour d'un déjeuner, le lien que Gigi et moi avions
tissé était aussi immuable qu'une transaction Bitcoin traitée quelques heures
plus tôt lorsque nous nous sommes salués.

Mon autre foyer spirituel, Christ Church à Oxford où j'ai eu le privilège
d'étudier pour mon MBA, est le lieu où m'est apparue la révélation
\enquote{terrier du lapin}. Comme Gigi, j'ai transcendé les sphères économiques,
techniques et sociales afin de laisser Bitcoin m'envelopper spirituellement.
Après avoir \enquote{acheté haut} pendant la bulle de novembre 2013, j'ai dû
tirer des enseignements très difficiles de l'interminable et destructeur marché
baisser de trois ans qui s'en suivit. Ces 21 leçons m'auraient particulièrement
bien aidé à ce moment-là. La plupart sont simplement des vérités naturelles qui,
pour le néophyte, sont assombries par un film opaque et fragile. Cependant,
d'ici la fin de ce livre, cette façade volera en éclats.

Par une nuit très claire de la fin août 2016 à Oxford, quelques semaines
seulement après le piratage de la plateforme d'échange Bitfinex, j'ai fait une
halte contemplative au Master's Garden de Christ Church. C'était une période
compliquée et j'étais sur le point de craquer psychologiquement et
émotionnellement après ce qui m'a paru une éternité de torture. Pas pour les
pertes financières, non, mais bien à cause du vide spirituel écrasant que je
ressentais, isolé dans ma vision du monde. Si seulement un livre comme celui-ci
avait existé à l'époque, j'aurais pu me rendre compte que je n'étais pas seul.
Le Master’s Garden est un endroit particulier à mes yeux et aux yeux de beaucoup
avant moi au cours des siècles. C'est ici qu'un certain Charles Dodgson,
professeur de mathématiques à Christ Church, remarqua l'une de ses jeunes
élèves, Alice Liddell, fille du doyen. Dodgson, plus connu sous son nom de plume
Lewis Carroll, s'est inspiré d'Alice et du Master's Garden ; et par la magie de
ce vénérable gazon, j'ai plongé mon regard dans le crypto-abîme, qui me l'a
ardemment rendu, étouffant toute arrogance, giflant mon orgueil en plein visage.
J'étais enfin en paix.

21 Leçons vous embarque pour un véritable voyage vers Bitcoin, non seulement
philosophique, technologique et économique, mais aussi spirituel.

En se plongeant plus profondément dans la philosophie sobrement exposée dans 7
des 21 Leçons, avec assez de temps et de réflexion, il est possible d'aller
jusqu'à comprendre l'origine de toute chose. Ses 7 leçons sur l'économie rendent
compte, en des termes simples, de la façon dont nous sommes à la merci d'un
petit groupe de chapeliers fous et comment ils ont réussi à nous mettre des
œillères dans la tête, dans le cœur et à l'âme. Les 7 leçons sur la technologie
décrivent la beauté et la perfection technologiquement darwinienne de Bitcoin.
En tant que bitcoiner non technique, ces leçons apportent une étude pertinente
sur la nature fondamentalement technologique de Bitcoin et, de fait, sur la
nature de la technologie elle-même.

Dans cette expérience éphémère que nous appelons la vie, nous vivons, nous
aimons et nous apprenons. Mais qu'est-ce que la vie sinon une suite
chronologique d'événements ?

Parvenir au sommet de la montagne Bitcoin n'est pas chose aisée. C'est truffé de
faux sommets, le terrain est très accidenté et les crevasses sont omniprésentes,
prêtes à vous engloutir. Après la lecture de ce livre, vous comprendrez que Gigi
est le sherpa Bitcoin ultime. Je lui en serai toujours reconnaissant.

\begin{flushright}
  Hass McCook \\
  29 novembre 2019
\end{flushright}


\newpage \vspace*{4cm}
\thispagestyle{empty}
\begin{quotation}
\begin{center}
  \large
  \enquote{Voudriez-vous me dire, s’il vous plaît, quel chemin je dois prendre
  pour m’en aller d’ici ?} \\~\\
  \enquote{Cela dépend beaucoup de l’endroit où tu veux aller.} \\~\\
  \enquote{Peu m’importe l’endroit…} \\~\\
  \enquote{En ce cas, peu importe la route que tu prendras.}
\end{center}
\begin{flushright} -- Lewis Carroll, \textit{Alice au pays des merveilles}\end{flushright}
\end{quotation}

\tableofcontents


\def\bitcoinB{\leavevmode
  {\setbox0=\hbox{\textsf{B}}%
    \dimen0\ht0 \advance\dimen0 0.2ex
    \ooalign{\hfil \box0\hfil\cr
      \hfil\vrule height \dimen0 depth.2ex\hfil\cr
    }%
  }%
}

\chapter*{À propos de ce livre \\ (... et de son auteur)}
\pdfbookmark{À propos de ce livre (... et de son auteur)}{about}

Il s'agit d'un livre un peu particulier. Mais bon, Bitcoin est aussi une
technologie un peu particulière, donc un livre particulier à propos de Bitcoin
est sans doute adapté. Je ne sais pas vraiment si je suis un mec particulier
(j'aime bien penser que je suis un mec \textit{normal}) mais l'histoire de ce
livre, et de comment j'en suis venu à devenir auteur, mérite d'être racontée.

Premièrement, je ne suis pas auteur. Je suis ingénieur. Je n'ai pas étudié
les lettres. J'ai appris le code et comment coder. Deuxièmement, je n'ai jamais
eu l'intention d'écrire un livre, encore moins un livre sur Bitcoin. Bon sang,
ce n'est même pas ma langue maternelle.\footnote{La raison pour laquelle
j'écris ce livre en anglais, c'est que mon cerveau fonctionne d'une manière
bizarre. Dès que ça devient technique, il passe tout seul à l'anglais.} Je suis
juste un gars qui a attrapé le virus Bitcoin. Gravement.

Qui suis-\textit{je} alors pour écrire un livre sur Bitcoin ? Bonne question. En
bref, la réponse est simple : je suis Gigi et je suis un bitcoiner.

Mais le développement est un peu plus nuancé.

\paragraph{}
Je viens de l'informatique et du développement logiciel. Dans une vie
antérieure, j'étais dans une équipe de recherche qui tentait d'apprendre à
penser et à réfléchir à des ordinateurs, entre autres choses. Dans une vie
encore plus ancienne, j'écrivais des logiciels de traitement automatisé de
passeports et d'autres trucs du même style, ce qui est encore plus effrayant. Je
m'y connais un peu en informatique et en réseaux, donc j'imagine que j'ai
quelques longueurs d'avance pour comprendre l'aspect technique de Bitcoin. En
revanche, comme j'essaie de le souligner dans ce livre, cet aspect technique ne
représente qu'une petite partie de l'animal qu'est Bitcoin. Et chacune de ses
parties est importante.

Ce livre a vu le jour grâce à une seule question toute bête :
\textit{\enquote{Qu'avez-vous appris de Bitcoin ?}}. J'ai tenté d'y répondre
d'un simple tweet. Puis le tweet est devenu tempête de tweets. Cette tempête
s'est transformée en article. L'article a évolué en trois articles. Trois
articles sont devenus 21 leçons. Et 21 leçons ont engendré ce livre. Du coup, je
suppose que je suis juste nul pour résumer ma pensée en un seul tweet.

\paragraph{}
\textit{\enquote{Pourquoi écrire ce livre ?}}, me direz-vous. À nouveau, deux
réponses : une courte et une longue. La courte, c'est que je devais le faire.
J'étais (et suis toujours) \textit{possédé} par Bitcoin. Il ne cesse de me
fasciner. Je ne peux m'arrêter de penser à lui et aux implications qu'il aura
dans nos sociétés. La réponse longue, c'est que je crois que Bitcoin est
l'invention la plus importante de notre époque et que la nature de cette
invention doit être comprise par le plus grand nombre. Bitcoin reste l'un des
phénomènes les plus mal compris du monde actuel et ça m'a pris des années pour
réaliser pleinement le sérieux de cette technologie extraterrestre. Comprendre
ce qu'est Bitcoin et comment il va transformer nos sociétés est une expérience
marquante. J'ai l'espoir de faire germer dans votre tête les graines qui
pourraient vous conduire à cette prise de conscience.

Dans l'ordre des choses, bien que ce passage soit intitulé \enquote{\textit{À
propos de ce livre (... et de son auteur)}}, ce livre, qui je suis et ce que
j'ai fait n'ont pas vraiment d'importance. Je suis juste un nœud du réseau, à la
fois littéralement \textit{et} métaphoriquement. De toute façon, vous ne devriez
pas croire ce que je dis. Comme nous, bitcoiners, aimons le répéter : faites vos
propres recherches. Et par-dessus tout : ne vous fiez pas, vérifiez.

J'ai fait mes recherches au mieux afin de vous permettre, cher lecteur, de vous
plonger dans de nombreuses ressources. En plus des notes et des citations de ce
livre, j'essaie de garder à jour une liste de contenu sur
\href{https://21lessons.com/rabbithole}{21lessons.com/rabbithole} et sur
\href{https://bitcoin-resources.com}{bitcoin-resources.com}, qui recense
également plein d'autres morceaux choisis, livres, podcasts, qui vous aideront à
comprendre ce qu'est Bitcoin.

\paragraph{}
En résumé, c'est juste un livre qui parle de Bitcoin, écrit par un bitcoiner.
Bitcoin n'a pas besoin de ce livre, et vous n'avez sans doute pas besoin de ce
livre pour comprendre Bitcoin. Je pense que vous comprendrez Bitcoin dès que
\textit{vous} serez prêt et je crois aussi que vos premières fractions d'un
bitcoin vous trouveront dès que vous serez prêt à les recevoir. Par essence,
chacun obtiendra \bitcoinB{}itcoin exactement au bon moment. Dans l'intervalle,
Bitcoin existe et c'est bien suffisant.\footnote{Beautyon, \textit{Bitcoin is.
And that is enough.}~\cite{bitcoin-is}}

\chapter*{Préface}

S'enfoncer dans le terrier du lapin Bitcoin est une expérience bizarre. Comme
tant d'autres, j'ai l'impression que ces deux dernières années passées à étudier
Bitcoin m'ont plus appris que deux décennies d'éducation classique.

Ces leçons forment la quintessence de ce que j'ai découvert. D'abord publié
comme une série d'articles intitulée \textit{« Ce que Bitcoin m'a enseigné »},
ce qui suit peut être vu comme la troisième édition de la série d'origine.

À l'instar de Bitcoin, ces leçons sont évolutives. Je compte revenir dessus
régulièrement, en publiant plus tard des mises à jour et du contenu
supplémentaire.

Contrairement à Bitcoin, les futures versions de ce projet ne seront pas
nécessairement rétro-compatibles. Certaines leçons pourront être complétées,
d'autres seront retravaillées voire même remplacées.

Bitcoin est un professeur intarissable, c'est pour cela que je ne considère pas
ces leçons comme exhaustives ou définitives. Elles sont le reflet de mon propre
périple au cœur du terrier. Il existe bien d'autres leçons à tirer, de fait
chaque personne qui entrera dans le monde de Bitcoin en retirera des
connaissances différentes.

J'espère que vous trouverez une utilité à ces leçons et que leur apprentissage
par la lecture vous paraîtra moins pénible et douloureux que je ne l'ai parfois
vécu par l'expérience.

% <!-- Internal -->
% [I]: 
%
% <!-- Twitter -->
% [dergigi]: https://twitter.com/dergigi
%
% <!-- Wikipedia -->
% [alice]: https://en.wikipedia.org/wiki/Alice%27s_Adventures_in_Wonderland
% [carroll]: https://en.wikipedia.org/wiki/Lewis_Carroll

%%
% Start the main matter (normal chapters)
\mainmatter

\part*{21 Leçons}

\newpage \vspace*{8cm}
\thispagestyle{empty}
\begin{quotation}
\begin{center}
  \large
  \enquote{Ma pauvre Alice, ce que tu peux être sotte !} se répondit-elle.
  \enquote{Comment pourrais-tu apprendre des leçons ici ? C’est tout juste s’il
  y a assez de place pour toi, et il n’y en a pas du tout pour un livre de classe !}
\end{center}
\begin{flushright} -- Lewis Carroll, \textit{Alice au pays des merveilles}\end{flushright}
\end{quotation}

\chapter*{Introduction}
\label{ch:introduction}

\begin{chapquote}{Lewis Carroll, \textit{Alice au pays des merveilles}}
\enquote{Mais je ne veux pas aller parmi les fous,} fit remarquer Alice.
\enquote{Impossible de faire autrement,} dit le Chat ; \enquote{nous sommes tous
fous ici. Je suis fou. Tu es folle.} \enquote{Comment savez-vous que je suis
folle ?} demanda Alice. \enquote{Tu dois l’être,} répondit le Chat,
\enquote{autrement tu ne serais pas venue ici.}
\end{chapquote}

En octobre 2018, Arjun Balaji posait cette question innocente :
\textit{Qu'avez-vous appris de Bitcoin ?} Après avoir essayé d'y répondre en un
court tweet, et avoir lamentablement échoué, j'ai compris que ce que j'avais
retenu était bien trop riche pour répondre en quelques mots, voire même répondre
tout court.

Ces connaissances que j'ai acquises, évidemment, concernent Bitcoin - ou tout du
moins lui sont liées. Cependant, bien que certains rouages de Bitcoin soient
expliqués ici, les leçons qui suivent ne sont pas une justification du
fonctionnement ou de la nature de Bitcoin. Elles pourront néanmoins aider à
explorer certains aspects satellites de Bitcoin comme les questions
philosophiques, les réalités économiques ou les innovations technologiques.

\begin{center}
  \includegraphics[width=7cm]{assets/images/the-tweet.png}
\end{center}

Les \textit{21 leçons} sont groupées par sept, formant ainsi trois chapitres.
Chacun de ces chapitres observe Bitcoin sous une lumière précise, récoltant les
enseignements à tirer de l'examen sous différents angles de cet étrange réseau.

\paragraph{Le \hyperref[ch:philosophy]{chapitre 1}}{explore les enseignements
philosophiques de Bitcoin. L'interaction entre immutabilité et changement, le
concept de rareté véritable, l'Immaculée Conception de Bitcoin, le problème de
l'identité, la contradiction entre réplication et localité, la force de la
liberté d'expression et les limites du savoir.}

\paragraph{Le \hyperref[ch:economics]{chapitre 2}}{s'intéresse aux enseignements
économiques de Bitcoin. Des leçons sur la méconnaissance financière,
l'inflation, la valeur, l'argent, son histoire, les réserves fractionnaires
des banques ainsi que la manière dont Bitcoin réintroduit la monnaie saine d'une
façon rusée et détournée.}

\paragraph{Le \hyperref[ch:technology]{chapitre 3}}{présente certaines leçons
acquises en examinant la technologie de Bitcoin. Pourquoi les nombres
renferment-ils une force, des remarques sur la confiance, pourquoi donner
l'heure demande du travail, comment une progression lente et prudente est une
fonctionnalité et pas un bug, ce que l'invention de Bitcoin peut nous apprendre
sur la vie privée, pourquoi les cypherpunks écrivent-ils du code (et non des
lois) et quelles métaphores pourraient être utiles pour imaginer l'avenir de
Bitcoin.}

~

Chaque leçon contient plusieurs citations et liens au fil du texte. Si une idée
vaut la peine d'être creusée, vous pouvez suivre les liens vers le contenu
pertinent dans les notes de bas de page ou la bibliographie.

Bien que quelques connaissances préalables sur Bitcoin puissent aider, j'ai bon
espoir que ces leçons pourront être assimilées par tout lecteur curieux. Malgré
les liens qui peuvent exister entre elles, chaque leçon devrait se suffire à
elle-même et pouvoir être lue indépendamment. J'ai accordé une attention
particulière à éviter le jargon technique, malgré cela quelques termes
spécifiques à certains domaines restent inévitables.

Je souhaite que mon récit puisse donner l'envie à d'autres personnes de gratter
le vernis et d'examiner certaines des questions les plus profondes qu'amène
Bitcoin. Ma propre inspiration émane d'une multitude d'auteurs et de créateurs
de contenu à qui je voue une éternelle gratitude.

Enfin, et surtout : en écrivant tout ceci mon but n'est pas de vous convaincre
de quoi que ce soit. Mon but est de vous amener à penser, de vous montrer que
Bitcoin représente bien plus que ce que l'on croit. Je ne peux même pas vous
dire ce qu'est Bitcoin ou ce qu'il va vous apprendre. Vous allez devoir le
découvrir par vous-même.

\begin{quotation}\begin{samepage}
\enquote{C'est ta dernière opportunité. Tu ne pourras pas rebrousser chemin. Si
tu choisis la bleue, tout s'arrête. Tu te réveilles dans ton lit et tu crois ce
que bon te semble. Si tu prends la rouge\footnote{la \textit{orange}}, tu restes
aux Pays des Merveilles et je t'emmène au tréfonds du terrier.}
\begin{flushright} -- Morpheus
\end{flushright}\end{samepage}\end{quotation}

\begin{figure}
  \includegraphics{assets/images/bitcoin-orange-pill.jpg}
  \caption*{Souvenez-vous : je n'offre que la vérité. Rien de plus.}
  \label{fig:bitcoin-orange-pill}
\end{figure}

%
% [Morpheus]: https://en.wikipedia.org/wiki/Red_pill_and_blue_pill#The_Matrix_(1999)
% [this question]: https://twitter.com/arjunblj/status/1050073234719293440
%
% <!-- Internal -->
% [chapter1]: {{ 'bitcoin/lessons/ch1-00-philosophy' | absolute_url }}
% [chapter2]: {{ 'bitcoin/lessons/ch2-00-economics' | absolute_url }}
% [chapter3]: {{ 'bitcoin/lessons/ch3-00-technology' | absolute_url }}
%
% <!-- Wikipedia -->
% [alice]: https://en.wikipedia.org/wiki/Alice%27s_Adventures_in_Wonderland
% [carroll]: https://en.wikipedia.org/wiki/Lewis_Carroll

\part{Philosophie}
\label{ch:philosophy}
\chapter*{Philosophie}

\begin{chapquote}{Lewis Carroll, \textit{Alice au pays des merveilles}}
  La Souris la regarda avec curiosité (Alice crut même la voir cligner l’un de
  ses petits yeux), mais elle ne répondit rien.
\end{chapquote}

Si l'on regarde Bitcoin en surface, on pourrait conclure qu'il est lent,
inefficace, inutilement redondant et excessivement paranoïaque. Si l'on observe
Bitcoin d'un esprit curieux, on pourrait bien découvrir que les choses ne sont
pas ce qu'elles paraissent au premier coup d'œil.

Bitcoin a le chic pour mettre vos présomptions sens dessus dessous.
Régulièrement, juste au moment où vous alliez retrouver votre zone de confort,
Bitcoin viendra à nouveau fracasser vos certitudes comme un éléphant dans un
magasin de porcelaine.

\begin{figure}
  \includegraphics{assets/images/blind-monks.jpg}
  \caption{Moines aveugles examinant le taureau Bitcoin}
  \label{fig:blind-monks}
\end{figure}

Bitcoin est l'enfant de nombreuses disciplines. Tout comme des moines aveugles
examinant un éléphant, chaque personne qui approche cette nouvelle technologie
le fait sous un angle particulier. Par conséquent, chacun arrivera à différentes
conclusions sur la nature de la bête.

Les leçons qui suivent présentent certaines idées préconçues que Bitcoin a
fracassées ainsi que les conclusions auxquelles je suis arrivé. Des questions
philosophiques à propos de l'immutabilité, de la rareté, de la localité et de
l'identité sont abordées au cours des quatre premières leçons. Chaque partie se
compose de sept leçons.

~

\begin{samepage}
Partie~\ref{ch:philosophy} -- Philosophie:

\begin{enumerate}
  \item Immutabilité et changement
  \item La rareté de la rareté
  \item Réplication et localité
  \item Le problème de l'identité
  \item L'Immaculée Conception
  \item La force de la liberté d'expression
  \item Les limites du savoir
\end{enumerate}
\end{samepage}

La leçon \ref{les:5} s'intéresse à la façon dont l'histoire de Bitcoin est non
seulement fascinante mais aussi absolument essentielle à un système sans
responsables. Les deux dernières leçons de ce chapitre couvriront la force de la
liberté d'expression et les limites de notre savoir individuel, auxquelles la
profondeur étonnante du terrier du lapin Bitcoin fait écho.

J'espère que vous trouverez l'univers Bitcoin aussi pédagogique, fascinant et
amusant que je l'ai trouvé et que je le trouve encore. Je vous invite à suivre
le lapin blanc et à explorer les tréfonds du terrier. Maintenant, accrochez-vous
à votre montre à gousset, sautez et profitez de la descente.

\chapter{Immutabilité et changement}
\label{les:1}

\begin{chapquote}{Alice}
\enquote{Je me demande si on m’a changée pendant la nuit ? Voyons,
réfléchissons : est-ce que j’étais bien la même quand je me suis levée ce
matin ? Je crois me rappeler que je me suis sentie un peu différente. Mais, si
je ne suis pas la même, la question qui se pose est la suivante : Qui diable
puis-je bien être ? Ah, c’est là le grand problème !}
\end{chapquote}

Bitcoin est fondamentalement difficile à décrire. C'est un \textit{truc
nouveau}, donc chaque tentative de comparaison à des concepts antérieurs -- y
compris l'appeler or numérique ou Internet de l'argent -- est condamnée à ne pas
pouvoir rendre compte de son entièreté. Quelle que soit votre analogie favorite,
il y a deux aspects de Bitcoin réellement essentiels : la décentralisation et
l'immutabilité.

\paragraph{}
On peut voir Bitcoin comme un contrat social automatisé\footnote{Hasu, Unpacking
Bitcoin's Social Contract~\cite{social-contract}}. Le logiciel est juste une des
pièces du puzzle, de sorte que vouloir changer Bitcoin en changeant le logiciel
est absolument futile. Il faudrait pour cela convaincre l'ensemble du réseau
d'adopter les changements, ce qui tient plus de l'effort psychologique que de
l'effort d'ingénierie.

\paragraph{}
Ce qui suit peut sembler absurde au départ, comme tant d'autres choses dans ce
domaine, mais je crois fermement en la vérité de cette maxime : ce n'est pas
vous qui changerez Bitcoin, c'est Bitcoin qui vous changera.

\begin{quotation}\begin{samepage}
\enquote{Bitcoin nous changera plus que nous ne le changerons.}
\begin{flushright} -- Marty Bent\footnote{Tales From the Crypt~\cite{tftc21}}
\end{flushright}\end{samepage}\end{quotation}

Ça m'a pris un bon moment pour comprendre la profondeur de cette phrase. Après
tout, comme Bitcoin est juste un logiciel et qu'il est entièrement libre, on
peut simplement changer les choses à volonté, non ? Faux. \textit{Totalement}
faux. Sans surprise, l'inventeur de Bitcoin le savait parfaitement.

\begin{quotation}\begin{samepage}
\enquote{La nature de Bitcoin est telle qu'une fois sortie la version 0.1, les
concepts essentiels étaient gravés dans le marbre pour le restant de ses jours.}
\begin{flushright} -- Satoshi Nakamoto\footnote{Message du forum BitcoinTalk :
`Re: Transactions and Scripts\ldots'~\cite{satoshi-set-in-stone}}
\end{flushright}\end{samepage}\end{quotation}

De nombreuses personnes ont tenté de modifier la nature de Bitcoin. Elles ont
toutes échoué jusqu'à présent. Bien qu'il existe une étendue infinie de forks et
d'altcoins, le réseau Bitcoin continue sa route, exactement comme à la mise en
ligne du prenier nœud. Les altcoins n'importent pas sur le long terme. Les forks
finiront par mourir de faim. Ce qui importe c'est Bitcoin. Tant que notre
compréhension fondamentale des mathématiques et/ou de la physique ne change pas,
le ratel Bitcoin continuera de s'en moquer.

\begin{quotation}\begin{samepage}
\enquote{Bitcoin est le premier exemple d'une nouvelle forme de vie. Il vit et
respire sur internet. Il vit car il est capable de payer des gens pour le
maintenir en vie. [\ldots] Il ne peut être changé. On ne peut le contredire. On
ne peut l'altérer. On ne peut le corrompre. On ne peut l'arrêter. [\ldots] Si
une guerre nucléaire détruisait la moitié de la planète, il continuerait à
vivre, intact.}
\begin{flushright} -- Ralph Merkle\footnote{DAOs, Democracy and
Governance,~\cite{merkle-dao}}
\end{flushright}\end{samepage}\end{quotation}

Le cœur de Bitcoin battra plus longtemps que tous les nôtres.

~

Comprendre tout ça m'a fait changer bien plus que ne le feront les précédents
blocs de Bitcoin. Ma préférence temporelle a été modifiée, ma compréhension de
l'économie, mes opinions politiques et bien plus encore. Mince, ça change même
le régime alimentaire des gens\footnote{Inside the World of the Bitcoin
Carnivores,~\cite{carnivores}}. Si tout ça vous semble dingue, vous êtes au bon
endroit. Tout ceci est dingue en effet ; et c'est pourtant ce qui se passe.

~

\paragraph{Bitcoin m'a appris qu'il ne changerait pas. C'est moi qui changerai.}

% ---
%
% #### Through the Looking-Glass
%
% - [Bitcoin's Gravity: How idea-value feedback loops are pulling people in][gravity]
% - [Lesson 18: Move slowly and don't break things][lesson18]
%
% #### Down the Rabbit Hole
%
% - [Unpacking Bitcoin's Social Contract][automated social contract]: A framework for skeptics by Hasu
% - [DAOs, Democracy and Governance][Ralph Merkle] by Ralph C. Merkle
% - [Marty's Bent][bent]: A daily newsletter highlighting signal in Bitcoin by Marty Bent
% - [Technical Discussion on Bitcoin's Transactions and Scripts][Satoshi Nakamoto] by Satoshi Nakamoto, Gavin Andresen, and others
% - [Inside the World of the Bitcoin Carnivores][carnivores]: Why a small community of Bitcoin users is eating meat exclusively by Jordan Pearson
% - [Tales From the Crypt][tftc] hosted by Marty Bent
%
% <!-- Internal -->
% [gravity]: 
% [lesson18]: {{ 'bitcoin/lessons/ch3-18-move-slowly-and-dont-break-things' | absolute_url }}
%
% <!-- Further Reading -->
% [automated social contract]: https://medium.com/@hasufly/bitcoins-social-contract-1f8b05ee24a9
% [carnivores]: https://motherboard.vice.com/en_us/article/ne74nw/inside-the-world-of-the-bitcoin-carnivores
% [tftc]: https://tftc.io/tales-from-the-crypt/
% [bent]: https://tftc.io/martys-bent/
%
% <!-- Quotes -->
% [Ralph Merkle]: http://merkle.com/papers/DAOdemocracyDraft.pdf
% [Satoshi Nakamoto]: https://bitcointalk.org/index.php?topic=195.msg1611#msg1611
%
% <!-- Twitter People -->
% [Marty Bent]: https://twitter.com/martybent
%
% <!-- Wikipedia -->
% [alice]: https://en.wikipedia.org/wiki/Alice%27s_Adventures_in_Wonderland
% [carroll]: https://en.wikipedia.org/wiki/Lewis_Carroll


\chapter{La rareté de la rareté}
\label{les:2}

\begin{chapquote}{Alice}
\enquote{Cela suffit comme cela\ldots J’espère que je ne grandirai plus\ldots}
\end{chapquote}

Généralement, le progrès technologique semble rendre les choses plus abondantes.
Ce qui était auparavant un produit de luxe devient accessible à de plus en plus
de gens. Bientôt, nous vivrons tous comme des rois. C'est déjà le cas pour la
plupart d'entre nous. Comme l'écrivait Peter Diamandis dans
Abundance~\cite{abundance} : \enquote{La technologie est un mécanisme de
libération des ressources. Elle peut rendre abondant ce qui était rare.}

Bitcoin, en tant que technologie avancée, casse cette tendance et crée une
nouvelle ressource authentiquement rare. Certains avancent même que c'est l'une
des ressources les plus rares de l'univers. L'offre ne peut pas grossir, quels
que soient les efforts déployés pour y parvenir.

\begin{quotation}\begin{samepage}
\enquote{Il n'y a que deux choses véritablement rares : le temps et Bitcoin.}
\begin{flushright} -- Saifedean Ammous\footnote{Présentation sur The Bitcoin
Standard~\cite{bitcoinstandard-pres}}
\end{flushright}\end{samepage}\end{quotation}

Paradoxalement, ceci se produit par un mécanisme de réplication. Les
transactions sont diffusées, les blocs se propagent, le registre distribué est
-- vous l'avez deviné -- distribué. Mais ce ne sont que des mots savants pour
désigner la copie. Bon sang, Bitcoin se réplique même tout seul sur autant
d'ordinateurs que possible, en incitant les gens à exécuter des nœuds complets
et à miner de nouveaux blocs.

Toute cette réplication œuvre magnifiquement de concert en vue de produire de la
rareté.

\paragraph{}

\paragraph{En ces temps d'abondance, Bitcoin m'a appris ce qu'était la véritable
rareté.}

% ---
%
% #### Through the Looking-Glass
%
% - [Lesson 14: Sound money][lesson14]
%
% #### Down the Rabbit Hole
%
% - [The Bitcoin Standard: The Decentralized Alternative to Central Banking][bitcoin-standard]
% - [Abundance: The Future Is Better Than You Think][Abundance] by Peter Diamandis
% - [Presentation on The Bitcoin Standard][bitcoin-standard-presentation] by Saifedean Ammous
% - [Modeling Bitcoin's Value with Scarcity][planb-scarcity] by PlanB
% - 🎧 [Misir Mahmudov on the Scarcity of Time & Bitcoin][tftc60] TFTC #60 hosted by Marty Bent
% - 🎧 [PlanB – Modelling Bitcoin's digital scarcity through stock-to-flow techniques][slp67] SLP #67 hosted by Stephan Livera
%
% <!-- Through the Looking-Glass -->
% [lesson14]: {{ 'bitcoin/lessons/ch2-14-sound-money' | absolute_url }}
%
% <!-- Down the Rabbit Hole -->
% [Abundance]: https://www.diamandis.com/abundance
% [bitcoin-standard]: http://amzn.to/2L95bJW
% [bitcoin-standard-presentation]: https://www.bayernlb.de/internet/media/de/ir/downloads_1/bayernlb_research/sonderpublikationen_1/bitcoin_munich_may_28.pdf
% [planb-scarcity]: https://medium.com/@100trillionUSD/modeling-bitcoins-value-with-scarcity-91fa0fc03e25
% [tftc60]: https://anchor.fm/tales-from-the-crypt/episodes/Tales-from-the-Crypt-60-Misir-Mahmudov-e3aibh
% [slp67]: https://stephanlivera.com/episode/67
%
% <!-- Wikipedia -->
% [alice]: https://en.wikipedia.org/wiki/Alice%27s_Adventures_in_Wonderland
% [carroll]: https://en.wikipedia.org/wiki/Lewis_Carroll

\chapter{Réplication et localité}
\label{les:3}

\begin{chapquote}{Lewis Carroll, \textit{Alice au pays des merveilles}}
Ensuite résonna une voix furieuse, celle du Lapin, en train de crier :
\enquote{Pat ! Pat ! Où es-tu ?}
\end{chapquote}

Si l'on met de côté la mécanique quantique, la localité au sein du monde
physique n'est pas un problème. La question \textit{\enquote{Où se trouve X ?}}
trouve une réponse sensée, peu importe si X est une personne ou un objet. Dans
le monde numérique, la question du \textit{où} est déjà délicate en soi mais on
peut potentiellement y répondre. Sans rire, où sont situés vos e-mails ?
\enquote{Le cloud} serait une mauvaise réponse, c'est juste l'ordinateur de
quelqu'un d'autre. Pourtant, si vous vouliez situer chaque stockage contenant
une copie de vos e-mails, en théorie, vous pourriez.

Avec Bitcoin, la question du \enquote{où} est \textit{vraiment} délicate. Où
sont situés vos bitcoins, exactement ?

\begin{quotation}\begin{samepage}
\enquote{J'ai ouvert les yeux, regardé autour de moi et j'ai posé l'inévitable,
la sempiternelle, la tristement banale question postopératoire : `Où suis-je ?'}
\begin{flushright} -- Daniel Dennett\footnote{Daniel Dennett, \textit{Where Am
I?}~\cite{where-am-i}}
\end{flushright}\end{samepage}\end{quotation}

C'est un double problème : d'abord, le registre distribué l'est par réplication
totale, ce qui signifie que le registre est partout. Deuxièmement, les bitcoins
n'existent pas. Pas seulement physiquement, non, \textit{techniquement} aussi.

Bitcoin gère un ensemble de transactions sortantes non-dépensées, sans jamais
devoir mentionner une quelconque entité représentant un bitcoin.
L'existence d'un bitcoin se déduit en observant cet ensemble de transactions,
tout en désignant comme bitcoin chaque entrée totalisant 100 millions d'unités
de base.

\begin{quotation}\begin{samepage}
\enquote{Où est-il pendant le transfert, à ce moment ? [...] Premièrement, il
n'y a pas de bitcoins. Il n'y en a simplement pas. Ils n'existent pas. Il y a
des entrées dans un registre qui est partagé [...] Il n'existent dans aucun
lieu physique. Le registre lui existe partout, en gros. La géographie n'a pas de
sens ici ; ça ne vous aidera pas à définir votre politique.}
\begin{flushright} -- Peter Van Valkenburgh\footnote{Peter Van Valkenburgh dans
le podcast \textit{What Bitcoin Did}, épisode 49 \cite{wbd049}}
\end{flushright}\end{samepage}\end{quotation}

Par conséquent, que détenez-vous vraiment lorsque vous dites
\textit{\enquote{j'ai un bitcoin}}, s'ils n'existent pas ? Eh bien, vous vous
souvenez de tous ces mots étranges que le portefeuille que vous utilisez vous a
forcé à écrire ? Ce que vous détenez ce sont justement ces mots sorciers : une
formule magique\footnote{The Magic Dust of Cryptography: Comment l'information
numérique change notre société \cite{gigi:magic-spell}} qui sert à ajouter des
entrées dans le registre public ; les clés qui permettent de \enquote{déplacer}
des bitcoins. À toutes fins utiles, c'est pour ça que vos clés privées
\textit{sont} vos bitcoins. Si vous vous dites que j'invente tout ça, n'hésitez
pas à m'envoyer vos clés privées.

\paragraph{Bitcoin m'a appris que la localité était une histoire complexe.}

% ---
%
% #### Through the Looking-Glass
%
% - [The Magic Dust of Cryptography: How digital information is changing our society][a magic spell]
%
% #### Down the Rabbit Hole
%
% - [Where Am I?][Daniel Dennett] by Daniel Dennett
% - 🎧 [Peter Van Valkenburg on Preserving the Freedom to Innovate with Public Blockchains][wbd049] WBD #49 hosted by Peter McCormack
%
% <!-- Through the Looking-Glass -->
% [a magic spell]: 
%
% <!-- Down the Rabbit Hole -->
% [Daniel Dennett]: https://www.lehigh.edu/~mhb0/Dennett-WhereAmI.pdf
% [1st Amendment]: https://en.wikipedia.org/wiki/First_Amendment_to_the_United_States_Constitution
% [wbd049]: https://www.whatbitcoindid.com/podcast/coin-centers-peter-van-valkenburg-on-preserving-the-freedom-to-innovate-with-public-blockchains
%
% <!-- Wikipedia -->
% [alice]: https://en.wikipedia.org/wiki/Alice%27s_Adventures_in_Wonderland
% [carroll]: https://en.wikipedia.org/wiki/Lewis_Carroll

\chapter{Le problème de l'identité}
\label{les:4}

\begin{chapquote}{Lewis Carroll, \textit{Alice au pays des merveilles}}
  \enquote{Qui es-tu ?} lui demanda-t-elle.
\end{chapquote}

Nic Carter, en hommage à Thomas Nagel qui posait cette même question à propos
des chauve-souris, a écrit un excellent article sur la question : quel effet
cela fait-il d'être un bitcoin ? Il y montre remarquablement qu'en général, les
blockchains publiques et ouvertes, et Bitcoin plus particulièrement, souffrent
du même dilemme que le bateau de Thésée\footnote{Dans la métaphysique de
l'identité, le bateau de Thésée est une expérience de pensée qui soulève la
question de savoir si un objet dont toutes les parties ont été remplacées reste
fondamentalement le même objet.~\cite{wiki:theseus}} : quel Bitcoin est le vrai
Bitcoin ?

\begin{quotation}\begin{samepage}
\enquote{Observez par exemple comment les éléments de Bitcoin font preuve de peu
de persistance. L'ensemble du code source a déjà été retravaillé, modifié et
étendu tant et si bien qu'il ne ressemble que difficilement à sa version
d'origine. [...] Les archives de qui possède quoi, le registre lui-même, est
virtuellement la seule caractéristique persistante du réseau [...]
Afin d'être vraiment perçu sans responsable, il faut tourner le dos à cette
solution simple qui consiste à ce qu'une entité puisse désigner une chaîne comme
étant la chaîne légitime.}
\begin{flushright} -- Nic Carter\footnote{Nic Carter, \textit{Quel effet cela
fait-il d'être un bitcoin ?} \cite{bitcoin-identity}}
\end{flushright}\end{samepage}\end{quotation}

Il semble que le progrès technologique nous force sans arrêt à prendre au
sérieux ces considérations philosophiques. Un jour ou l'autre, les voitures
autonomes feront face au dilemme du tramway de façon bien réelle, ce qui les
obligera à prendre des décisions d'ordre éthique sur quelles vies comptent et
quelles vies ne comptent pas.

Les cryptomonnaies, particulièrement depuis le premier hard-fork litigieux, nous
forcent à réfléchir et à se mettre d'accord sur la métaphysique de l'identité.
Il est intéressant de constater que les deux meilleurs exemples que nous avons
connu jusqu'à présent ont mené à deux réponses différentes. Le premier août
2017, Bitcoin se sépara en deux camps. Le marché décida que la chaîne inchangée
était le Bitcoin originel. Un an plus tôt, le 25 octobre 2016, Ethereum se
séparait en deux camps. Le marché décidait que la chaîne \textit{modifiée} était
l'Ethereum originel.

Aussi longtemps que ces réseaux de transfert de valeur existeront, pour peu
qu'ils soient correctement décentralisés, les questions posées par le
\textit{bateau de Thésée} devront sans cesse trouver des réponses.

\paragraph{Bitcoin m'a appris que la décentralisation entrait en contradiction
avec l'identité.}

% ---
%
% #### Down the Rabbit Hole
%
% - [What Is It Like to be a Bat?][in regards to a bat] by Thomas Nagel
% - [What is it like to be a bitcoin?] by Nic Carter
% - [Ship of Theseus], [trolley problem] on Wikipedia
%
% [in regards to a bat]: https://en.wikipedia.org/wiki/What_Is_it_Like_to_Be_a_Bat%3F
% [What is it like to be a bitcoin?]: https://medium.com/s/story/what-is-it-like-to-be-a-bitcoin-56109f3e6753
% [Ship of Theseus]: https://en.wikipedia.org/wiki/Ship_of_Theseus
% [trolley problem]: https://en.wikipedia.org/wiki/Trolley_problem
%
% <!-- Wikipedia -->
% [alice]: https://en.wikipedia.org/wiki/Alice%27s_Adventures_in_Wonderland
% [carroll]: https://en.wikipedia.org/wiki/Lewis_Carroll

\chapter{L'Immaculée Conception}
\label{les:5}

\begin{chapquote}{Lewis Carroll, \textit{Alice au pays des merveilles}}
\enquote{Leur tête a disparu, [...]} répondirent les soldats.
\end{chapquote}

Ça parle à tout le monde lorsqu'une belle histoire donne lieu à une naissance.
Celle de Bitcoin est fascinante et ses détails sont plus importants qu'on
pourrait le croire à première vue. Qui est Satoshi Nakamoto ? Était-ce une seule
personne ou un groupe ? Était-ce un homme ou une femme ? Un extraterrestre qui
aurait voyagé dans le temps, ou une intelligence artificielle avancée ? Sans
tenir compte des théories absurdes, nous ne le saurons sans doute jamais. Et ça
a toute son importance.

Satoshi a choisi de rester anonyme. Il a fait germer la graine de Bitcoin. Il
est resté dans le coin suffisamment longtemps pour s'assurer que le réseau ne
connaîtrait pas une mort prématurée. Et il s'est évaporé.

Ce qui peut sembler une étrange pirouette à propos de l'anonymat est en réalité
une chose cruciale pour qu'un système soit vraiment décentralisé. Pas de
contrôle centralisé. Pas d'autorité centrale. Pas d'inventeur identifié.
Personne à poursuivre, à torturer, à faire chanter ou à extorquer. L'Immaculée
Conception d'une technologie.

\begin{quotation}\begin{samepage}
\enquote{L'une des meilleures choses que Satoshi ait faites a été de
disparaître.}
\begin{flushright} -- Jimmy Song\footnote{Jimmy Song, \textit{Pourquoi Bitcoin
est différent} \cite{bitcoin-different}}
\end{flushright}\end{samepage}\end{quotation}

\newpage

Depuis la naissance de Bitcoin, des milliers d'autres cryptomonnaies ont été
créées. Aucun de ces clones ne partage l'histoire de sa naissance. Si vous
voulez remplacer Bitcoin, vous allez devoir transcender cette histoire. Dans une
guerre d'idées, le récit impose la survie.

\begin{quotation}\begin{samepage}
\enquote{L'or a d'abord été travaillé sous forme de bijoux et utilisé pour le
troc il y a plus de 7000 ans. L'éclat captivant de l'or l'a mené à être
considéré comme un cadeau des dieux.}
\begin{flushright} Austrian Mint\footnote{The Austrian Mint, \textit{Gold: The
Extraordinary Metal} \cite{gold-gift-gods}}
\end{flushright}\end{samepage}\end{quotation}

Comme l'or il y a bien longtemps, Bitcoin pourrait être perçu comme un cadeau
des dieux. Mais contrairement à l'or, les origines de Bitcoin sont très
humaines. Et cette fois, nous connaissons les dieux du développement et de la
maintenance : des gens du monde entier, anonymes ou pas.

\paragraph{Bitcoin m'a appris que le narratif était important.}

% ---
%
% #### Down the Rabbit Hole
%
% - [Why Bitcoin is different][Jimmy Song] by Jimmy Song
% - [Gold: The Extraordinary Metal] by the Austrian Mint
%
% <!-- Down the Rabbit Hole -->
% [Jimmy Song]: https://medium.com/@jimmysong/why-bitcoin-is-different-e17b813fd947
% [Gold: The Extraordinary Metal]: https://www.muenzeoesterreich.at/eng/discover/for-investors/gold-the-extraordinary-metal
%
% <!-- Wikipedia -->
% [alice]: https://en.wikipedia.org/wiki/Alice%27s_Adventures_in_Wonderland
% [carroll]: https://en.wikipedia.org/wiki/Lewis_Carroll

\chapter{La force de la liberté d'expression}
\label{les:6}

\begin{chapquote}{Lewis Carroll, \textit{Alice au pays des merveilles}}
\enquote{Je te demande pardon !} dit la Souris très poliment, mais en fronçant
le sourcil. \enquote{Tu as dit quelque chose ? }
\end{chapquote}

Bitcoin est une idée. Une idée qui, dans sa forme actuelle, est la manifestation
de rouages purement alimentés par du texte. Tous les aspects de Bitcoin sont du
texte : le livre blanc, c'est du texte. Le logiciel qui s'exécute sur les nœuds,
c'est du texte. Le registre, c'est du texte. Les transactions, ce sont du texte.
Les clés publiques et privées, ce sont du texte. Tous les aspects de Bitcoin
sont du texte, en conséquence ils sont équivalents à de la parole.

\begin{quotation}\begin{samepage}
\enquote{Le Congrès n'adoptera aucune loi relative à l'établissement d'une
religion, ou à l'interdiction de son libre exercice ; ou pour limiter la liberté
d'expression, de la presse ou le droit des citoyens de se réunir pacifiquement
ou d'adresser au Gouvernement des pétitions pour obtenir réparations des torts
subis.}
\begin{flushright} -- Premier amendement de la Constitution des États-Unis
\end{flushright}\end{samepage}\end{quotation}

Bien que la bataille finale des Crypto Wars\footnote{Les \textit{Crypto Wars}
est le nom officieux des tentatives par les États-Unis et les gouvernements
alliés de saboter le chiffrement des
données.~\cite{eff-cryptowars}~\cite{wiki:cryptowars}} n'ait pas encore été
livrée, il sera extrêmement difficile de criminaliser une idée, qui plus est une
idée basée sur l'échange de messages écrits. Chaque fois qu'un gouvernement
tente d'interdire du texte ou de la parole, nous glissons sur le chemin de
l'absurdité qui mène inévitablement aux abominations telles que les nombres
illégaux\footnote{Un nombre illégal est un nombre qui représente une information
qu'il est interdit de posséder, prononcer, diffuser ou transmettre dans une
juridiction légale donnée.\cite{wiki:illegal-number}} et les nombres premiers
illégaux\footnote{Un nombre premier illégal est un nombre premier qui représente
une information dont la possession ou la distribution est interdite dans une
quelconque juridiction légale. L'un des premiers nombres premiers illégaux fut
découvert en 2001. Lorsqu'interprété d'une façon particulière, il décrit un
programme informatique qui permet de contourner le système de gestion des
droits numériques sur les DVD. La distribution d'un tel programme aux États-Unis
est illégale selon le Digital Millennium Copyright Act. Un nombre premier
illégal est une sorte de nombre illégal.\cite{wiki:illegal-prime}}.

Tant que quelque part dans le monde, l'expression restera libre comme dans
\textit{liberté}, Bitcoin sera inarrêtable.

\begin{quotation}\begin{samepage}
\enquote{Il n'y a aucun moment lors d'une transaction où Bitcoin ne cesse d'être
du \textit{texte}. Ce n'est \textit{que du texte}, tout le temps. [...] Bitcoin,
c'est du \textit{texte}. Bitcoin, c'est de la \textit{parole}. Il ne peut pas
être réglementé dans un pays libre tel que les États-Unis qui possède des droits
inaliénables garantis et un premier amendement qui retire explicitement le droit
de publier du contrôle du gouvernement.}
\begin{flushright} -- Beautyon\footnote{Beautyon, \textit{Pourquoi l'Amérique ne
peut réglementer Bitcoin} \cite{america-regulate-bitcoin}}
\end{flushright}\end{samepage}\end{quotation}

\paragraph{Bitcoin m'a appris que dans une société libre, la liberté
d'expression et le logiciel libre étaient inarrêtables.}

% ---
%
% #### Through the Looking-Glass
%
% - [The Magic Dust of Cryptography: How digital information is changing our society][a magic spell]
%
% #### Down the Rabbit Hole
%
% - [Why America can't regulate Bitcoin][Beautyon] by Beautyon
% - [First Amendment to the United States Constitution][1st Amendment], [Crypto Wars], [illegal numbers], [illegal primes] on Wikipedia
%
% <!-- Through the Looking-Glass -->
% [a magic spell]: 
%
% <!-- Down the Rabbit Hole -->
% [1st Amendment]: https://en.wikipedia.org/wiki/First_Amendment_to_the_United_States_Constitution
% [Crypto Wars]: https://en.wikipedia.org/wiki/Crypto_Wars
% [illegal numbers]: https://en.wikipedia.org/wiki/Illegal_number
% [illegal primes]: https://en.wikipedia.org/wiki/Illegal_prime
% [Beautyon]: https://hackernoon.com/why-america-cant-regulate-bitcoin-8c77cee8d794
%
% <!-- Wikipedia -->
% [alice]: https://en.wikipedia.org/wiki/Alice%27s_Adventures_in_Wonderland
% [carroll]: https://en.wikipedia.org/wiki/Lewis_Carroll

\chapter{Les limites du savoir}
\label{les:7}

\begin{chapquote}{Lewis Carroll, \textit{Alice au pays des merveilles}}
\enquote{Plus bas, encore plus bas, toujours plus bas. Est-ce que cette chute ne
finirait jamais ?}
\end{chapquote}

S'embarquer dans Bitcoin est une expérience humiliante. Je croyais que je savais
des trucs. Je pensais que j'étais instruit. Au minimum, je tenais ce que j'avais
appris en informatique pour acquis. J'ai étudié pendant des années, alors je
dois à peu près tout savoir sur les signatures numériques, le hachage, le
chiffrement, la sécurité opérationnelle et les réseaux, non ?

\paragraph{}
Faux.

\paragraph{}
C'est difficile d'étudier tous les fondamentaux qui rendent le fonctionnement de
Bitcoin possible. C'est limite impossible de tous les comprendre en profondeur.

\begin{quotation}\begin{samepage}
\enquote{Personne n'a atteint le fond du terrier du lapin Bitcoin.}
\begin{flushright} -- Jameson Lopp\footnote{Jameson Lopp, tweet du 11 novembre
2018 \cite{lopp-tweet}}
\end{flushright}\end{samepage}\end{quotation}

\begin{figure}
  \centering
  \includegraphics[width=7cm]{assets/images/rabbit-hole-bottomless.png}
  \caption{Le terrier du lapin Bitcoin n'a pas de fond.}
  \label{fig:rabbit-hole-bottomless}
\end{figure}

Ma liste de livres à lire ne cesse de s'agrandir bien plus vite que je ne suis
capable de la faire fondre. La liste de documents et d'articles à lire est
virtuellement infinie. Il y a bien plus de podcasts sur tous ces sujets que je
ne pourrais jamais en écouter. C'est réellement humiliant. En plus, Bitcoin
continue d'évoluer et il est presque impossible de suivre le rythme de
l'innovation qui s'accélère. La poussière soulevée par la première couche du
réseau n'a même pas encore fini de retomber, que des gens ont déjà construit la
seconde et travaillent sur la troisième.

\paragraph{Bitcoin m'a appris que je savais très peu sur presque tout. Il m'a
appris que ce terrier de lapin n'avait pas de fond.}

% ---
%
% #### Down the Rabbit Hole
%
% - [Bitcoin Literature] by the Satoshi Nakamoto Institute
% - [Bitcoin Information & Resources][lopp-resources] by Jameson Lopp
% - [Educational Resources][bitcoin-only] by Bitcoin Only
%
% <!-- Twitter -->
% [Jameson Lopp]: https://twitter.com/lopp/status/1061415918616698881
%
% <!-- Down the Rabbit Hole -->
% [lopp-resources]: https://www.lopp.net/bitcoin-information.html
% [bitcoin-only]: https://bitcoin-only.com/#learning
% [Bitcoin Literature]: https://nakamotoinstitute.org/literature/
%
% <!-- Wikipedia -->
% [alice]: https://en.wikipedia.org/wiki/Alice%27s_Adventures_in_Wonderland
% [carroll]: https://en.wikipedia.org/wiki/Lewis_Carroll

\part{Économie}
\label{ch:economics}
\chapter*{Économie}

\begin{chapquote}{Lewis Carroll, \textit{Alice au pays des merveilles}}
\enquote{Un grand rosier se dressait près de l’entrée du jardin ; il était tout
couvert de roses blanches, mais trois jardiniers s’affairaient à les peindre en
rouge. Ceci sembla très curieux à Alice\ldots}
\end{chapquote}

L'argent ne pousse pas sur les arbres. C'est idiot de croire ça et nos parents
ont fait en sorte de nous l'inculquer en le répétant comme un mantra. Nous
sommes encouragés à utiliser judicieusement l'argent, à ne pas le dépenser
inconsidérément et à l'épargner quand tout va bien pour les mauvais moments.
Après tout, l'argent ne pousse pas sur les arbres.

Bitcoin m'a plus appris sur l'argent que ce que j'aurais jamais cru devoir
savoir. Grâce à lui, j'ai été forcé d'explorer l'histoire de l'argent, du
secteur bancaire, diverses écoles de pensée économique et bien d'autres choses.
La quête pour la compréhension de Bitcoin m'a mené sur une multitude de chemins
et je tente d'en explorer certains au long de ce chapitre.

Dans les sept premières leçons j'ai abordé certaines questions philosophiques
qui entourent Bitcoin. Les sept suivantes se pencheront plutôt sur l'argent et
l'économie.

~

\begin{samepage}
Partie~\ref{ch:economics} -- Économie:

\begin{enumerate}
  \setcounter{enumi}{7}
  \item La méconnaissance financière
  \item L'inflation
  \item La valeur
  \item L'argent
  \item L'histoire et le déclin de la monnaie
  \item La folie des réserves fractionnaires
  \item Une monnaie saine
\end{enumerate}
\end{samepage}

À nouveau, je ne pourrai qu'effleurer la surface. Bitcoin est non seulement
ambitieux, mais il couvre aussi profondément un large spectre de domaines,
rendant impossibles à balayer tous les sujets pertinents en une seule leçon, un
seul essai, article ou livre. Je doute même que ce soit tout simplement
possible.

Bitcoin est une nouvelle forme de monnaie, qui rend l'étude de l'économie
primordiale à sa compréhension. S'agissant de la nature humaine et des
interactions entre agents économiques, l'économie est sans doute l'une des
pièces les plus grandes et les plus floues du puzzle Bitcoin.

À nouveau, ces leçons explorent diverses choses que Bitcoin m'a apprises. Elles
sont un reflet de ma chute dans le terrier du lapin. N'ayant pas de formation en
économie, je suis clairement en-dehors de ma zone de confort et je suis tout à
fait conscient que ma compréhension est potentiellement incomplète. Je ferai de
mon mieux pour présenter ce que j'ai retenu, au risque même de passer pour un
idiot. Après tout, je cherche toujours à répondre à la question :
\textit{\enquote{Qu'avez-vous appris de Bitcoin ?}}

Après sept leçons observées sous l'angle de la philosophie, passons à l'angle de
l'économie pour en examiner sept de plus. Tout ce que j'ai à vous offrir cette
fois, c'est un cours d'économie. Terminus : \textit{une monnaie saine}.

% [the question]: https://twitter.com/arjunblj/status/1050073234719293440

\chapter{La méconnaissance financière}
\label{les:8}

\begin{chapquote}{Lewis Carroll, \textit{Alice au pays des merveilles}}
\enquote{Et la dame pensera que je suis une petite fille ignorante ! Non, il
vaudra mieux ne rien demander ; peut-être que je verrai le nom écrit quelque
part.}
\end{chapquote}

L'une des choses qui m'a le plus frappé, c'est la quantité de finance,
d'économie et de psychologie nécessaire à la compréhension de ce qui semble, à
première vue, un système purement \textit{technique} -- un réseau informatique.
Pour paraphraser un petit gars aux pieds poilus : \enquote{Il est fort
dangereux, Frodon, d'étudier Bitcoin. On lit le livre blanc et si l'on ne
regarde pas où l'on met les pieds, on ne sait pas jusqu'où cela peut nous
mener.}

Pour comprendre un nouveau système monétaire, il faut apprendre à connaître
l'ancien. J'ai très vite réalisé que la quantité d'éducation financière
reçue au sein du système scolaire était à peu près égale à \textit{zéro}.

\paragraph{}
Comme un enfant, j'ai commencé à me poser bon nombre de questions : comment
fonctionne le système bancaire ? Comment fonctionnent les marchés financiers ?
Qu'est-ce que la monnaie fiduciaire ? Qu'est-ce que la monnaie
\textit{normale} ? Pourquoi existe-t-il autant de
dette ?\footnote{\url{https://www.usdebtclock.org/}} Combien de monnaie est
véritablement émise, et qui décide de ça ?

\newpage

Après une légère panique face à l'ampleur de mon ignorance, je me suis senti
rassuré lorsque j'ai constaté que j'étais en bonne compagnie.

\begin{quotation}\begin{samepage}
\enquote{C'est pas ironique que Bitcoin m'en ait appris plus sur la
monnaie que toutes ces années à travailler pour des institutions financières ?
\ldots y compris en ayant commencé dans une banque centrale}
\begin{flushright} -- Aaron\footnote{Aaron (\texttt{@aarontaycc},
\texttt{@fiatminimalist}), tweet du 12 déc. 2018~\cite{aarontaycc-tweet}}
\end{flushright}\end{samepage}\end{quotation}

\begin{quotation}\begin{samepage}
\enquote{J'ai plus appris sur la finance, l'économie, la technologie, la
cryptographie, la psychologie, la politique, la théorie des jeux, la législation
et moi-même pendant les trois derniers mois dans la crypto que mes trois années
et demie d'université}
\begin{flushright} -- Dunny\footnote{Dunny (\texttt{@BitcoinDunny}), tweet du 28
nov. 2017~\cite{bitcoindunny-tweet}}
\end{flushright}\end{samepage}\end{quotation}

Deux exemples des nombreuses confessions que l'on trouve un peu partout sur
Twitter\footnote{Voir \url{http://bit.ly/btc-learned} pour plus de confessions
Twitter.}. Bitcoin, comme nous l'avons vu dans la Leçon \ref{les:1}, est un
organisme vivant. Mises prétendait que l'économie aussi est vivante. Et nous le
savons tous par expérience, le vivant est difficile à appréhender par nature.

\begin{quotation}\begin{samepage}
\enquote{Un système scientifique est simplement une étape atteinte dans la
recherche indéfiniment continuée de la connaissance. Il est forcément affecté
par l'imperfection inhérente à tout effort humain. Mais reconnaître ces faits ne
signifie pas que la science économique de notre temps soit arriérée. Cela veut
dire seulement qu'elle est chose vivante, et vivre implique à la fois
imperfection et changement.}
\begin{flushright} -- Ludwig von Mises\footnote{Ludwig von Mises, \textit{L'Action Humaine}
\cite{human-action}}
\end{flushright}\end{samepage}\end{quotation}

\newpage

On voit tous passer des articles sur diverses crises financières, en se
demandant comment ces grands sauvetages fonctionnent et nous restons perplexes
face à l'absence de responsable de ces milliers de milliards de dégâts. Je reste
perplexe, mais au moins je commence à entrevoir ce qui se passe dans le monde de
la finance.

Certaines personnes vont jusqu'à attribuer la méconnaissance générale de ces
sujets à une méconnaissance plus systémique et délibérée. Tandis que l'histoire,
la physique, la biologie, les maths et les langues font toutes partie de notre
cursus, l'univers monétaire et financier n'est étonnamment abordé qu'en surface,
voire pas du tout. Je me demande si les gens continueraient d'accroître la dette
autant qu'ils le font si nous étions tous éduqués sur la gestion personnelle et
les rouages de la monnaie et du crédit. Puis je réfléchis à combien de couches
d'aluminium feraient un bon chapeau. Trois, probablement.

\begin{quotation}\begin{samepage}
\enquote{Ces crashs, ces sauvetages, ce ne sont pas des accidents. Et ce n'est 
pas non plus par accident qu'il n'y a pas d'éducation financière à l'école.
[...] C'est prémédité. Comme avant la Guerre Civile où il était illégal
d'instruire un esclave, nous n'avons pas le droit d'étudier la monnaie à
l'école.}
\begin{flushright} -- Robert Kiyosaki\footnote{Robert Kiyosaki, \textit{Pourquoi
les riches deviennent encore plus riches}\cite{robert-kiyosaki}}
\end{flushright}\end{samepage}\end{quotation}

Comme dans Le Magicien d'Oz, on nous demande de ne pas prêter attention à
l'homme derrière le rideau. Contrairement au Magicien d'Oz, nous possédons
maintenant une véritable
sorcellerie\footnote{\url{http://bit.ly/btc-wizardry}} : un réseau de transfert
de valeur, résistant à la censure, ouvert et sans frontières. Il n'y a pas de
rideau et chacun peut en apprécier la
magie\footnote{\url{https://github.com/bitcoin/bitcoin}}.

\paragraph{Bitcoin m'a appris à regarder derrière le rideau et à surmonter ma
méconnaissance financière.}

% ---
%
% #### Down the Rabbit Hole
%
% - [Human Action][Ludwig von Mises] by Ludwig von Mises
% - [Why the Rich are Getting Richer][Robert Kiyosaki] by Robert Kiyosaki
%
% [real wizardry]: https://external-preview.redd.it/8d03MWWOf2HIyKrT8ThBGO4WFv-u25JaYqhbEO9b1Sk.jpg?width=683&auto=webp&s=dc5922d84717c6a94527bafc0189fd4ca02a24bb
% [visible to anyone]: https://github.com/bitcoin/bitcoin
%
% <!-- Wikipedia -->
% [alice]: https://en.wikipedia.org/wiki/Alice%27s_Adventures_in_Wonderland
% [carroll]: https://en.wikipedia.org/wiki/Lewis_Carroll

\chapter{L'inflation}
\label{les:9}

\begin{chapquote}{La Reine Rouge} 
\enquote{Ici, vois-tu, on est obligé de courir tant qu'on peut pour rester au
même endroit. Si on veut aller ailleurs, il faut courir au moins deux fois plus
vite que ça !}
\end{chapquote}

Tenter de comprendre l'inflation monétaire et comment un système
désinflationniste tel que Bitcoin pourrait changer nos comportements a constitué
le début de ma plongée dans l'économie. Je savais que l'inflation était le taux
auquel la monnaie était nouvellement émise, mais je n'en savais pas beaucoup
plus que ça.

Tandis que certains économistes prétendent que l'inflation est bonne, d'autres
prétendent qu'une monnaie \enquote{dure} qui ne peut être facilement produite
--- comme nous avions durant la période de l'étalon-or --- est indispensable à
une économie saine. Bitcoin, avec son offre fixe de 21 millions, partage l'avis
du second camp.

Habituellement, les effets de l'inflation ne sautent pas aux yeux. Selon le taux
d'inflation (ainsi que d'autres facteurs), le délai entre la cause et la
conséquence peut s'étendre sur plusieurs années. De plus, l'inflation touche
certains catégories plus que d'autres. Comme Henry Hazlitt le fait remarquer
dans \textit{L'Économie Politique en Une Leçon} : \enquote{L'art de la politique
économique consiste à ne pas considérer uniquement l'aspect immédiat d'un
problème ou d'un acte, mais à envisager ses effets plus lointains ; il consiste
essentiellement à considérer les conséquences que cette politique peut avoir,
non seulement sur un groupe d'hommes ou d'intérêts donnés, mais sur tous les
groupes existants.}

L'une de mes révélations fut le moment où j'ai compris que l'émission monétaire
--- imprimer plus d'argent --- était une activité économique \textit{totalement}
différente de toutes les autres activités économiques. Pendant que les vrais
biens et services produisent de la vraie valeur pour les vrais gens, imprimer
plus d'argent a l'exact effet contraire : on retire de la valeur à tous ceux
qui détiennent cette monnaie dont la quantité augmente.

\begin{quotation}\begin{samepage}
\enquote{L'inflation en soi --- c'est-à-dire la simple émission de plus de
monnaie, avec pour conséquences la hausse des salaires et l'accroissement des
prix --- peut très bien avoir l'air de créer une demande supplémentaire. Mais si
on raisonne en termes de production et d'échange des biens réels, il n'en est
rien.}
\begin{flushright} -- Henry Hazlitt\footnote{Henry Hazlitt, \textit{L'Économie
Politique en Une Leçon} \cite{hazlitt}}
\end{flushright}\end{samepage}\end{quotation}

La force destructrice de l'inflation devient évidente dès qu'un peu d'inflation
se change en \textit{beaucoup}. Si la monnaie subit une hyperinflation, les
choses tournent au vinaigre très rapidement
\footnote{\url{https://en.wikipedia.org/wiki/Hyperinflation}\cite{wiki:hyperinflation}}.
Au fur et à mesure qu'une monnaie se désagrège, elle échouera à stocker la
valeur à travers le temps et les gens se précipiteront pour mettre la main sur
n'importe quel bien qui pourra y parvenir.

\paragraph{}
Une autre conséquence de l'hyperinflation, c'est que tout l'argent épargné par
les gens durant leur vie va littéralement s'évaporer. L'argent liquide qui se
trouve dans votre portefeuille sera toujours là, bien entendu. Mais ça ne sera
effectivement plus que ça : du papier sans valeur.

\begin{figure}
  \includegraphics{assets/images/children-playing-with-money.png}
  \caption{Hyperinflation pendant la République de Weimar (1921-1923)}
  \label{fig:children-playing-with-money}
\end{figure}

\paragraph{}
La monnaie perd également de la valeur avec une inflation soi-disant
\enquote{modérée}. C'est juste qu'elle arrive suffisamment lentement pour que la
plupart des gens ne s'aperçoivent pas de la baisse de leur pouvoir d'achat. Et
dès lors que la planche à billets tourne, la quantité de monnaie peut être
facilement accrue et ce qui était auparavant une inflation modérée peut se
transformer par l'appui d'une simple bouton en une bonne dose d'inflation bien
forte. Friedrich Hayek le faisait remarquer dans l'un de ses essais, l'inflation
modérée mène généralement à l'inflation pure et simple.

\begin{quotation}\begin{samepage}
\enquote{Une inflation `modérée' et stable ne peut pas nous aider --- cela peut
seulement mener à l'inflation totale.}
\begin{flushright} -- Friedrich Hayek\footnote{Friedrich Hayek, \textit{Le
chômage des années 80 et les syndicats} \cite{hayek-inflation}}
\end{flushright}\end{samepage}\end{quotation}

L'inflation est particulièrement sournoise, puisqu'elle favorise ceux qui sont
au plus près du procédé d'émission. Il faut du temps pour que la monnaie
nouvellement émise circule et que les prix s'ajustent. Donc si vous avez la
possibilité de mettre la main sur plus d'argent avant que celui des autres ne se
dévalue, vous avez de l'avance sur la courbe d'inflation. C'est aussi pour ça
que l'on peut voir l'inflation comme un impôt caché, car au final ce sont les
gouvernements qui en profitent tandis que tout le monde en paye le prix.

\begin{quotation}\begin{samepage}
\enquote{Je ne pense pas que cela soit une exagération de dire que l’Histoire
est largement une histoire d’inflation, une inflation habituellement fabriquée
par les gouvernements, pour le gain des gouvernements.}
\begin{flushright} -- Friedrich Hayek\footnote{Friedrich Hayek, \textit{La Bonne
Monnaie} \cite{hayek-good-money}}
\end{flushright}\end{samepage}\end{quotation}

\paragraph{}
Jusqu'à présent, les devises contrôlées par les gouvernements ont toutes fini
par être remplacées ou s'effondrer totalement. Peu importe la faiblesse du taux
d'inflation, parler de croissance \enquote{stable} revient à parler de
croissance exponentielle. Dans la nature comme en économie, tous les systèmes
qui se développent de façon exponentielle devront se stabiliser sous peine de
subir un effondrement catastrophique.

\paragraph{}
\enquote{Ça ne peut pas arriver dans mon pays}, c'est probablement ce que vous
vous dites. Ce n'est pas du tout ce que vous vous dites si vous vivez au
Vénézuela, qui est en ce moment touché par l'hyperinflation. Avec un taux
d'inflation supérieur à un million de pourcents, leur argent ne vaut plus rien.
\cite{wiki:venezuela}

\paragraph{}
Ça pourrait encore prendre plusieurs années, ou ne pas toucher votre devise.
Mais il suffit de jeter un coup d'œil à la liste des anciennes
monnaies\footnote{Voir \textit{Liste des anciennes monnaies} sur Wikipedia.
\cite{wiki:historical-currencies}} pour voir que cela se produit invariablement,
au cours d'un temps suffisamment long. Je me souviens avoir réellement utilisé
toutes celles-ci : le schilling autrichien, le Deutsche mark, la lire italienne,
le franc français, la livre irlandaise, le dinar croate, etc. Ma grand-mère a
même utilisé la couronne austro-hongroise. Au fil du temps, les monnaies en
circulation\footnote{Voir \textit{Liste des monnaies en circulation} sur
Wikipedia \cite{wiki:list-of-currencies}} vont lentement mais sûrement se
diriger vers leurs cimetières respectifs. Elles subiront une hyperinflation ou
seront remplacées. Elles seront bientôt des monnaies anciennes. Nous les
rendrons obsolètes.

\begin{quotation}\begin{samepage}
\enquote{L'Histoire a montré que les gouvernements cèdent immanquablement à la
tentation de gonfler l'offre de monnaie.}
\begin{flushright} -- Saifedean Ammous\footnote{Saifedean Ammous,
\textit{L'Étalon Bitcoin} \cite{bitcoin-standard}}
\end{flushright}\end{samepage}\end{quotation}

\paragraph{}
Pourquoi Bitcoin est-il différent ? Contrairement aux monnaies imposées par les
états, les biens monétaires qui ne sont pas réglementés par des gouvernements,
mais par les lois de la physique\footnote{Gigi, \textit{La consommation
énergétique de Bitcoin - Une nouvelle perspective} \cite{gigi:energy}}, ont une
tendance à la survie et même à maintenir leur valeur au fil du temps. Jusqu'à
présent, le meilleur exemple est l'or qui, comme l'atteste le bien nommé
\textit{rapport or sur costume correct}\footnote{L'Histoire montre que le prix
d'une once d'or est égal au prix d'un costume pour homme de bonne facture, selon
le cabinet de conseil en investissement
Sionna\cite{web:gold-to-decent-suite-ratio}}, conserve sa valeur sur des
centaines et même des milliers d'années. Il n'est peut-être pas parfaitement
\enquote{stable} --- un concept discutable dès le départ --- mais la valeur
qu'il renferme reste au moins dans les mêmes ordres de grandeur.

Lorsqu'un bien monétaire ou une devise conserve efficacement sa valeur à travers
le temps et l'espace, il est perçu comme \textit{fort}. Si à l'inverse il ne
peut la maintenir, parce qu'il s'abîme ou gonfle facilement son offre, il est
considéré comme \textit{faible}. Le concept de dureté est essentiel à la
compréhension de Bitcoin et mérite un examen détaillé. Nous y reviendrons dans
la dernière leçon sur l'économie : la monnaie saine.

\paragraph{}
Alors que de plus en plus de pays souffrent d'hyperinflation, de plus en plus de
gens devront affronter la réalité des monnaies fortes et faibles. Si nous avons
de la chance, il se peut même que certaines banques centrales se trouvent
forcées de réévaluer leurs politiques monétaires. Quoi qu'il arrive, la lucidité
que m'a procuré Bitcoin s'avérera sans doute inestimable, quelle que soit
l'issue.

\paragraph{Bitcoin m'a appris que l'inflation était un impôt caché et que
l'hyperinflation était une catastrophe.}

% ---
%
% #### Down the Rabbit Hole
%
% - [Economics in One Lesson][Henry Hazlitt] by Henry Hazlitt
% - [1980's Unemployment and the Unions][unions] by Friedrich Hayek
% - [Good Money, Part II][good-money]: Volume Six of the Collected Works of F.A. Hayek
% - [The Bitcoin Standard] by Saifedean Ammous
% - [Hyperinflation][hyperinflates], [economic crisis in Venezuela][wiki-venezuela], [list of historical currencies], [list of currencies][currently in use] on Wikipedia
%
% [unions]: https://books.google.com/books/about/1980s_unemployment_and_the_unions.html?id=xM9CAQAAIAAJ
% [good-money]: https://books.google.com/books?id=l_A1vVIaYBYC
%
% [Henry Hazlitt]: https://mises.org/library/economics-one-lesson
% [hyperinflates]: https://en.wikipedia.org/wiki/Hyperinflation
% [inflation cannot help]: https://books.google.com/books?id=zZu3AAAAIAAJ&dq=%22only+while+it+accelerates%22&focus=searchwithinvolume&q=%22steady+inflation+cannot+help%22
% [history of inflation]: https://books.google.com/books?id=l_A1vVIaYBYC&pg=PA142&dq=%22history+is+largely+a+history+of+inflation%22&hl=en&sa=X&ved=0ahUKEwi90NDLrdnfAhUprVkKHUx1CmIQ6AEIKjAA#v=onepage&q=%22history%20is%20largely%20a%20history%20of%20inflation%22&f=false
% [wiki-venezuela]: https://en.wikipedia.org/wiki/Crisis_in_Venezuela#Economic_crisis
% [by the laws of physics]: https://link.medium.com/9fzq2L0J3S
% [\textit{Gold-to-Decent-Suit Ratio}]: https://www.businesswire.com/news/home/20110819005774/en/History-Shows-Price-Ounce-Gold-Equals-Price
% [The Bitcoin Standard]: https://thesaifhouse.wordpress.com/book/
%
% <!-- Wikipedia -->
% [alice]: https://en.wikipedia.org/wiki/Alice%27s_Adventures_in_Wonderland
% [carroll]: https://en.wikipedia.org/wiki/Lewis_Carroll

\chapter{La valeur}
\label{les:10}

\begin{chapquote}{Lewis Carroll, \textit{Alice au pays des merveilles}}
\enquote{C’était le Lapin Blanc qui revenait en trottant lentement et en jetant
autour de lui des regards inquiets comme s’il avait perdu quelque chose\ldots}
\end{chapquote}

La valeur est en quelque sorte paradoxale et il existe de multiples
théories\footnote{Voir \textit{Théorie de la valeur (économie)} sur Wikipédia
\cite{wiki:theory-of-value}} qui tentent d'expliquer pourquoi nous donnons de la
valeur à certaines choses plutôt qu'à d'autres. Les gens sont conscients de ce
paradoxe depuis des millénaires. Comme Platon l'écrivait dans son dialogue avec
Euthydème, nous estimons certaines choses parce qu'elles sont rares, pas
seulement sur leur nécessité à notre survie.

\begin{quotation}\begin{samepage}
\enquote{Même, pour bien faire, vous avertiriez vos écoliers d'en user de la
sorte, et de n'en parler qu'entre eux ou avec vous ; car la rareté, Euthydème,
met le prix aux choses, et l'eau, comme dit Pindare, se vend à vil prix
quoiqu'elle soit ce qu'il y a de plus précieux.}
\begin{flushright} -- Platon\footnote{Platon, \textit{Euthydème}
\cite{euthydemus}}
\end{flushright}\end{samepage}\end{quotation}

Ce paradoxe de la valeur\footnote{Voir \textit{Paradoxe de l'eau et du diamant}
sur Wikipedia\cite{wiki:paradox-of-value}} montre une chose intéressante à
propos de nous autres, humains : il semblerait que nous estimions les choses sur
une base subjective\footnote{Voir \textit{Conception subjective de la valeur}
sur Wikipedia \cite{wiki:subjective-theory-of-value}}, tout en observant
certains critères raisonnés. Une chose peut nous être \textit{précieuse} pour de
nombreuses raisons, mais celles auxquelles nous accordons de la valeur partagent
certains traits. Si cette chose se copie facilement ou qu'elle est naturellement
abondante, nous ne l'estimons pas.

Apparemment, nous accordons de la valeur à quelque chose par sa rareté (l'or,
les diamants, le temps), sa complexité ou sa quantité de travail nécessaire, son
irremplaçabilité (une vieille photo d'un être cher), son utilité à permettre des
choses autrement impossibles ou encore une combinaison de tout ça, comme les
grandes œuvres d'art.

Bitcoin est tout ça à la fois : il est extrêmement rare (21 millions), de plus
en plus dur à produire (la réduction des récompenses), impossible à remplacer
(une clé privée perdue l'est à jamais) et nous permet de faire des choses plutôt
utiles. C'est vraisemblablement le meilleur outil pour transférer de la valeur
au-delà des frontières, il est virtuellement résistant à la censure et à la
saisie, ce qui permet à n'importe qui de stocker sa valeur sans l'aval des
banques et du gouvernement, pour ne citer qu'eux.

\paragraph{Bitcoin m'a appris que la valeur était subjective, mais pas
arbitraire.}

% ---
%
% #### Down the Rabbit Hole
%
% - [Euthydemus] by Plato
% - [Theory of Value][multiple theories], [Paradox of Value][paradox of value], [Subjective Theory of Value][subjective] on Wikipedia
%
% [Euthydemus]: http://www.perseus.tufts.edu/hopper/text?doc=Perseus:text:1999.01.0178:text=Euthyd.
% [Plato]: http://www.perseus.tufts.edu/hopper/text?doc=plat.+euthyd.+304b
%
% <!-- Wikipedia -->
% [multiple theories]: https://en.wikipedia.org/wiki/Theory_of_value_%28economics%29
% [paradox of value]: https://en.wikipedia.org/wiki/Paradox_of_value
% [subjective]: https://en.wikipedia.org/wiki/Subjective_theory_of_value
% [alice]: https://en.wikipedia.org/wiki/Alice%27s_Adventures_in_Wonderland
% [carroll]: https://en.wikipedia.org/wiki/Lewis_Carroll

\chapter{L'argent}
\label{les:11}

\begin{chapquote}{Le Sage}
\enquote{Quand j'étais jeune, \ldots \\
j'ai bien entretenu la souplesse de mes muscles \\
en me frottant avec cette crème \\
-- un shilling la boite ! – \\
Voulez-vous m'en acheter un lot ?}
\end{chapquote}

Qu'est-ce que l'argent ? On l'utilise tous les jours, pourtant il est
étonnamment complexe de répondre à cette question. Nous en dépendons de toutes
les manières possibles et imaginables et si nous en manquons nos vies deviennent
très difficiles. Toutefois, nous réfléchissons rarement à cette chose qui fait
prétendument tourner le monde. Bitcoin m'a contraint à répondre inlassablement à
cette question : mais enfin c'est quoi, l'argent ?

Dans notre monde \enquote{moderne}, la plupart des gens penseront probablement à
des bouts de papier lorsqu'ils parleront d'argent, alors même qu'il n'est en
majorité qu'un nombre sur un compte bancaire. Notre argent est déjà fait de un
et de zéros, alors en quoi Bitcoin est-il différent ? Il l'est car en son sein,
c'est un \textit{type} de monnaie très différent de celle que l'on utilise
habituellement. Pour le comprendre, nous allons devoir nous pencher sur ce
qu'est la monnaie, comment elle a vu le jour et pourquoi l'or et l'argent ont
été utilisés durant la majeure partie de l'histoire du commerce.

\paragraph{}
Les coquillages, l'or, l'argent, le papier, le bitcoin. En fin de compte,
\textbf{la monnaie c'est ce dont les gens se servent}, peu importe son aspect et
sa forme, ou l'absence de celles-ci.

L'argent est ingénieux, en tant qu'invention. Un monde sans argent s'en
retrouverait compliqué à l'absurde : combien de poissons pour ces nouvelles
chaussures ? Combien de vaches pour acheter une maison ? Qu'est-ce que je fais
si je n'ai aucun besoin immédiat mais que je dois me débarrasser de mes pommes
bien mûres ? Pas besoin d'être très imaginatif pour comprendre qu'une économie
basée sur le troc serait terriblement inefficace.

Le truc excellent avec l'argent c'est qu'on peut l'échanger contre
\textit{n'importe quoi d'autre} -- c'est une sacrée invention ! Tel que Nick
Szabo\footnote{\url{http://unenumerated.blogspot.com/}} le résume
brillamment dans \textit{Shelling Out: The Origins of Money}
\cite{shelling-out}, les êtres humains ont utilisé toutes sortes de choses en
tant que monnaie : des perles de matériaux rares comme l'ivoire, des
coquillages, des os spécifiques, divers types de bijoux, puis plus tard des
métaux rares comme l'argent ou l'or.

\begin{quotation}\begin{samepage}
\enquote{En ce sens, il est plus représentatif d'un métal précieux. Au lieu que
ce soit l'offre qui change afin de maintenir la valeur, celle-ci est
prédéterminée et c'est la valeur qui change.}
\begin{flushright} -- Satoshi Nakamoto\footnote{Satoshi Nakamoto, dans une
réponse à Sepp Hasslberger \cite{satoshi-precious-metal}}
\end{flushright}\end{samepage}\end{quotation}

Tels les bons paresseux que nous sommes, nous ne passons pas trop de temps à
réfléchir à ce qui marche. L'argent, pour la plupart d'entre nous, ça fonctionne
très bien. Comme avec nos voitures ou nos ordinateurs, nous ne sommes obligés
d'y penser que lorsqu'un de ces trucs tombe en panne. Les personnes qui ont vu
leurs économies d'une vie s'évaporer avec l'hyperinflation savent très bien la
valeur d'une monnaie forte, tout comme ceux qui ont vu leurs amis et famille
disparaître à cause des atrocités de l'Allemagne nazie ou de l'Union Soviétique
connaissent très bien la valeur de la confidentialité.

Le problème avec l'argent, c'est qu'il est partout. Il représente la moitié de
chaque transaction, ce qui confère un pouvoir considérable à ceux qui sont
responsables de l'émission monétaire.

\begin{quotation}\begin{samepage}
\enquote{Étant donné que l'argent représente la moitié de chaque transaction
commerciale et que des civilisations entières s'épanouissent et s'effondrent
selon la qualité de leur monnaie, on parle ici d'un pouvoir incommensurable, un
pouvoir qui est passé sous silence. C'est le pouvoir de tisser des mirages qui
semblent vrais aussi longtemps qu'ils durent. C'est là le cœur du pouvoir de la
Réserve Fédérale.}
\begin{flushright} -- Ron Paul\footnote{Ron Paul, \textit{End the Fed}
\cite{end-the-fed}}
\end{flushright}\end{samepage}\end{quotation}

Bitcoin retire ce pouvoir pacifiquement, puisqu'il met fin à l'émission
monétaire sans recourir à la force.

L'argent a connu de multiples itérations. La plupart d'entre elles étaient
bonnes. Elles amélioraient notre monnaie d'une façon ou d'une autre. En
revanche, très récemment, ses rouages ont été corrompus. Aujourd'hui, la
quasi-totalité de notre argent est fabriqué \textit{de toutes pièces} par les
pouvoirs en place. Pour comprendre comment nous en sommes arrivés là, j'ai dû
étudier l'histoire de la monnaie et de son déclin consécutif.

Il reste encore à voir s'il faudra une série de catastrophes ou simplement un
effort éducatif monumental pour réparer cette corruption. Je prie les dieux de
la monnaie saine afin que ce soit le second.

\paragraph{Bitcoin m'a appris ce qu'était l'argent.}

% ---
%
% #### Down the Rabbit Hole
%
% - [End the Fed][Ron Paul] by Ron Paul
% - [Money, blockchains, and social scalability][social-scalability] by Nick Szabo
%
% [social-scalability]: https://unenumerated.blogspot.co.at/2017/02/money-blockchains-and-social-scalability.html
%

\chapter{L'histoire et le déclin de la monnaie}
\label{les:12}

\begin{chapquote}{Lewis Carroll, \textit{Alice au pays des merveilles}}
\enquote{\ldots ils avaient refusé de se rappeler les simples règles de conduite
que leurs amis leur avaient enseignées : par exemple, qu’un tisonnier chauffé au
rouge vous brûle si vous le tenez trop longtemps, ou que, si vous vous faites au
doigt une coupure très profonde avec un couteau, votre doigt, d’ordinaire, se
met à saigner ; et Alice n’avait jamais oublié que si l’on boit une bonne partie
du contenu d’une bouteille portant l’étiquette : poison, cela ne manque presque
jamais, tôt ou tard, de vous causer des ennuis.}
\end{chapquote}

Beaucoup de personnes pensent que la monnaie est soutenue par de l'or, qui
serait enfermé dans de grandes chambres fortes, protégées par d'épais murs. Ce
n'est plus vrai depuis plusieurs décennies. Je ne suis pas certain de ce que
j'en ai pensé quand je l'ai appris, puisque j'étais encore plus embêté, n'ayant
potentiellement aucune compréhension de l'or, de l'argent papier ou même de
pourquoi il faudrait le soutenir avec quelque chose, pour commencer.

Un des aspects de l'étude de Bitcoin est l'étude de la monnaie fiduciaire : ce
que ça signifie, comment c'est apparu et pourquoi c'est peut-être pas la
meilleure idée que nous ayons eu. Alors, qu'est-ce que la monnaie fiduciaire
exactement ? Et comment en est-on arrivés à l'utiliser ?

Lorsque quelque chose est imposé par \textit{décret}, ça veut simplement dire
c'est imposé par une autorisation ou une proposition officielle. Par conséquent,
la monnaie fiduciaire est monnaie simplement parce que \textit{quelqu'un} le
décide. De nos jours, comme tous les gouvernements utilisent de la monnaie
fiduciaire, ce quelqu'un c'est \textit{votre} gouvernement. Malheureusement,
vous n'êtes pas \textit{libre} d'être en désaccord avec cette proposition de
valeur. Vous vous apercevrez rapidement qu'elle est tout sauf non-violente. Si
vous refusez d'utiliser cette monnaie papier pour mener vos affaires et payer
vos impôts, les seules personnes avec qui vous pourrez parler économie seront
vos compagnons de cellule.

La valeur de la monnaie fiduciaire ne découle pas de ses attributs intrinsèques.
La qualité d'une catégorie donnée de monnaie fiduciaire n'est corrélée qu'à 
l'(in)stabilité politique et fiscale de ceux qui la font passer du rêve à la
réalité. Sa valeur est décrétée, arbitrairement.

\begin{figure}
  \centering
  \includegraphics[width=8cm]{assets/images/fiat-definition.png}
  \caption{fiat --- `Qu'il en soit ainsi'}
  \label{fig:fiat-definition}
\end{figure}

\paragraph{}
Jusqu'à récemment, deux sortes de monnaie étaient d'usage : la \textbf{monnaie
de commodité}, faite de \textit{choses} précieuses ; et la \textbf{monnaie
représentative}, qui \textit{représente} uniquement la chose précieuse,
généralement par des jeux d'écriture.

\paragraph{}
Nous avons déjà abordé la monnaie de commodité plus haut. Les gens utilisaient
des os, des coquillages et des métaux précieux comme monnaie. Plus tard, ce sont
principalement des pièces faites de ces métaux comme l'or ou l'argent qui
furent utilisées. La plus vieille pièce qu'on ait trouvée jusqu'à aujourd'hui
est faite d'un alliage naturel d'or et d'argent et a été frappée il y a plus de
2700 ans\footnote{D'après l'historien grec Hérodote, contemporain du Ve siècle
avant J.-C., les Lydiens furent le premier peuple à utiliser des pièces d'or et
d'argent. \cite{coinage-origins}}. S'il y a de l'innovation dans Bitcoin, ce
n'est pas le concept de pièce.

\begin{figure}
  \centering
  \includegraphics[width=5cm]{assets/images/lydian-coin-stater.png}
  \caption{Pièce lydienne en électrum. Crédit photo CC-BY-SA Classical
  Numismatic Group, Inc.}
  \label{fig:lydian-coin-stater}
\end{figure}

\paragraph{}
Il s'avère que la thésaurisation, ou hodling, pour reprendre le jargon moderne,
est presque aussi vieille que les pièces elles-mêmes. Le plus ancien hodler de
pièces était une personne qui a mis une centaine de celles-ci dans une jarre et
l'a enterrée sous les fondations d'un temple, pour qu'on ne les retrouve qu'au
bout de 2500 ans. Plutôt un bon stockage à froid si vous voulez mon avis.

L'un des inconvénients d'utiliser des pièces faites de métaux précieux est
qu'elles peuvent être rognées, dépréciant de fait leur valeur. De nouvelles
pièces peuvent être frappées à partir des copeaux, faisant gonfler l'offre de
monnaie au fil du temps, dévaluant chaque autre pièce au passage. Les gens
rabotaient littéralement tout ce qu'ils pouvaient de leurs dollars d'argent. Je
me demande quel genre de pubs \textit{Dollar Shave Club} faisait à l'époque.

Puisque la seule inflation que les gouvernements tolèrent est celle dont ils
sont responsables, des efforts furent entrepris pour mettre un terme à cette
guérilla de la dépréciation. Au jeu traditionnel du gendarme et du voleur, les
rogneurs de pièces ont fait preuve d'imagination dans leurs procédés, forçant
les \enquote{Maîtres de la Monnaie} à faire preuve d'encore plus d'imagination
dans leurs contre-mesures. Isaac Newton, le physicien mondialement reconnu,
auteur de \textit{Principia Mathematica}, était l'un de ces maîtres. C'est à lui
qu'on attribue l'ajout des stries sur la tranche des pièces que l'on peut
toujours observer aujourd'hui. L'époque du rognage facile était révolue.

\begin{figure}
  \includegraphics{assets/images/clipped-coins.png}
  \caption{Pièces d'argent rognées à divers degrés.}
  \label{fig:clipped-coins}
\end{figure}

Même en gardant un œil sur les procédés de dépréciation des pièces\footnote{En
plus du rognage, le grippage (secouer les pièces dans un sac et récupérer la
poussière de métal qui s'est détachée) ainsi que le tamponnage (faire un trou au
centre de la pièce et l'aplatir au marteau jusqu'à combler le trou) étaient les
techniques de dépréciation les plus répandues. \cite{wiki:coin-debasement}},
celles-ci faisaient encore face à d'autres problèmes. Elles sont encombrantes et
pas très pratiques à transporter, surtout en cas de grands transferts de valeur.
C'est pas vraiment faisable d'arriver avec un gros sac de dollars en argent
chaque fois que vous voulez acheter une Mercedes.

En parlant de trucs allemands : l'origine du nom du \textit{dollar} américain
est une autre histoire intéressante. Le mot \enquote{dollar} est dérivé du mot
allemand \textit{Thaler}, l'abréviation de
\textit{Joachimsthaler}~\cite{wiki:thaler}. Un Joachimsthaler était une pièce
frappée dans la ville de \textit{Sankt Joachimsthal}. Thaler est simplement le
raccourci pour désigner quelqu'un (ou quelque chose) qui vient de la vallée. Et
puisque Joachimsthal était \textit{la} vallée ou l'on produisait les pièces
d'argent, les gens appelaient naturellement ces pièces des \textit{Thaler}.
Thaler (en allemand) a glissé vers daalders (en hollandais) puis finalement vers
dollars (en anglais).

\begin{figure}
  \centering
  \includegraphics[width=5cm]{assets/images/joachimsthaler.png}
  \caption{Le `dollar' originel. Saint Joachim est représenté avec sa robe et
  son chapeau de mage. Crédit photo CC-BY-SA Wikipedia utilisateur
  Berlin-George}
  \label{fig:joachimsthaler}
\end{figure}

L'introduction de la monnaie représentative sonna le glas de la monnaie dure.
Les certificats sur l'or furent introduits en 1863 et quinze ans après, le
dollar d'argent fut lui aussi lentement mais sûrement remplacé par un
intermédiaire de papier : le certificat sur l'argent.
\cite{wiki:silver-certificate}

Il aura ensuite fallu environ un demi-siècle après l'arrivée des certificats
pour que ces morceaux de papier se changent en ce que nous connaissons de nos
jours comme le dollar américain.

\begin{figure}
  \centering
  \includegraphics{assets/images/us-silver-dollar-note-smaller.png}
  \caption{Un dollar d'argent américain de 1928. `Payable au porteur sur
  demande.' Crédit photo CC-BY-SA Collection numismatique nationale de
  l'institut Smithsonian}
  \label{fig:us-silver-dollar-note-smaller}
\end{figure}

Notez que le dollar d'argent américain de 1928 dans la
figure~\ref{fig:us-silver-dollar-note-smaller} porte toujours le nom de
\textit{certificat sur l'argent}, ce qui indique qu'il s'agit bien d'un document
stipulant que l'on doit au porteur de ce bout de papier un peu de métal argenté.
Il est intéressant de remarquer que le texte l'indiquant s'est vu rétrécir au
fil du temps. La trace du mot \textit{certificat} a fini par totalement
disparaître, remplacée par la déclaration rassurante qu'il s'agit de billets de
la réserve fédérale.

Nous l'avons évoqué plus haut, il s'est passé la même chose avec l'or. Dans sa
majorité, le monde fonctionnait sur un étalon
bimétallique~\cite{wiki:bimetallism}, ce qui signifie que les pièces étaient
principalement composées d'or et d'argent. C'était à n'en pas douter une avancée
technologique d'avoir des certificats sur l'or, échangeables contre des pièces
de ce même métal. Le papier est plus pratique, plus léger et puisqu'il est
divisible arbitrairement en inscrivant simplement dessus un nombre plus petit,
il est facile de le réduire en plus petites unités.

Afin de rappeler aux porteurs (utilisateurs) que ces certificats représentaient
de l'or ou de l'argent bien réels, ils revêtaient une couleur évocatrice et
l'indiquaient clairement en toutes lettres. Vous pouvez facilement lire ce
message de haut en bas :

\begin{quotation}\begin{samepage}
\enquote{Il est certifié par la présente que cent dollars en pièces d'or,
payables au porteur sur demande, ont été déposés au trésor des États-Unis
d'Amérique.}
\end{samepage}\end{quotation}

\begin{figure}
  \centering
  \includegraphics{assets/images/us-gold-cert-100-smaller.png}
  \caption{Un certificat américain sur l'or de 100\$ de 1928. Crédit photo
  CC-BY-SA Collection numismatique nationale, Musée National de l'Histoire
  américaine.}
  \label{fig:us-gold-cert-100-smaller}
\end{figure}

En 1963, les mots \enquote{PAYABLES AU PORTEUR SUR DEMANDE} furent retirés de
tous les nouveaux billets. Cinq ans plus tard, il fut mis fin à la
convertibilité des billets en or ou en argent.

Les mots qui rappelaient l'origine de la monnaie papier et l'idée qui
l'accompagne furent supprimés. La couleur dorée disparut. Il ne restait plus que
le papier et avec lui, pour le gouvernement, la possibilité d'en imprimer autant
qu'il le voulait.

Ce tour de passe-passe long de plus d'un siècle fut achevé en 1971 par
l'abolition de l'étalon-or. L'argent devint l'illusion que nous partageons tous
dorénavant : la monnaie fiduciaire. Elle a de la valeur car quelqu'un, qui
commande une armée et gère des prisons, a dit qu'elle en avait. On peut très
clairement le lire sur chaque dollar en circulation aujourd'hui, \enquote{CE
BILLET A COURS LÉGAL}. Autrement dit : il vaut quelque chose parce que c'est
écrit dessus.

\begin{figure}
  \centering
  \includegraphics{assets/images/us-dollar-2004.jpg}
  \caption{Un billet américain de vingt dollars de 2004 utilisé de nos jours.
  `CE BILLET A COURS LÉGAL'}
  \label{fig:us-dollar-2004}
\end{figure}

À ce propos, il y a une autre leçon intéressante sur les billets modernes qui se
cache sous nos yeux. La deuxième ligne indique que le cours légal concerne
\enquote{TOUTES LES DETTES, PUBLIQUES ET PRIVÉES}. J'ai été surpris par ce qui
peut paraître une évidence pour les économistes : tout l'argent consiste en de
la dette. J'en ai encore des migraines, je laisserai donc au lecteur la tâche
d'étudier le lien entre l'argent et la dette.

\paragraph{}
Nous l'avons vu, l'or et l'argent ont été utilisés comme monnaie pendant des
millénaires. Au fil du temps, les pièces d'or et d'argent furent remplacées par
du papier. Celui-ci a lentement été accepté comme moyen de paiement. Cette
adoption créa une illusion --- l'illusion que le papier lui-même a de la valeur.
Le coup de grâce fut de totalement rompre le lien entre la représentation et le
réel : abolir l'étalon-or en convainquant tout le monde que c'est le papier qui
est précieux.

\paragraph{Bitcoin m'a appris l'Histoire de la monnaie et le plus important tour
de passe-passe dans l'Histoire de l'économie : la monnaie fiduciaire.}

% ---
%
% #### Down the Rabbit Hole
%
% - [Shelling Out: The Origins of Money] by Nick Szabo
% - [Methods of Coin Debasement][coin debasement], [Thaler], [U.S. Silver Certificate][silver certificates], [Bimetallism][bimetallic standard] on Wikipedia
%
% [oldest coin]: https://www.britishmuseum.org/explore/themes/money/the_origins_of_coinage.aspx
% [coin debasement]: https://en.wikipedia.org/wiki/Methods_of_coin_debasement
% [Thaler]: https://en.wikipedia.org/wiki/Thaler
% [Berlin-George]: https://en.wikipedia.org/wiki/File:Bohemia,_Joachimsthaler_1525_Electrotype_Copy._VF._Obverse..jpg
% [silver certificates]: https://en.wikipedia.org/wiki/Silver_certificate_%28United_States%29
% [bimetallic standard]: https://en.wikipedia.org/wiki/Bimetallism
% [Shelling Out: The Origins of Money]: https://nakamotoinstitute.org/shelling-out/
%
% <!-- Wikipedia -->
% [alice]: https://en.wikipedia.org/wiki/Alice%27s_Adventures_in_Wonderland
% [carroll]: https://en.wikipedia.org/wiki/Lewis_Carroll

\chapter{La folie des réserves fractionnaires}
\label{les:13}

\begin{chapquote}{Lewis Carroll, \textit{Alice au pays des merveilles}}
Hélas ! les regrets étaient inutiles ! Elle continuait à grandir sans arrêt, et,
bientôt, elle fût obligée de s’agenouiller sur le plancher : une minute plus
tard, elle n’avait même plus assez de place pour rester à genoux, et elle
essayait de voir si elle serait mieux en se couchant, un coude contre la porte,
son autre bras replié sur la tête. Puis, comme elle ne cessait toujours pas de
grandir, elle passa un bras par la fenêtre, mit un pied dans la cheminée, et se
dit : \enquote{À présent je ne peux pas faire plus, quoi qu’il arrive. Que
vais-je devenir ?}
\end{chapquote}

La valeur et la monnaie ne sont pas des sujets simples, particulièrement de nos
jours. Le processus d'émission monétaire du système bancaire ne l'est pas plus
et je ne peux m'empêcher de croire que c'est délibéré. Ce phénomène, que je
n'avais jusqu'à présent rencontré que dans les papiers de recherche et les
documents légaux, semble également courant dans les milieux financiers : rien
n'est expliqué de façon simple, non pas parce que c'est réellement complexe,
mais parce que la vérité est dissimulée sous des couches et des couches de
jargon d'une complexité \textit{apparente}. \enquote{Politique monétaire
expansionniste, assouplissement quantitatif, relance fiscale de l'économie}. Le
public hoche de la tête, hypnotisé par les mots savants.

Les banques à réserve fractionnaire et l'assouplissement quantitatif sont
justement deux de ces mots sophistiqués, dissimulant la réalité des choses en la
présentant comme complexe et difficile à comprendre. Si vous deviez expliquer ça
à un enfant, vous verriez rapidement la folie qu'ils renferment.

Godfrey Bloom l'a exprimé bien mieux que je ne pourrai jamais le faire, en
s'adressant au Parlement Européen lors d'un débat commun :

\begin{quotation}\begin{samepage}
\enquote{[...] vous ne comprenez pas vraiment le concept de banque. Toutes les
banques sont fauchées. La Santander, la Deutsche Bank, la Royal Bank of Scotland
--- elles sont toutes fauchées ! Et pourquoi elles sont fauchées ? Ce
n'est pas une volonté divine. Ce n'est pas une espèce de tsunami. Elles sont
fauchées parce qu'elles ont un système appelé `réserves fractionnaires' qui leur
permet de prêter de l'argent qu'elles n'ont même pas ! C'est un scandale
orchestré par des malfaiteurs et ça fait trop longtemps que ça dure. [...] Il y
a de la contrefaçon --- qu'on appelle parfois assouplissement quantitatif ---
mais ça reste de la contrefaçon sous un autre nom. La création monétaire
artificielle coûterait au simple quidam un long moment derrière les barreaux
[...] et tant que l'on n'enverra pas les banquiers --- et j'inclus ici les
banquiers centraux et les politiques --- en prison pour ce scandale cela
continuera.}
\begin{flushright} -- Godfrey Bloom\footnote{Débat commun sur l'union
bancaire~\cite{godfrey-bloom}}
\end{flushright}\end{samepage}\end{quotation}

Permettez-moi de répéter le passage essentiel : les banques peuvent prêter de
l'argent qu'elles ne possèdent pas.

Grâce au système de réserve fractionnaire, une banque a besoin de ne garder
qu'une \textit{fraction} de chaque dollar qu'elle reçoit. Ça se situe quelque
part entre $0$ et $10\%$, plutôt vers la limite inférieure d'ailleurs, ce qui
n'arrange rien.

Prenons un exemple concret afin de mieux illustrer cette idée insensée : une
fraction de $10\%$ fera l'affaire et nous devrions pouvoir calculer de tête. 
Avantageux pour tout le monde. Donc, si vous déposez 100\$ à la banque ---
parce que vous n'avez pas envie de les garder sous votre matelas --- celle-ci
n'a besoin de garder que la \textit{fraction} consentie. Dans notre exemple,
cela fait 10\$, car 10\% de 100\$ font 10\$. Simple, non ?

Mais alors que font les banques avec le reste de l'argent ? Qu'arrive-t-il aux
90\$ restants ? Elles font ce que les banques savent faire, elles les prêtent à
d'autres. Cela produit un effet multiplicateur sur la monnaie, qui accroît
énormément son offre dans l'économie (Figure~\ref{fig:money-multiplier}). Votre
dépôt initial de 100\$ se transforme rapidement en 190\$. En prêtant 90\% de ces
90\$ fraîchement créés, cela fera bientôt 271\$ dans l'économie. Et 343,90\$
après ça. L'offre de monnaie gonfle de façon récursive, puisque les banques
prêtent littéralement de l'argent qu'elles n'ont
pas~\cite{wiki:money-multiplier}. Sans la moindre formule, les banques changent
100\$ en plus d'un milier, comme par magie. Il s'avère qu'il est simple
d'arriver à un facteur 10. Ça prend seulement quelques cycles de prêt.

\begin{figure}
  \centering
  \includegraphics{assets/images/money-multiplier.png}
  \caption{L'effet multiplicateur sur la monnaie}
  \label{fig:money-multiplier}
\end{figure}

\paragraph{}
Ne vous méprenez pas : il n'y a rien de mal à prêter. Il n'y a rien de mal à
percevoir des intérêts. Il n'y a même rien de mal avec cette bonne vieille
banque qui garde votre patrimoine plus sûrement que dans votre tiroir à
chaussettes.

Les banques centrales sont une toute autre affaire, en revanche. Elles sont les
abominations de la régulation financière, mi-publiques mi-privées, jouant à Dieu
avec des choses qui impactent tout membre de la civilisation mondiale, sans
morale, avec pour seul intérêt le futur à court terme et apparemment aucune
responsabilité ni vérifiabilité (voir la Figure~\ref{fig:bsg}).

\begin{figure}
  \centering
  \includegraphics{assets/images/bsg.jpg}
  \caption{Yellen est fermement opposée à un audit de la réserve fédérale,
  pendant que le gars au panneau Bitcoin soutient fermement l'achat de bitcoin.}
  \label{fig:bsg}
\end{figure}

Bien que Bitcoin soit encore inflationniste, il cessera de l'être relativement
rapidement. La limite stricte sur l'offre de 21 millions de bitcoins finira par
éliminer totalement l'inflation. Nous avons maintenant deux mondes monétaires :
l'un inflationniste où l'argent est créé arbitrairement et le monde de Bitcoin,
où l'offre finale est fixe et facilement vérifiable par tout un chacun. L'un
nous est imposé par la violence, l'autre peut être rejoint par quiconque le
désire. Pas de barrière à l'entrée, personne à qui demander la permission. Une
participation volontaire. Voilà la beauté de Bitcoin.

J'ajouterais que le débat entre les économistes Keynésiens\footnote{Les théories
de John Maynard Keynes et ses adeptes~\cite{wiki:keynesian}} et l'école
autrichienne\footnote{École de pensée économique basée sur l'individualisme
méthodologique~\cite{wiki:austrian}} n'est plus seulement académique. Satoshi
est parvenu à créer un système de transfert de valeur sous stéroïdes, inventant
dans le même temps la monnaie la plus saine ayant jamais existé. D'une façon ou
d'une autre, de plus en plus de gens prendront conscience de l'arnaque qu'est le
système de réserve fractionnaire. S'ils en arrivent aux mêmes conclusions que la
plupart des Bitcoiners et économistes de l'école autrichienne, ils pourraient
rejoindre l'internet de l'argent, qui ne cesse de grandir. Personne ne peut les
empêcher de faire ce choix.

\paragraph{Bitcoin m'a appris que les réserves fractionnaires des banques
n'étaient que pure folie.}

% ---
%
% #### Down the Rabbit Hole
%
% - [The Creature From Jekyll Island] by G. Edward Griffin
% - [Money Multiplier][money multiplier], [Keynesian Economics][Keynesian], [Austrian School][Austrian] on Wikipedia
%
% [The Creature From Jekyll Island]: https://archive.org/details/pdfy--Pori1NL6fKm2SnY
%
% [joint debate]: https://www.youtube.com/watch?v=hYzX3YZoMrs
% [money multiplier]: https://en.wikipedia.org/wiki/Money_multiplier
% [auditability]: https://i.ytimg.com/vi/ThFGs347MW8/maxresdefault.jpg
% [Keynesian]: https://en.wikipedia.org/wiki/Keynesian_economics
% [Austrian]: https://en.wikipedia.org/wiki/Austrian_School
%
% <!-- Wikipedia -->
% [alice]: https://en.wikipedia.org/wiki/Alice%27s_Adventures_in_Wonderland
% [carroll]: https://en.wikipedia.org/wiki/Lewis_Carroll

\chapter{Une monnaie saine}
\label{les:14}

\begin{chapquote}{Lewis Carroll, \textit{Alice au pays des merveilles}}
\enquote{La première chose que je dois faire,} se dit-elle tout en marchant dans
le bois à l’aventure, \enquote{c’est retrouver ma taille normale ; la seconde,
c’est de trouver le chemin qui mène à ce charmant jardin. Je crois que c’est un
très bon plan.}
\end{chapquote}

La leçon la plus importante que j'ai tirée de Bitcoin c'est qu'en fin de compte,
la monnaie forte est meilleure que la monnaie faible. La monnaie forte,
également appelée \textit{monnaie saine}, consiste en toute monnaie échangeable
sur le marché mondial pouvant servir de réserve de valeur solide.

D'accord, Bitcoin est encore jeune et volatil. Les critiques avanceront qu'il
n'est pas fiable en tant que réserve de valeur. Mais l'argument de la volatilité
passe à côté du sujet. Il faut s'attendre à de la volatilité. Ça prendra un
moment au marché pour déterminer le juste prix de cette nouvelle monnaie. De
plus, il est fondé sur une erreur de métrique, comme le souligne une
plaisanterie récurrente. Si vous réfléchissez en dollars, vous passerez à côté
du fait qu'un bitcoin vaudra toujours un bitcoin.

\begin{quotation}\begin{samepage}
\enquote{Une offre monétaire fixe, ou une offre modifiée uniquement sur la base
de critères objectifs et calculables, est une condition nécessaire à un prix de
la monnaie juste et significatif.}
\begin{flushright} -- Père Bernard W. Dempsey, S.J.\footnote{Perry J. Roets,
S.J., \textit{Revue de l'économie sociale} \cite{review-social-economy}}
\end{flushright}\end{samepage}\end{quotation}

\newpage

Comme l'a montré une courte ballade dans le cimetière des monnaies disparues,
l'argent qui peut être créé le sera. Dans l'Histoire, aucun être humain n'a su
jusqu'à présent résister à la tentation.

Bitcoin élimine cette tentation de la création monétaire de façon astucieuse.
Satoshi avait conscience de notre cupidité et de notre faillibilité --- c'est
pour cela qu'il a choisi une chose plus fiable que le contrôle humain : les
mathématiques.

\begin{figure}
  \centering
  \begin{equation}
  \sum\limits_{i=0}^{32} \frac{210000 \lfloor \frac{50*10^8}{2^i} \rfloor}{10^8}
  \end{equation}
  \caption{Formule de l'offre de Bitcoin}
  \label{fig:supply-formula-white}
\end{figure}

Bien que la formule ci-dessus soit utile pour décrire l'offre de Bitcoin, on ne
la trouve nulle part dans le code. L'émission de nouveaux bitcoins est contrôlée
par un algorithme, qui réduit tous les quatre ans~\cite{btcwiki:supply} la
récompense payée aux mineurs. Cette formule permet donc de résumer rapidement ce
qui se passe sous le capot. On comprend mieux ce qui se passe vraiment en
regardant la variation des récompenses de bloc, payées à quiconque trouve un
bloc valide, ce qui arrive à peu près toutes les 10 minutes.

\begin{figure}
  \includegraphics{assets/images/you-are-here.png}
  \caption{L'offre contrôlée de Bitcoin}
  \label{fig:you-are-here.png}
\end{figure}

Les formules, les fonctions logarithmiques et les exponentielles ne sont pas
particulièrement intuitives à comprendre. Le concept de \textit{sain} peut
s'appréhender plus facilement si on le voit autrement. Une fois que l'on sait
combien il existe d'une chose et que l'on sait combien cette chose est
difficile à produire ou à obtenir, nous comprenons immédiatement sa valeur. Ce
qui est vrai avec un tableau de Picasso, une guitare d'Elvis Presley ou un
violon de Stradivarius est également vrai pour la monnaie fiduciaire, l'or et
les bitcoins. 

La dureté des monnaies fiduciaires dépend des responsables de leurs planches à
billets respectives. Certains gouvernements seront sans doute plus enclins à
créer de plus larges quantités de monnaie que d'autres, aboutissant à une
monnaie plus faible. D'autres gouvernements seront plus modérés sur leur
émission monétaire, entraînant une monnaie plus forte.

\begin{samepage}\begin{quotation}
\enquote{Un aspect important de cette nouvelle réalité est que les institutions
telles que la réserve fédérale ne peuvent pas faire faillite. Elles peuvent
créer autant d'argent qu'elles en ont besoin pour un coût virtuellement nul.}
\begin{flushright} -- Jörg Guido Hülsmann\footnote{Jörg Guido Hülsmann,
\textit{L'éthique de la création monétaire}~\cite{hulsmann2008ethics}}
\end{flushright}\end{quotation}\end{samepage}

Avant que nous n'ayons des monnaies fiduciaires, la dureté de l'argent était
déterminée par les propriétés naturelles de ce qui servait de monnaie. La
quantité d'or sur Terre est limitée par les lois de la physique. L'or est rare
car les collisions de supernovas et d'étoiles à neutrons sont rare. Le
\enquote{flux} d'or est limité car il demande des efforts à extraire. Comme
c'est un élément lourd, il est en majorité enterré bien profondément dans le
sol.

L'abolition de l'étalon-or a engendré une nouvelle réalité : il suffit d'un peu
d'encre pour créer de l'argent. Dans le monde actuel, ça demande encore moins
d'efforts d'ajouter quelques zéros au solde d'un compte bancaire : il suffit de
modifier quelques octets sur l'ordinateur d'une banque.

On peut exprimer plus généralement le principe énoncé ci-dessus comme étant le
rapport entre les \enquote{stocks} et les \enquote{flux}. Plus simplement, le
\textit{stock} représente la quantité existante de quelque chose. Pour nos
besoins, le stock est la mesure de l'offre actuelle de monnaie. Le \textit{flux}
quantifie la production de cette même chose sur une durée donnée (par an, par
exemple). La clé pour comprendre la monnaie saine est de comprendre le rapport
stock-à-flux.

Calculer le rapport stock-à-flux de la monnaie fiduciaire est complexe, car
l'offre de monnaie dépend de ce que vous y intégrez~\cite{wiki:money-supply}.
Vous pouvez ne compter que les billets et les pièces (M0), ajouter les chèques
de voyage et les remises de chèques (M1), ajouter les comptes épargne, les fonds
communs et quelques autres trucs (M2) et même ajouter à tout ça les certificats
de dépôt (M3). De plus, la façon dont tout cela est défini et calculé dépend de
chaque pays et puisque la réserve fédérale des États-Unis a cessé de publier
\cite{web:fed-m3} les chiffres pour M3, nous allons devoir faire avec l'offre
monétaire M2. J'adorerais pouvoir vérifier ces chiffres, mais pour le moment
j'imagine que nous devrons faire confiance à la réserve fédérale.

C'est l'or, l'un des métaux les plus rares sur Terre, qui a le rapport
stock-à-flux le plus élevé. Selon l'Institut d'études géologiques des
États-Unis, un peu plus de 190 000 tonnes en ont été minées au total. Au cours
des dernières années, environ 3100 tonnes ont été minées par
an~\cite{mineral-commodity-summaries}.

À partir de ces chiffres, nous pouvons facilement calculer le rapport
stock-à-flux de l'or (voir la Figure~\ref{fig:stock-to-flow-gold}).

\begin{figure}
  \centering
  \begin{equation}
  \frac{190,000 t}{3,100 t} = ~ 61
  \end{equation}
  \caption{Rapport stock-à-flux de l'or}
  \label{fig:stock-to-flow-gold}
\end{figure}

Il n'y a rien qui ait un rapport stock-à-flux plus élevé. C'est pour cette
raison que l'or, jusqu'à maintenant, était la monnaie la plus forte et la plus
saine qui soit. On raconte souvent que tout l'or qui a déjà été miné tiendrait
dans deux piscines olympiques. Selon mes
calculs\footnote{\url{https://bit.ly/gold-pools}}, il en faudrait quatre. Donc
soit les piscines olympiques ont rétréci, soit il faudrait peut-être revoir ça.

Arrive alors le Bitcoin. Comme vous le savez sans doute déjà, le minage de
bitcoin fait fureur depuis quelques années. Cela s'explique car nous sommes
encore au début de ce qu'on appelle \textit{le temps des récompenses}, où les
nœuds de minage sont récompensés avec \textit{beaucoup} de bitcoin pour leurs
efforts de calcul. Nous sommes en ce moment dans l'époque numéro 3, qui a débuté
en 2016 et s'achèvera au début de 2020, sans doute en mai. Tandis que l'offre de
bitcoins est limitée, les rouages internes de Bitcoin ne permettent d'établir
que des dates approximatives. Pourtant, nous pouvons prédire avec certitude à
quel niveau le rapport stock-à-flux de Bitcoin se situera. Alerte spoiler : ça
sera élevé.

Élevé comment ? Eh bien, il s'avère que Bitcoin finira par devenir infiniment
fort (voir la Figure~\ref{fig:stock-to-flow-white-cropped}).

\begin{figure}
  \includegraphics{assets/images/stock-to-flow-white-cropped.png}
  \caption{Visualisation du stock et du flux du dollar US, de l'or et de
  Bitcoin}
  \label{fig:stock-to-flow-white-cropped}
\end{figure}

\paragraph{}
À cause de la réduction exponentielle des récompenses de minage, le flux de
nouveaux bitcoins va diminuer, engendrant un rapport stock-à-flux qui grimpe en
flèche. Il rattrapera l'or en 2020, pour mieux le surpasser quatre ans plus tard
en doublant à nouveau sa dureté. Au total, un tel doublement se produira 64
fois. Grâce à la puissance des exponentielles, le nombre de bitcoins minés par
an tombera à moins de 100 bitcoins dans 50 ans et à moins de 1 bitcoin dans 75
ans. Le robinet mondial que représentent les récompenses de bloc se tarira aux
environs de l'année 2140, mettant effectivement un terme à la production de
bitcoin. C'est un jeu de longue haleine. Si vous lisez ceci, vous êtes encore en
avance.

\begin{figure}
  \includegraphics{assets/images/soundness-over-time.png}
  \caption{Le rapport stock-à-flux du bitcoin comparé à l'or}
  \label{fig:soundness-over-time}
\end{figure}

Alors que le bitcoin tend vers un rapport stock-à-flux infini, il deviendra la
monnaie la plus saine qui soit. La dureté infinie semble difficile à battre.

D'un point de vue économique, \textit{l'ajustement de la difficulté} de Bitcoin
est son aspect le plus important. La difficulté à miner du bitcoin dépend de la
rapidité avec laquelle de nouveaux bitcoins sont minés\footnote{En réalité ça
dépend de la vitesse à laquelle des blocs valides sont trouvés, mais pour nos
besoins, c'est la même chose que de \enquote{miner des bitcoins} et ça le
restera pour les 120 prochaines années.}. C'est l'ajustement dynamique de la
difficulté de minage du réseau qui nous permet de prédire son offre future.

La simplicité de l'algorithme d'ajustement de la difficulté pourrait détourner
de sa profondeur, mais il est véritablement une révolution aux proportions
dignes d'Einstein. Il garantit que quels que soient les efforts déployés dans le
minage, l'offre maîtrisée de Bitcoin ne sera pas perturbée. À la différence de
toutes les autres ressources, peu importe l'énergie dépensée par quelqu'un dans
le minage de bitcoin, la récompense totale n'augmentera pas.

Tout comme $E=mc^2$ impose une limite universelle à la vitesse dans notre
univers, l'ajustement de la difficulté de minage impose sa \textbf{limite
monétaire universelle} à Bitcoin.

\paragraph{}
Sans cet ajustement de la difficulté, tous les bitcoins auraient déjà été minés.
Sans cet ajustement de la difficulté, Bitcoin n'aurait probablement pas survécu
à ses premiers pas. C'est ce qui sécurise le réseau durant le temps des
récompenses. C'est ce qui garantit une distribution stable et
impartiale\footnote{Dan Held, \textit{La distribution de Bitcoin était
juste}~\cite{distribution-was-fair}} des nouveaux bitcoins. C'est le thermostat
qui régule la politique monétaire de Bitcoin.

Einstein nous a enseigné une chose novatrice : peu importe la force imprimée à
un objet, à un moment donné vous ne pourrez pas le faire aller plus vite.
Satoshi nous a aussi enseigné une chose novatrice : peu importent les efforts
mis dans le minage de cet or numérique, à un moment donné vous ne pourrez pas en
tirer plus de bitcoins. Pour la première fois dans l'Histoire de l'Humanité,
nous avons un bien monétaire dont vous ne pourrez pas augmenter la production,
peu importe à quel point vous essaierez.

\paragraph{Bitcoin m'a appris que la monnaie saine était indispensable.}

% ---
%
% #### Through the Looking-Glass
%
% - [Bitcoin's Energy Consumption: A Shift in Perspective][much energy]
%
% #### Down the Rabbit Hole
%
% - [The Ethics of Money Production][Jörg Guido Hülsmann] by Jörg Guido Hülsmann
% - [Mineral Commodity Summaries 2019][last few years] by the United States Geological Survey
% - [Bitcoin’s Distribution was Fair][fair distribution] by Dan Held
% - [Bitcoin's Controlled Supply][algorithmically controlled] on the Bitcoin Wiki
% - [Money Supply][how much money there is], [Speed of Light][universal speed limit] on Wikipedia
%
% <!-- Internal -->
% [much energy]: 
%
% [Fr. Bernard W. Dempsey, S.J.]: https://www.jstor.org/stable/29769582
% [Jörg Guido Hülsmann]: https://mises.org/sites/default/files/The%20Ethics%20of%20Money%20Production_2.pdf
% [stopped publishing]: https://www.federalreserve.gov/Releases/h6/discm3.htm
% [last few years]: https://minerals.usgs.gov/minerals/pubs/mcs/2018/mcs2018.pdf
% [my calculations]: https://www.wolframalpha.com/input/?i=volume+of+190000+metric+tons+gold+%2F+olympic+swimming+pool+volume
% [fair distribution]: https://blog.picks.co/bitcoins-distribution-was-fair-e2ef7bbbc892
%
% <!-- Bitcoin Wiki -->
% [algorithmically controlled]: https://en.bitcoin.it/wiki/Controlled_supply
%
% <!-- Wikipedia -->
% [how much money there is]: https://en.wikipedia.org/wiki/Money_supply
% [universal speed limit]: https://en.wikipedia.org/wiki/Speed_of_light#Upper_limit_on_speeds
% [alice]: https://en.wikipedia.org/wiki/Alice%27s_Adventures_in_Wonderland
% [carroll]: https://en.wikipedia.org/wiki/Lewis_Carroll

\part{Technologie}
\label{ch:technology}
\chapter*{Technologie}

\begin{chapquote}{Lewis Carroll, \textit{Alice au pays des merveilles}}
\enquote{Cette fois-ci, je vais m’y prendre un peu mieux} se dit-elle, et elle
commença par s’emparer de la petite clé d’or et par ouvrir la porte qui donnait
sur le jardin.
\end{chapquote}

Des clés d'or, des horloges qui ne fonctionnent qu'au hasard, des courses pour
résoudre d'étranges énigmes et des bâtisseurs anonymes et sans visages. On
dirait des contes de fée tirés du pays des merveilles mais c'est pourtant la
routine dans l'univers Bitcoin.

Comme nous l'avons vu dans le Chapitre~\ref{ch:economics}, des pans entiers du
système financier actuel sont systématiquement défaillants. À la manière
d'Alice, nous ne pouvons qu'espérer faire mieux cette fois-ci. Pourtant, grâce à
un inventeur pseudonyme, nous disposons dorénavant d'une technologie
incroyablement sophistiquée pour nous aider : Bitcoin.

Résoudre des problèmes dans un environnement radicalement décentralisé et
hostile demande des solutions uniques. Des problèmes habituellement triviaux à
résoudre n'ont rien de trivial dans cet étrange monde fait de nœuds. Pour la
plupart de ses solutions, Bitcoin repose sur une cryptographie solide, tout du
moins du point de vue de la technologie. Nous verrons dans une des leçons qui
suivent à quel point cette cryptographie est solide.

Bitcoin se sert de la cryptographie pour se détacher de la confiance dans les
autorités. Au lieu d'être tributaire d'institutions centralisées, le système
s'appuie sur l'autorité ultime de notre univers : la physique. Cependant, il
reste tout de même quelques traces de confiance. Nous examinerons ces traces
dans la seconde leçon de ce chapitre.

~

\begin{samepage}
Partie~\ref{ch:technology} -- Technologie:

\begin{enumerate}
  \setcounter{enumi}{14}
  \item La force dans les nombres
  \item Remarques sur \enquote{Ne vous fiez pas, vérifiez}
  \item Donner l'heure demande du travail
  \item Avancer lentement sans rien casser
  \item La vie privée n'est pas morte
  \item Les cypherpunks écrivent du code
  \item Métaphores pour le futur de Bitcoin
\end{enumerate}
\end{samepage}

Les leçons suivantes traitent de l'éthique du développement technologique de
Bitcoin, un aspect potentiellement aussi important que la technologie elle-même.
Bitcoin, ce n'est pas la prochaine appli à la mode sur votre téléphone. C'est la
base d'une nouvelle réalité économique, ce qui explique qu'il devrait être
considéré comme un logiciel financier de qualité nucléaire.

Où en sommes-nous de cette révolution financière, sociétale et technologique ?
Les réseaux et technologies du passé peuvent figurer des métaphores au futur de
Bitcoin, que nous aborderons dans la toute dernière leçon de ce chapitre.

Une dernière fois, attachez vos ceintures et profitez de la descente. Comme
toutes les technologies exponentielles, nous allons devenir paraboliques.

\chapter{La force dans les nombres}
\label{les:15}

\begin{chapquote}{Lewis Carroll, \textit{Alice au pays des merveilles}}
\enquote{Voyons un peu : quatre fois cinq font douze, quatre fois six font
treize, et quatre fois sept font… Oh ! mon Dieu ! jamais je n’arriverai jusqu’à
vingt à cette allure !}
\end{chapquote}

Les nombres sont indispensables à notre vie quotidienne. Les grands nombres, en
revanche, sont peu familiers à la plupart d'entre nous. Les plus grands nombres
que l'on rencontrera généralement au jour le jour seront de l'ordre des
millions, des milliards voire des billions\footnote{NdT : \textit{billion} est
la traduction de l'anglais \textit{trillion}.}. On pourra par exemple parler de
millions de gens dans la misère, de milliards de dollars dépensés à renflouer
les banques ou encore de billions de dette publique. Malgré la difficulté à se
représenter ce genre de gros titres, nous sommes relativement à l'aise avec
l'ordre de grandeur de ces nombres.

Même si nous sommes familiers avec les milliards et les billions, notre
intuition fait déjà défaut face à leur magnitude. Avez-vous une idée du temps
qu'il faut pour qu'un million, milliard ou billion de secondes ne s'écoulent ?
Si vous êtes comme moi, vous êtes perdu si vous ne faites pas les calculs.

Creusons cet exemple : la différence entre chaque représente trois ordres de
magnitude ; $10^6$, $10^9$, $10^{12}$. Ce n'est pas très utile de penser en
secondes, alors transformons-les afin de pouvoir les appréhender :

\begin{itemize}
  \item $10^6$: Un million de secondes représentent la dernière semaine et demie.
  \item $10^9$: Un milliard de secondes représentent les 32 dernières années.
  \item $10^{12}$: Il y a un billion de secondes, Manhattan était couverte d'une
  épaisse couche de glace\footnote{Un billion de secondes ($10^{12}$)
  représentent les $31710$ dernières années. Le Dernier Maximum Glaciaire a eu
  lieu il y a $33,000$ ans.~\cite{wiki:LGM}}.
\end{itemize}

\begin{figure}
  \includegraphics{assets/images/xkcd-1225.png}
  \caption{Il y a environ 1 billion de secondes. Source : xkcd n°1225}
  \label{fig:xkcd-1225}
\end{figure}

Au moment où nous entrons dans les échelles astronomiques de la cryptographie
moderne, notre instinct échoue dans les grandes largeurs. Bitcoin est bâti
autour des grands nombres et de l'impossibilité virtuelle à les deviner. Ces
nombres sont bien, bien plus grands que ceux qu'on peut rencontrer au quotidien.
Plus grands de beaucoup d'ordres de magnitude. Il est indispensable
d'appréhender à quel point ces nombres sont réellement grands pour pouvoir
comprendre Bitcoin dans son ensemble.

Prenons comme exemple concret SHA-256\footnote{SHA-256 fait partie de la famille
des fonctions de hachage cryptographique SHA-2 développée par la
NSA.~\cite{wiki:sha2}}, l'une des fonctions de hachage\footnote{Bitcoin utilise
SHA-256 dans son algorithme de hachage de bloc.~\cite{btcwiki:block-hashing}}
utilisée par Bitcoin. Il est naturel de se dire que 256 bits ne sont que
\enquote{deux cent cinquante-six}, ce qui n'est pas du tout un grand nombre.
Pourtant, le nombre dans SHA-256 représente des ordres de magnitude -- une chose
pour laquelle nos cerveaux ne sont pas équipés.

Bien que le nombre de bits soit une métrique appropriée, le vrai sens d'une
sécurité 256-bit se perd dans l'interprétation. À l'image des millions ($10^6$)
et des milliards ($10^9$) ci-dessus, le nombre dans SHA-256 parle d'ordres
de magnitude ($2^{256}$).

Mais alors à quel point SHA-256 est-il solide, au juste ?

\begin{quotation}\begin{samepage}
\enquote{SHA-256 est très solide. Ce n'est pas incrémental comme passer de MD5 à
SHA1. Ça pourrait prendre plusieurs décennies avant une attaque massive
révolutionnaire.}
\begin{flushright} -- Satoshi Nakamoto\footnote{Satoshi Nakamoto, dans une
réponse aux questions sur les collisions SHA-256. \cite{satoshi-sha256}}
\end{flushright}\end{samepage}\end{quotation}

Appelons un chat un chat. $2^{256}$ représente ce nombre :

\begin{quotation}\begin{samepage}
115 duodecilliards 792 duodecillions 89 undecilliards 237 undecillions 316
decilliards 195 decillions 423 nonilliards 570 nonillions 985 octilliards 8
octillions 687 septilliards 907 septillions 853 sextilliards 269 sextillions 984
quintilliards 665 quintillions 640 quadrilliards 564 quadrillions 39 trilliards
457 trillions 584 billiards 7 billions 913 milliards 129 millions 639 mille 936.
\end{samepage}\end{quotation}

Ça fait un paquet de quintillions ! Se faire une idée de ce nombre est somme
toute impossible. Il n'existe rien dans l'univers physique à quoi le comparer.
Ça représente bien plus que le nombre d'atomes de l'univers observable. Le
cerveau humain n'est tout simplement pas fait pour le comprendre.

\newpage

On trouve l'une des meilleures représentations de la vraie force de SHA-256 dans
une vidéo de Grant Sanderson. Judicieusement nommée \textit{\enquote{À quel
point la sécurité 256-bit est-elle sûre ?}}\footnote{Regardez la vidéo sur
\url{https://youtu.be/S9JGmA5_unY}}, elle montre parfaitement l'immensité d'un
espace de cette taille. Rendez-vous service et prenez cinq minutes pour la
regarder. Comme toutes les vidéos de \textit{3Blue1Brown} celle-ci est tout
autant fascinante qu'exceptionnellement bien faite. Mais attention : vous
pourriez bien tomber dans un terrier de lapin mathématique.

\begin{figure}
  \includegraphics{assets/images/youtube-vid-inverted.png}
  \caption{Illustration de la sécurité dans SHA-256. Schéma original de Grant
  Sanderson alias 3Blue1Brown.}
  \label{fig:youtube-vid-inverted}
\end{figure}

Bruce Schneier~\cite{web:schneier} s'est servi des limites physiques de
l'informatique afin de relativiser ce nombre : même si nous parvenions à
construire un ordinateur optimal, qui utiliserait sans perte l'énergie fournie
pour manipuler les bits~\cite{wiki:landauer}, que nous construisions une sphère
de Dyson\footnote{Une sphère de Dyson est une mégastructure hypothétique qui
entoure totalement une étoile et capture un grand pourcentage de son
énergie.~\cite{wiki:dyson}} autour du Soleil et que nous les laissions tourner
pendant 100 milliards de milliards d'années, nous n'aurions que $25\%$ de
chances de trouver une aiguille dans une botte de 256 bits.

\begin{quotation}\begin{samepage}
\enquote{Ces nombres n'ont rien à voir avec la technologie des appareils ; ce
sont les maximums autorisés par la thermodynamique. Et ils suggèrent fortement
que les attaques par force brute contre des clés de 256 bits ne seront pas
envisageables avant que les ordinateurs ne soient faits d'autre chose que la
matière et occupent autre chose que l'espace.}
\begin{flushright} -- Bruce Schneier\footnote{Bruce Schneier,
\textit{Cryptographie appliquée} \cite{bruce-schneier}}
\end{flushright}\end{samepage}\end{quotation}

On ne peut surestimer la profondeur de ces mots. Une cryptographie solide
renverse l'équilibre des pouvoirs du monde physique auquel nous sommes habitués.
Dans notre réalité, les choses incassables n'existent pas. Si vous y mettez
assez de force, vous pourrez ouvrir n'importe quelle porte, boîte ou coffre au
trésor.

Le coffre au trésor de Bitcoin est très différent. Il est protégé par une
cryptographie solide, qui ne laisse pas la place à la force brute. Tant que les
hypothèses mathématiques sous-jacentes s'appliqueront, nous n'aurons que cette
force brute à notre disposition. Bon, d'accord, il y a aussi l'option d'une
attaque globale à la clé à molette à 5\$ (Figure~\ref{fig:xkcd-538}). Mais la
torture ne fonctionnera pas pour toutes les adresses bitcoin et les remparts
cryptographiques de bitcoin mettront en échec les attaques par force brute.
Même si vous attaquez avec la puissance d'un millier d'étoiles. Littéralement.

\begin{figure}
  \centering
  \includegraphics[width=8cm]{assets/images/xkcd-538.png}
  \caption{Attaque à la clé à molette à 5\$. Source : xkcd n°538}
  \label{fig:xkcd-538}
\end{figure}

Ce fait et ses répercussions ont été résumés de façon bouleversante dans l'appel
aux armes cryptographiques : \textit{\enquote{Aucune force coercitive ne
résoudra jamais un problème de maths.}}

\begin{quotation}\begin{samepage}
\enquote{Ça n'avait rien d'évident, que le monde finirait par fonctionner comme
ça. Mais d'une façon ou d'une autre, l'univers sourit au chiffrement.}
\begin{flushright} -- Julian Assange\footnote{Julian Assange, \textit{Un appel
aux armes cryptographiques} \cite{call-to-cryptographic-arms}}
\end{flushright}\end{samepage}\end{quotation}

Personne ne sait encore avec certitude si le sourire de l'univers est
authentique. Il est possible que notre hypothèse des asymétries mathématiques
soit fausse et que nous découvrions qu'en réalité P est égal à NP
\cite{wiki:pnp} ou que nous trouvions étonnamment rapidement une solution à des
problèmes spécifiques \cite{wiki:discrete-log} que nous estimons actuellement
complexes. Si cela devait arriver, la cryptographie telle que nous la
connaissons disparaîtrait et les conséquences changeraient sans doute
radicalement la face du monde.

\begin{quotation}\begin{samepage}
\enquote{Vires in Numeris} = \enquote{Les forces dans les
nombres}\footnote{\textit{Vires in Numeris} à été proposé comme devise de
Bitcoin pour la première fois par l'utilisateur de bitcointalk
\textit{epii}~\cite{epii}}
\end{samepage}\end{quotation}

\textit{Vires in numeris} n'est pas qu'un slogan accrocheur pour les bitcoiners.
La prise de conscience d'une force inimaginable présente dans les nombres est
intense. En faire l'expérience et comprendre l'inversion dans l'équilibre des
pouvoirs existants qui en découle a changé ma façon de voir le monde et l'avenir
qui nous attend.

Un effet direct de cela, c'est que vous n'avez pas à demander la permission à
quiconque pour participer à Bitcoin. Il n'y a pas de page d'inscription, pas
d'entreprise qui en est responsable, pas d'agence publique à qui envoyer les
formulaires. Générez simplement un grand nombre et vous êtes à peu près paré à y
aller. L'autorité centrale sur la création des comptes, ce sont les
mathématiques. Et Dieu seul sait qui en est responsable.

\begin{figure}
  \includegraphics{assets/images/elliptic-curve-examples.png}
  \caption{Exemples de courbes elliptiques. Crédit schéma CC-BY-SA Emmanuel
  Boutet.}
  \label{fig:elliptic-curve-examples}
\end{figure}

Bitcoin est bâti sur notre meilleure compréhension de la réalité. Certes, il
reste bien des problèmes non-résolus en physique, en informatique et en
mathématiques, mais il y a des choses dont nous sommes plutôt sûrs. Qu'il y ait
une asymétrie entre trouver des solutions et valider la justesse de ces
solutions en est une. Que les calculs requièrent de l'énergie en est une autre.
En d'autres termes : trouver une aiguille dans une botte de foin est plus
difficile que de vérifier si le truc pointu dans votre main est bien une
aiguille. Et trouver l'aiguille prend du temps.

L'immensité de l'espace d'adressage de Bitcoin est vraiment ahurissante. Le
nombre de clés privées l'est encore plus. C'est fascinant de se rendre compte à
quel point notre monde moderne se résume à l'improbabilité de trouver une
aiguille dans une botte de foin incommensurable. J'en suis conscient plus que
jamais, dorénavant.

\paragraph{Bitcoin m'a appris que les nombres renfermaient de la puissance.}

% ---
%
% #### Down the Rabbit Hole
%
% - [How secure is 256 bit security?]["How secure is 256 bit security?"] by 3Blue1Brown
% - [Block Hashing Algorithm][hash functions] on the Bitcoin Wiki
% - [Last Glacial Maximum][thick layer of ice], [SHA-2][SHA-256], [Dyson Sphere][Dyson sphere], [Landauer's Principle][flip bits perfectly] [P versus NP][P actually equals NP], [Discrete Logarithm][specific problems] on Wikipedia
%
% [thick layer of ice]: https://en.wikipedia.org/wiki/Last_Glacial_Maximum
% [xkcd \#1125]: https://xkcd.com/1225/
% [SHA-256]: https://en.wikipedia.org/wiki/SHA-2
% [hash functions]: https://en.bitcoin.it/wiki/Block_hashing_algorithm
% ["How secure is 256 bit security?"]: https://www.youtube.com/watch?v=S9JGmA5_unY
% [Bruce Schneier]: https://www.schneier.com/
% [flip bits perfectly]: https://en.wikipedia.org/wiki/Landauer%27s_principle#Equation
% [Dyson sphere]: https://en.wikipedia.org/wiki/Dyson_sphere
% [2]: https://books.google.com/books?id=Ok0nDwAAQBAJ&pg=PT316&dq=%22These+numbers+have+nothing+to+do+with+the+technology+of+the+devices;%22&hl=en&sa=X&ved=0ahUKEwjXttWl8YLhAhUphOAKHZZOCcsQ6AEIKjAA#v=onepage&q&f=false
% [wrench attack]: https://xkcd.com/538/
% [call to cryptographic arms]: https://cryptome.org/2012/12/assange-crypto-arms.htm
% [P actually equals NP]: https://en.wikipedia.org/wiki/P_versus_NP_problem#P_=_NP
% [specific problems]: https://en.wikipedia.org/wiki/Discrete_logarithm#Cryptography
% [3Blue1Brown]: https://twitter.com/3blue1brown
%
% <!-- Wikipedia -->
% [alice]: https://en.wikipedia.org/wiki/Alice%27s_Adventures_in_Wonderland
% [carroll]: https://en.wikipedia.org/wiki/Lewis_Carroll

\chapter{Remarques sur \enquote{Ne vous fiez pas, vérifiez}}
\label{les:16}

\begin{chapquote}{Lewis Carroll, \textit{Les aventures d’Alice sous terre}}
\enquote{Préparez-vous à entendre les témoignages,} dit le Roi, \enquote{et
ensuite la sentence !}
\end{chapquote}

Bitcoin cherche à remplacer, ou tout du moins fournir une alternative aux
monnaies conventionnelles. Ces monnaies sont liées à une autorité centrale, peu
importe que l'on parle d'un cours légal comme le dollar américain ou des V-Bucks
de Fortnite. Dans les deux cas, vous êtes tenus de faire confiance à une
autorité centrale pour émettre, gérer et faire circuler votre argent. Bitcoin
rompt ce lien et résout cette préoccupation principale : la question de la
\textit{confiance}.

\begin{quotation}\begin{samepage}
\enquote{Le problème fondamental des monnaies traditionnelles, c'est toute la
confiance nécessaire à son fonctionnement. [...] Ce dont nous avons besoin,
c'est d'un système de paiement électronique basé sur la preuve cryptographique
au lieu de la confiance}
\begin{flushright} -- Satoshi Nakamoto\footnote{Satoshi Nakamoto, annonce
officielle de Bitcoin~\cite{bitcoin-announcement} et livre
blanc~\cite{whitepaper}}
\end{flushright}\end{samepage}\end{quotation}

Bitcoin remédie au problème de la confiance en étant totalement décentralisé,
sans serveur central ni parties de confiance. Pas seulement sans
\textit{tierces} parties de confiance, mais sans parties de confiance tout
court. Quand il n'y a pas d'autorité centrale, il n'y a \textit{personne} à qui
faire confiance, tout simplement. L'innovation c'est la décentralisation totale.
C'est la source de la ténacité de Bitcoin, la raison pour laquelle il est
toujours en vie. C'est aussi ce qui explique pourquoi nous avons du minage, des
nœuds, des portefeuilles physiques et oui, la blockchain. La seule chose à
laquelle vous devez faire \enquote{confiance}, c'est le fait que notre
connaissance des mathématiques et de la physique n'est pas totalement à côté de
la plaque et que la majorité des mineurs sont honnêtes (et ils sont incités à
l'être).

Tandis que le monde normal part du postulat \textit{\enquote{fiez vous, mais
vérifiez}}, Bitcoin se fonde sur le postulat \textit{\enquote{ne vous fiez pas,
vérifiez}}. Satoshi a très clairement insisté sur l'importance de se passer de
la confiance à la fois dans l'introduction et la conclusion du livre blanc de
Bitcoin.

\begin{quotation}\begin{samepage}
\enquote{Conclusion : nous avons proposé un système de transactions
électroniques se passant de confiance.}
\begin{flushright} -- Satoshi Nakamoto\footnote{Satoshi Nakamoto, livre blanc de
Bitcoin~\cite{whitepaper}}
\end{flushright}\end{samepage}\end{quotation}

Notez que \textit{se passant de confiance} est utilisé dans un contexte très
particulier. Nous parlons des tierces parties de confiance, c'est-à-dire
d'autres entités à qui vous vous fiez pour produire, détenir et traiter votre
argent. On partira par exemple du principe que vous pouvez faire confiance à
votre ordinateur.

Comme Ken Thompson l'a montré dans sa conférence au Turing Award, la confiance
est une chose extrêmement délicate en informatique. Quand vous lancez un
programme, vous devez vous fier à toutes sortes de logiciels (et de matériels)
qui, en théorie, pourraient modifier ce programme par malveillance. Pour citer
Thompson dans \textit{Remarques sur la confiance envers la confiance} :
\enquote{La morale est évidente. Vous ne pouvez pas faire confiance à du code
que vous n'avez pas entièrement écrit.}~\cite{trusting-trust}

\begin{figure}
  \includegraphics{assets/images/ken-thompson-hack.png}
  \caption{Extraits de l'article de Ken Thompson `Remarques sur la confiance
  envers la confiance'}
  \label{fig:ken-thompson-hack}
\end{figure}

Thompson a démontré que même si vous avez accès au code source, votre
compilateur --- ou tout autre programme ou matériel d'exécution --- pourrait
être corrompu et que la détection de cette porte dérobée serait très délicate.
Par conséquent, en pratique, un système sans aucun \textit{besoin de confiance}
n'existe pas. Pour ça vous auriez à créer tous vos logiciels (assembleurs,
compilateurs, éditeurs de liens, etc.) \textit{et} tout votre matériel de bout
en bout, sans l'aide d'un quelconque logiciel externe ou machine programmable.

\begin{quotation}\begin{samepage}
\enquote{Si vous voulez faire une tarte aux pommes à partir de rien, vous devez
d'abord inventer l'univers.}
\begin{flushright} -- Carl Sagan\footnote{Carl Sagan, \textit{Cosmos}
\cite{cosmos}}
\end{flushright}\end{samepage}\end{quotation}

Le piratage de Ken Thompson consiste en une porte dérobée particulièrement
astucieuse et difficile à détecter, qui fonctionne sans modifier aucun logiciel.
Des chercheurs ont trouvé le moyen de corrompre du matériel critique à la
sécurité en manipulant la polarité des impuretés dans le
silicium.~\cite{becker2013stealthy} Ils ont alors été capables de compromettre
un générateur cryptographique de nombres aléatoires, juste en modifiant les
propriétés physiques du truc dont sont faites les puces électroniques. Et
puisque cette modification est invisible, la porte dérobée est indétectable à la
vérification optique, l'un des mécanismes de détection de sabotage les plus
utilisés pour ce genre de puces.

\begin{figure}
  \includegraphics{assets/images/stealthy-hardware-trojan.png}
  \caption{Chevaux de Troie matériels furtifs de niveau dopant par Becker,
  Regazzoni, Paar, Burleson}
  \label{fig:stealthy-hardware-trojan}
\end{figure}

Ça vous fait peur ? Eh bien, même si vous étiez capable de tout fabriquer et
programmer à partir de rien, vous devriez quand même faire confiance aux
mathématiques sous-jacentes. Il vous faudrait être convaincu que
\textit{secp256k1} est une courbe elliptique sans porte dérobée. Oui, des portes
dérobées malveillantes peuvent être insérées dans les fondations mathématiques
des fonctions cryptographiques et c'est vraisemblablement déjà arrivé au moins
une fois.~\cite{wiki:Dual_EC_DRBG} Il y a de quoi être paranoïaque pour de
bonnes raisons et le fait que tout, depuis votre matériel, en passant par vos
logiciels, jusqu'aux courbes elliptiques que l'on utilise, peut receler une
porte dérobée~\cite{wiki:backdoors} en fait partie.

\begin{quotation}\begin{samepage}
\enquote{Ne vous fiez pas. Vérifiez.}
\begin{flushright} -- Des bitcoiners, un peu partout
\end{flushright}\end{samepage}\end{quotation}

Les exemples ci-dessus ont pour but d'illustrer que l'informatique \textit{sans
confiance} est utopique. Bitcoin est sans doute le seul système qui effleure
cette utopie et pourtant, il est \textit{à confiance réduite} --- visant à
l'éliminer partout où c'est possible. On peut soutenir que la chaîne de
confiance est infinie, puisque vous devrez également vous convaincre que les
calculs demandent de l'énergie, que P n'est pas égal à NP et que vous vivez bien
dans la réalité et pas dans une simulation orchestrée par des acteurs
malveillants.

Des développeurs travaillent sur des outils et des procédures cherchant à
minimiser encore plus toute confiance restante. Par exemple, les développeurs de
Bitcoin ont créé Gitian\footnote{\url{https://gitian.org/}}, qui est une méthode
de diffusion de logiciels permettant de faire des compilations déterministes.
L'idée, c'est que les chances de manipulation malveillante sont réduites lorsque
plusieurs développeurs parviennent à reproduire des exécutables identiques. Mais
les portes dérobées imaginaires ne sont pas le seul vecteur d'attaque. Un simple
chantage ou de l'extorsion sont des menaces bien réelles. Comme dans le
protocole principal, la décentralisation sert à réduire la confiance nécessaire.

Divers efforts sont déployés afin de résoudre le problème de l'amorçage,
similaire à celui de l'œuf et de la poule, brillamment mis en évidence par le
piratage de Ken Thompson~\cite{web:bootstrapping}.
Guix\footnote{\url{https://guix.gnu.org}} (à prononcer \textit{geeks}), qui
utilise une gestion de paquets fonctionnellement déclarés et permet dès la
conception une compilation reproductible bit à bit, est un de ces efforts. Il en
résulte que vous n'avez plus à faire confiance aux serveurs qui vous fournissent
les logiciels, puisque vous pouvez vérifier par vous-même que l'exécutable
fourni est intact en le recompilant de zéro. Une \textit{pull request} a
récemment été fusionnée afin d'intégrer Guix dans le procédé de compilation de
Bitcoin.\footnote{Voir la PR 15277 de \texttt{bitcoin-core}: \\
\url{https://github.com/bitcoin/bitcoin/pull/15277}}

\begin{figure}
  \includegraphics{assets/images/guix-bootstrap-dependencies.png}
  \caption{Qui était là le premier, l'œuf ou la poule ?}
  \label{fig:guix-bootstrap-dependencies}
\end{figure}

Par chance, Bitcoin ne se repose pas sur un seul algorithme ou une seule sorte
de matériel. Une conséquence de la décentralisation radicale de Bitcoin, c'est
un modèle de sécurité distribué. Bien que les portes dérobées précédemment
décrites doivent être prises au sérieux, il est improbable que chaque
portefeuille logiciel, chaque portefeuille matériel, chaque bibliothèque de
fonctions cryptographiques, chaque implémentation de nœud et chaque compilateur
de chaque langage soient compromis. C'est possible, mais très hautement
improbable.

Notez par ailleurs que vous pouvez très bien générer une clé privée sans même
vous servir d'un quelconque logiciel ou matériel. Vous pouvez tirer un certain
nombre de fois à pile ou face~\cite{antonopoulos2014mastering}, mais selon votre
style de lancer cette source d'aléatoire ne le sera peut-être pas assez. Ce
n'est pas pour rien que des protocoles de stockage comme
Glacier\footnote{\url{https://glacierprotocol.org/}} recommandent d'utiliser un
dé de qualité casino comme une des deux sources d'entropie.

J'ai été forcé par Bitcoin à réfléchir à ce qu'implique réellement de ne se fier
à personne. Il m'a sensibilisé au problème de l'amorçage et de la chaîne de
confiance implicite dans le développement et l'exécution des programmes. Il m'a
fait également prendre conscience des multiples manières de compromettre un
logiciel ou un matériel.

\paragraph{Bitcoin m'a appris à ne pas me fier, mais à vérifier.}

% ---
%
% #### Down the Rabbit Hole
%
% - [The Bitcoin whitepaper][Nakamoto] by Satoshi Nakamoto
% - [Reflections on Trusting Trust][\textit{Reflections on Trusting Trust}] by Ken Thompson
% - [51% Attack][majority] on the Bitcoin Developer Guide
% - [Bootstrapping][bootstrapping], Guix Manual
% - [Secp256k1][secp256k1] on the Bitcoin Wiki
% - [ECC Backdoors][backdoors], [Dual EC DRBG][has already happened] on Wikipedia
%
% [Emmanuel Boutet]: https://commons.wikimedia.org/wiki/User:Emmanuel.boutet
% [\textit{Reflections on Trusting Trust}]: https://www.archive.ece.cmu.edu/~ganger/712.fall02/papers/p761-thompson.pdf
% [found a way]: https://scholar.google.com/scholar?hl=en&as_sdt=0%2C5&q=Stealthy+Dopant-Level+Hardware+Trojans&btnG=
% [Gitian]: https://gitian.org/
% [bootstrapping]: https://www.gnu.org/software/guix/manual/en/html_node/Bootstrapping.html
% [Guix]: https://www.gnu.org/software/guix/
% [pull-request]: https://github.com/bitcoin/bitcoin/pull/15277
% [flip a coin]: https://github.com/bitcoinbook/bitcoinbook/blob/develop/ch04.asciidoc#private-keys
% [Glacier]: https://glacierprotocol.org/
% [secp256k1]: https://en.bitcoin.it/wiki/Secp256k1
% [majority]: https://bitcoin.org/en/developer-guide#term-51-attack
%
% <!-- Wikipedia -->
% [backdoors]: https://en.wikipedia.org/wiki/Elliptic-curve_cryptography#Backdoors
% [has already happened]: https://en.wikipedia.org/wiki/Dual_EC_DRBG
% [Carl Sagan]: https://en.wikipedia.org/wiki/Cosmos_%28Carl_Sagan_book%29
% [alice]: https://en.wikipedia.org/wiki/Alice%27s_Adventures_in_Wonderland
% [carroll]: https://en.wikipedia.org/wiki/Lewis_Carroll

\chapter{Donner l'heure demande du travail}
\label{les:17}

\begin{chapquote}{Lewis Carroll, \textit{Alice au pays des merveilles}}
\enquote{Oh, mon Dieu ! Oh, mon Dieu ! Je vais être en retard !}
\end{chapquote}

On dit souvent qu'on mine des bitcoins parce que des milliers d'ordinateurs
travaillent à résoudre des problèmes mathématiques \textit{très complexes}. Des
problèmes doivent êtres résolus et si vous calculez la bonne réponse, vous
\enquote{produisez} un bitcoin. Bien que cette vision simplifiée du minage de
bitcoin soit plus facile à exprimer, elle passe un peu à côté du sujet. Les
bitcoins ne sont pas produits ou créés et toute la difficulté ne consiste pas
réellement à résoudre certains problèmes de maths. En plus, les maths en
question ne sont pas particulièrement complexes. Ce qui l'est, c'est de
\textit{donner l'heure} dans un système décentralisé.

Comme le démontre le livre blanc, le système de preuve de travail (alias le
minage) est une manière d'implémenter un serveur d'horodatage distribué.

\begin{figure}
  \includegraphics{assets/images/bitcoin-whitepaper-timestamp-wide.png}
  \caption{Extraits du livre blanc. Ai-je entendu \enquote{timechain} ?}
  \label{fig:bitcoin-whitepaper-timestamp-wide}
\end{figure}

Au début, lorsque j'étudiais le fonctionnement de Bitcoin, j'ai moi aussi pensé
que la preuve de travail était inefficace et générait du gaspillage. Au bout
d'un moment, j'ai commencé à changer de point de vue sur la consommation
d'énergie de Bitcoin~\cite{gigi:energy}. Il semblerait qu'aujourd'hui, en l'an
10 ap. B (après Bitcoin), la preuve de travail soit toujours largement
incomprise.

Beaucoup de gens semblent croire que c'est un travail \textit{inutile}, puisque
les problèmes à résoudre par la preuve de travail sont inventés. Si l'on se
focalise uniquement sur les calculs, c'est compréhensible d'arriver à cette
conclusion. Mais Bitcoin n'est pas une question de calculs. C'est une question
de \textit{s'accorder indépendamment sur l'ordre des choses}.

La preuve de travail est un système dans lequel chacun peut valider ce qui s'est
passé et dans quel ordre. Cette validation indépendante est la source du
consensus, un accord unique entre de multiples parties à propos de qui possède
quoi.

Dans un environnement radicalement décentralisé, nous ne possédons pas le luxe
du temps absolu. Toute horloge aurait pour effet d'introduire une tierce partie,
un point central du système qui pourrait être attaqué et sur lequel il faudrait
se reposer. \enquote{La mesure du temps est le problème fondamental}, comme le
souligne Grisha Trubetskoy~\cite{pow-clock}. Et Satoshi a brillamment résolu ce
problème en proposant l'implémentation d'une horloge décentralisée via une
chaîne de blocs par preuve de travail. Chacun accepte préalablement que la
source de vérité provient de la chaîne avec le plus de travail cumulé. C'est
littéralement ce qui s'est passé. Cette entente est dorénavant connue sous le
nom de consensus de Nakamoto.

\begin{quotation}\begin{samepage}
\enquote{Le réseau horodate les transactions en les hachant en une chaîne
continue de preuves de travail [...] (qui) sert de preuve par témoignage de la
séquence des événements}
\begin{flushright} -- Satoshi Nakamoto\footnote{Satoshi Nakamoto, livre blanc de
Bitcoin~\cite{whitepaper}}
\end{flushright}\end{samepage}\end{quotation}

Sans moyen constant pour donner l'heure, il n'existe pas de manière cohérente de
distinguer l'avant de l'après. Un ordre fiable est impossible. Comme nous
l'avons vu, le consensus de Nakamoto est le chemin qu'a pris Bitcoin pour donner
l'heure continuellement. La structure incitative du système produit une horloge
probabiliste et décentralisée en se servant à la fois de la cupidité et de
l'intérêt personnel des participants qui se concurrencent. L'imprécision de
cette horloge n'a aucune importance car à la fin, l'ordre des événements est
indiscutable et chacun peut le vérifier.

Grâce à la preuve de travail, la décentralisation radicale touche à la fois le
travail \textit{et} la vérification du travail. Chacun peut rejoindre et quitter
le réseau à volonté et chacun peut vérifier tout, tout le temps. Non seulement
ça, mais chacun peut vérifier l'état du système \textit{personnellement}, sans
devoir se fier à quiconque.

Ça prend du temps de comprendre la preuve de travail. C'est bien souvent
contre-intuitif et malgré les règles simples, ça donne lieu à des phénomènes
plutôt complexes. Personnellement, me focaliser sur le minage m'a aidé. C'est
utile, pas inutile. La vérification, pas les calculs. C'est du temps, pas des
blocs.

\paragraph{Bitcoin m'a appris que c'était délicat de donner l'heure, surtout
quand on est décentralisé.}

% ---
%
% #### Through the Looking-Glass
%
% - [Bitcoin's Energy Consumption: A shift in perspective][energy]
%
% #### Down the Rabbit Hole
%
% - [Blockchain Proof-of-Work Is a Decentralized Clock][points out] by Gregory Trubetskoy
% - [The Anatomy of Proof-of-Work][pow-anatomy] by Hugo Nguyen
% - [PoW is efficient][pow-efficient] by Dan Held
% - [Mining][bw-mining], [Controlled supply][bw-supply] on the Bitcoin Wiki
%
% [points out]: https://grisha.org/blog/2018/01/23/explaining-proof-of-work/
% [energy]: 
% [whitepaper]: https://bitcoin.org/bitcoin.pdf
%
% [pow-efficient]: https://blog.picks.co/pow-is-efficient-aa3d442754d3
% [pow-anatomy]: https://bitcointechtalk.com/the-anatomy-of-proof-of-work-98c85b6f6667
% [bw-mining]: https://en.bitcoin.it/wiki/Mining
% [bw-supply]: https://en.bitcoin.it/wiki/Controlled_supply
%
% <!-- Wikipedia -->
% [alice]: https://en.wikipedia.org/wiki/Alice%27s_Adventures_in_Wonderland
% [carroll]: https://en.wikipedia.org/wiki/Lewis_Carroll

\chapter{Avancer lentement sans rien casser}
\label{les:18}

\begin{chapquote}{Lewis Carroll, \textit{Les aventures d'Alice sous terre}}
Aussi la barque serpentait-elle doucement, sous le brillant jour d’été, avec son
joyeux équipage et sa musique de voix et d’éclats de rire\ldots
\end{chapquote}

J'enfonce peut-être des portes ouvertes, mais le monde de la tech fonctionne
toujours aujourd'hui en \enquote{avançant vite et en cassant des trucs}. L'idée
de ne pas s'efforcer à réussir du premier coup est une des bases de la mentalité
\textit{échoue tôt, échoue souvent}. Le succès se mesure à la croissance, donc
tant que vous grandissez, tout ira bien. Si quelque chose ne fonctionne pas
comme prévu vous n'avez qu'à pivoter et itérer. En d'autres termes : si vous
jetez assez de merde contre le mur, vous verrez bien ce qui colle.

Bitcoin est très différent. Il l'est dès sa conception. Il est différent par
besoin. Satoshi l'a dit lui-même, les monnaies électroniques ont été tentées à
de nombreuses reprises et tous ces essais ont échoué car il y avait une tête à
faire tomber. L'innovation de Bitcoin, c'est d'être un animal sans tête.

\begin{quotation}\begin{samepage}
\enquote{Beaucoup de gens refusent par réflexe l'idée des monnaies électroniques
à cause de toutes les entreprises qui ont échoué dans les années 90. J'espère
qu'il est clair que c'est la nature du contrôle centralisé de ces systèmes qui a
causé leur perte.}
\begin{flushright} -- Satoshi Nakamoto\footnote{Satoshi Nakamoto, dans une
réponse à Sepp Hasslberger \cite{satoshi-centralized-nature}}
\end{flushright}\end{samepage}\end{quotation}

Une des conséquences de cette décentralisation radicale est la résistance
inhérente au changement. \enquote{Avancer vite en cassant des trucs} ne
fonctionne et ne fonctionnera jamais sur la couche de base de Bitcoin. Même si
on le voulait, ce ne serait pas possible sans convaincre \textit{tout le monde}
de changer sa façon de faire. Voilà ce qu'est le consensus distribué. Voilà la
nature de Bitcoin.

\begin{quotation}\begin{samepage}
\enquote{La nature de Bitcoin est telle qu'une fois sortie la version 0.1, les
concepts essentiels étaient gravés dans le marbre pour le restant de ses jours.}
\begin{flushright} -- Satoshi Nakamoto\footnote{Satoshi Nakamoto, dans une
réponse à Gavin Andresen \cite{satoshi-centralized-nature}}
\end{flushright}\end{samepage}\end{quotation}

C'est l'une des nombreuses propriétés paradoxales de Bitcoin. Nous avons tous
fini par croire que tout logiciel peut facilement être modifié. Mais la nature
de cet animal rend tout changement sacrément difficile.

Comme Hasu l'écrit élégamment dans Décortiquer le contrat social de Bitcoin
\cite{social-contract}, on ne peut changer ses règles qu'en
\textit{proposant} un changement et par conséquent en \textit{convainquant} tous
les utilisateurs de Bitcoin de l'adopter. Bien qu'il soit un logiciel, cela rend
Bitcoin très résilient au changement.

Cette résilience est l'un des attributs de Bitcoin les plus importants. Les
systèmes logiciels critiques se doivent d'être anti-fragile, ce qui est garanti
par l'interaction entre les couches sociales et techniques de Bitcoin. Les
systèmes monétaires sont agressifs par nature et nous le savons depuis des
milliers d'années, un environnement agressif requiert des fondations solides.

\begin{quotation}\begin{samepage}
\enquote{La pluie est tombée, les torrents sont venus, les vents ont soufflé et
se sont jetés contre cette maison : elle n'est point tombée, parce qu'elle était
fondée sur le roc.}
\begin{flushright} -- Matthieu 7:25
\end{flushright}\end{samepage}\end{quotation}

Vraisemblablement, dans cette parabole des bâtisseurs sages et sots, Bitcoin
n'est pas la maison. C'est le roc. Invariable, stable, apportant le socle d'un
nouveau système financier.

À l'image des géologues qui savent que les formations rocheuses évoluent et sont
toujours en mouvement, on peut s'apercevoir que Bitcoin bouge et évolue. Il faut
juste savoir où et comment regarder.

L'introduction du \textit{pay to script hash}\footnote{Les transactions Pay to
Script Hash (P2SH) ont été normalisées dans le BIP16. Elles permettent aux
transactions d'être envoyées à un hash de script (adresse commençant par 3) au
lieu d'un hash d'adresse publique (adresse commençant par
1).~\cite{btcwiki:p2sh}} et de \textit{segregated witness}\footnote{Segregated
Witness (abrégé en SegWit) est une mise à jour implémentée du protocole qui vise
à fournir une protection envers la plasticité des transactions et à élargir la
taille des blocs. SegWit sépare le \textit{témoin} (witness en anglais) de la
liste des entrées.~\cite{btcwiki:segwit}} sont la preuve que les règles de
Bitcoin peuvent être changées tant qu'il y assez d'utilisateurs convaincus que
ce changement bénéficie au réseau. SegWit a permis le développement du réseau
Lightning\footnote{\url{https://lightning.network/}} qui est l'une des maisons
en construction sur le socle solide de Bitcoin. Des mises à jour futures comme
les signatures de Schnorr~\cite{bip:schnorr} amélioreront l'efficacité et la
confidentialité, ainsi que les scripts (comprendre: contrats intelligents) qui
seront indiscernables des transactions classiques grâce à
Taproot~\cite{taproot}. Les sages bâtisseurs construisent bel et bien sur des
fondations solides.

Satoshi n'était pas seulement un sage bâtisseur de technologie. Il comprenait
aussi la nécessité de prendre de sages décisions idéologiques.

\begin{quotation}\begin{samepage}
\enquote{Être open source veut dire que quiconque peut personnellement vérifier
le code. Si les sources étaient privées, personne ne pourrait vérifier la
sécurité. Je pense qu'un programme de cette nature doit obligatoirement être
open source.}
\begin{flushright} -- Satoshi Nakamoto\footnote{Satoshi Nakamoto, dans une
réponse à SmokeTooMuch \cite{satoshi-open-source}}
\end{flushright}\end{samepage}\end{quotation}

L'ouverture est primordiale à la sécurité et inhérente à l'open source et au
mouvement du logiciel libre. Comme Satoshi le faisait remarquer, les protocoles
sécurisés et le code qui les implémente doivent être ouverts --- l'obscurité ne
procure aucune sécurité. Encore une fois, un autre avantage est lié à la
décentralisation : du code qui peut être exécuté, lu, modifié, copié et
distribué librement garantit sa diffusion rapide et étendue.

La nature radicalement décentralisée de Bitcoin est ce qui lui permet de se
déplacer lentement et sciemment. Un réseau de nœuds, détenus par des
individus souverains, est intrinsèquement résistant au changement ---
malveillant ou pas. Sans possibilité de forcer des mises à jour, la seule façon
d'introduire des modifications est de convaincre lentement chacun des
participants de l'adopter. Ce procédé décentralisé de proposer et de déployer
les changements est ce qui rend le réseau extraordinairement résilient face aux
modifications malveillantes. C'est aussi ce qui rend les réparations plus
complexes que dans un environnement centralisé et qui explique que tout le monde
cherche avant tout à ne rien casser.

\paragraph{Bitcoin m'a appris qu'avancer lentement était une fonctionnalité, pas
un bug.}

% ---
%
% #### Through the Looking-Glass
%
% - [Lesson 1: Immutability and Change][lesson1]
%
% #### Down the Rabbit Hole
%
% - [Unpacking Bitcoin's Social Contract] by Hasu
% - [Schnorr signatures BIP][Schnorr signatures] by Pieter Wuille
% - [Taproot proposal][Taproot] by Gregory Maxwell
% - [P2SH][pay to script hash], [SegWit][segregated witness] on the Bitcoin Wiki
% - [Parable of the Wise and the Foolish Builders][Matthew 7:24--27] on Wikipedia
%
% <!-- Down the Rabbit Hole -->
% [lesson1]: {{ '/bitcoin/lessons/ch1-01-immutability-and-change' | absolute_url }}
%
% [Unpacking Bitcoin's Social Contract]: https://uncommoncore.co/unpacking-bitcoins-social-contract/
% [Matthew 7:24--27]: https://en.wikipedia.org/wiki/Parable_of_the_Wise_and_the_Foolish_Builders
% [pay to script hash]: https://en.bitcoin.it/wiki/Pay_to_script_hash
% [segregated witness]: https://en.bitcoin.it/wiki/Segregated_Witness
% [lightning network]: https://lightning.network/
% [Schnorr signatures]: https://github.com/sipa/bips/blob/bip-schnorr/bip-schnorr.mediawiki#cite_ref-6-0
% [Taproot]: https://lists.linuxfoundation.org/pipermail/bitcoin-dev/2018-January/015614.html
%
% <!-- Wikipedia -->
% [alice]: https://en.wikipedia.org/wiki/Alice%27s_Adventures_in_Wonderland
% [carroll]: https://en.wikipedia.org/wiki/Lewis_Carroll

\chapter{La vie privée n'est pas morte}
\label{les:19}

\begin{chapquote}{Lewis Carroll, \textit{Alice in Wonderland}}
Les joueurs jouaient tous en même temps sans attendre leur tour ; ils se
disputaient sans arrêt et s’arrachaient les hérissons. Au bout d’un instant, la
Reine, entrant dans une furieuse colère, parcourut le terrain en tapant du pied
et en criant : \enquote{Qu’on lui coupe la tête ! Qu’on lui coupe la tête !} à
peu près une fois par minute.
\end{chapquote}

À en croire les experts, la vie privée est morte depuis les années
80\footnote{\url{https://bit.ly/privacy-is-dead}}. L'invention pseudonyme de
Bitcoin et d'autres événements de l'histoire récente montrent que ce n'est pas
vrai. La vie privée est vivante, bien qu'il ne soit pas facile d'échapper à cet
état de surveillance.

Satoshi a pris d'infinies précautions afin de se couvrir et de dissimuler son
identité. Dix ans plus tard, personne ne peut dire si Satoshi Nakamoto était une
personne ou un groupe, un homme, une femme ou un intelligence artificielle venue
du futur pour s'auto-amorcer afin de régner sur le monde. Théories complotistes
mises à part, Satoshi a choisi de s'identifier à un homme japonais, c'est
pourquoi je ne présume de rien en respectant son choix de genre et en le
désignant par \textit{il}.

\begin{figure}
  \includegraphics{assets/images/nope.png}
  \caption{Je ne suis pas Dorian Nakamoto.}
  \label{fig:nope}
\end{figure}

Quelle que soit sa vraie identité, Satoshi a réussi à la cacher. Il a créé un
précédent encourageant pour tous ceux qui désirent rester anonymes : c'est
possible d'avoir une vie privée en ligne.

\begin{quotation}\begin{samepage}
\enquote{Le chiffrement fonctionne. Les systèmes crypto robustes et correctement
implémentés sont l'une des rares choses sur lesquelles vous pouvez compter.}
\begin{flushright} -- Edward Snowden\footnote{Edward Snowden, réponses au
courrier des lecteurs \cite{snowden}}
\end{flushright}\end{samepage}\end{quotation}

Satoshi n'est pas le premier inventeur pseudonyme ou anonyme et ne sera pas le
dernier. Certains ont directement repris son style de publication pseudonyme,
comme Tom Elvis Jedusor, connu pour MimbleWimble~\cite{mimblewimble-origin},
quand d'autres ont publié des preuves mathématiques avancées en restant
totalement anonymes~\cite{4chan-math}.

C'est un nouveau monde étrange que nous habitons. Un monde où l'identité est
facultative, où les contributions sont acceptées sur la base du mérite et où les
gens peuvent collaborer et négocier librement. Il faudra quelques ajustements
pour être à l'aise avec ces nouveaux paradigmes, mais je crois fermement que
tout ceci a le potentiel pour rendre le monde meilleur.

Chacun d'entre nous devrait se souvenir que la vie privée est un droit humain
fondamental\footnote{Déclaration universelle des Droits de l'Homme,
\textit{Article 12}.~\cite{article12}}. Tant que le peuple exercera et défendra
ces droits, la bataille pour la vie privée sera loin d'être achevée.

\paragraph{Bitcoin m'a appris que la vie privée n'était pas morte.}

% ---
%
% #### Down the Rabbit Hole
%
% - [Universal Declaration of Human Rights][fundamental human right] by the United Nations
% - [A lower bound on the length of the shortest superpattern][anonymous] by Anonymous 4chan Poster, Robin Houston, Jay Pantone, and Vince Vatter
%
% [since the 80ies]: https://books.google.com/ngrams/graph?content=privacy+is+dead&year_start=1970&year_end=2019&corpus=15&smoothing=3&share=&direct_url=t1%3B%2Cprivacy%20is%20dead%3B%2Cc0
% [time-traveling AI]: https://blockchain24-7.com/is-crypto-creator-a-time-travelling-ai/
% ["I am not Dorian Nakamoto."]: http://p2pfoundation.ning.com/forum/topics/bitcoin-open-source?commentId=2003008%3AComment%3A52186
% [MimbleWimble]: https://github.com/mimblewimble/docs/wiki/MimbleWimble-Origin
% [anonymous]: https://oeis.org/A180632/a180632.pdf
% [fundamental human right]: http://www.un.org/en/universal-declaration-human-rights/
%
% <!-- Wikipedia -->
% [alice]: https://en.wikipedia.org/wiki/Alice%27s_Adventures_in_Wonderland
% [carroll]: https://en.wikipedia.org/wiki/Lewis_Carroll

\chapter{Les cypherpunks écrivent du code}
\label{les:20}

\begin{chapquote}{Lewis Carroll, \textit{Alice au pays des merveilles}}
\enquote{Je vois bien que vous essayez d’inventer quelque chose !}
\end{chapquote}

Comme beaucoup de grandes idées, Bitcoin n'est pas sorti de nulle part. Il a vu
le jour en utilisant et en combinant beaucoup d'innovations et de découvertes en
mathématiques, en physique, en informatique et dans d'autres domaines. Satoshi
est sans conteste un génie mais il n'aurait pas pu inventer Bitcoin sans se
tenir sur des épaules de géants.

\begin{quotation}\begin{samepage}

\enquote{Celui qui simplement souhaite et espère n'intervient pas activement
dans le cours des événements ni dans le profil de sa destinée.}
\begin{flushright} -- Ludwig von Mises\footnote{Ludwig von Mises,
\textit{L’Action Humaine} \cite{human-action}}
\end{flushright}\end{samepage}\end{quotation}
% > <cite>[Ludwig Von Mises]</cite>

L'un de ces géants, c'est Eric Hughes, un des fondateurs du mouvement cypherpunk
et auteur du \textit{Manifeste d'un Cypherpunk}. Ça paraît difficile d'imaginer
que Satoshi n'ait pas été influencé par ce manifeste. Il parle de tellement de
choses que Bitcoin permet et utilise, telles que les transactions privées
directes, l'argent électronique et liquide, les systèmes anonymes et la
protection de la vie privée par la cryptographie et les signatures numériques.

\begin{quotation}\begin{samepage}
\enquote{La confidentialité est nécessaire pour une société ouverte à l'ère
électronique. [...] Puisque nous désirons la confidentialité, nous devons nous
assurer que chaque partie à une transaction n'a connaissance que de ce qui est
directement nécessaire à cette transaction. [...]
Par conséquent, la vie privée dans une société ouverte nécessite des systèmes de
transaction anonymes. Jusqu'à présent, l'argent liquide a été le principal
système de ce type. Un système de transaction anonyme n'est pas un système de
transaction secret. [...]
Nous les Cypherpunks sommes dédiés à la construction de systèmes anonymes. Nous
défendons notre vie privée avec la cryptographie, avec des systèmes de transfert
de courrier anonyme, avec des signatures numériques et avec de la monnaie
électronique.
Les Cypherpunks écrivent du code.}
\begin{flushright} -- Eric Hughes\footnote{Eric Hughes, Manifeste d'un
Cypherpunk \cite{cypherpunk-manifesto}}
\end{flushright}\end{samepage}\end{quotation}

Les Cypherpunks ne trouvent pas de réconfort dans les espoirs et les vœux. Ils
s'immiscent activement dans le cours des événements et forgent leur propre
destinée. Les Cypherpunks écrivent du code.

Et donc, en fidèle cypherpunk, Satoshi s'est assis et s'est mis à coder. Du code
parti d'une idée abstraite pour prouver au monde qu'elle pouvait marcher. Du
code semant la graine d'une nouvelle réalité économique. Grâce au code, chacun
peut vérifier que ce système novateur fonctionne vraiment et qu'à peu près
toutes les 10 minutes, Bitcoin prouve au monde qu'il est encore vivant.

\begin{figure}
  \includegraphics{assets/images/bitcoin-code-white.png}
  \caption{Extraits du code de la version 0.1 de Bitcoin}
  \label{fig:bitcoin-code-white}
\end{figure}

Afin de s'assurer que son invention ne resterait pas du domaine du rêve, Satoshi
a écrit le code de son idée avant d'écrire le livre blanc. Il a aussi pris soin
de ne pas retarder\footnote{\enquote{Nous ne devrions pas reporter indéfiniment
tant que chaque fonction n'est pas terminée.} -- Satoshi
Nakamoto~\cite{satoshi-delay}} chaque version indéfiniment. Après tout,
\enquote{il y aura toujours autre chose à faire}.

\begin{quotation}\begin{samepage}
\enquote{J'ai dû écrire tout le code avant d'être convaincu que je pouvais
résoudre chaque problème, puis j'ai écrit le livre blanc.}
\begin{flushright} -- Satoshi Nakamoto\footnote{Satoshi Nakamoto, dans Re:
Bitcoin P2P e-cash paper \cite{satoshi-mail-code-first}}
\end{flushright}\end{samepage}\end{quotation}

Dans ce monde aux promesses infinies et au déroulement douteux, la mise en
pratique d'un développement dévoué manquait cruellement. Soyez volontaires,
persuadez-vous d'être capable de résoudre les problèmes et implémentez les
solutions. On devrait tous essayer d'être un peu plus cypherpunk.

\paragraph{Bitcoin m'a appris que les cypherpunks écrivaient du code.}

% ---
%
% #### Down the Rabbit Hole
%
% - [Bitcoin version 0.1.0 announcement][version 0.1.0] by Satoshi Nakamoto
% - [Bitcoin P2P e-cash paper announcement][mail-announcement] by Satoshi Nakamoto
%
% [mail-announcement]: http://www.metzdowd.com/pipermail/cryptography/2008-October/014810.html
% [Ludwig Von Mises]: https://mises.org/library/human-action-0/html/pp/613
% [version 0.1.0]: https://bitcointalk.org/index.php?topic=68121.0
% [not to delay]: https://bitcointalk.org/index.php?topic=199.msg1670#msg1670
% [6]: http://www.metzdowd.com/pipermail/cryptography/2008-November/014832.html
%
% <!-- Wikipedia -->
% [alice]: https://en.wikipedia.org/wiki/Alice%27s_Adventures_in_Wonderland
% [carroll]: https://en.wikipedia.org/wiki/Lewis_Carroll

\chapter{Métaphores pour le futur de Bitcoin}
\label{les:21}

\begin{chapquote}{Lewis Carroll, \textit{Alice au pays des merveilles}}
\enquote{Je sais qu’il arrive toujours quelque chose d’intéressant\ldots}
\end{chapquote}

Au cours des deux dernières décennies, il est devenu évident que l'innovation
technologique ne suivait pas une courbe linéaire. Que vous croyiez ou pas à la
singularité technologique, le progrès est indéniablement exponentiel dans de
nombreux domaines. En plus de ça, le taux d'adoption des technologies s'accélère
et sans vous en rendre compte, le buisson de la cour d'école du coin a disparu
car vos enfants utilisent Snapchat à la place. Les courbes exponentielles ont
cette tendance à vous exploser au visage bien avant que vous ne l'ayez vu venir.

Bitcoin est une technologie exponentielle reposant sur d'autres technologies
exponentielles. \textit{Our World in
Data}\footnote{\url{https://ourworldindata.org/}} montre admirablement la
vitesse croissante de l'adoption technologique à partir de 1903 avec l'arrivée
des lignes téléphoniques (voir la Figure\ref{fig:tech-adoption}). Les lignes
téléphoniques, l'électricité, les ordinateurs, Internet, les smartphones ; tous
observent des tendances exponentielles en termes de qualité-prix et d'adoption.
C'est pareil pour Bitcoin~\cite{tech-adoption}.

\begin{figure}
  \includegraphics{assets/images/tech-adoption.png}
  \caption{Bitcoin est littéralement hors-normes.}
  \label{fig:tech-adoption}
\end{figure}

Bitcoin possède de multiples effets de réseau\footnote{Trace Mayer, \textit{Les
sept effets de réseau de Bitcoin}~\cite{7-network-effects}} découlant tous de
configurations de croissance exponentielle dans leurs propres domaines : le
prix, les usagers, la sécurité, les développeurs, la part de marché et
l'adoption comme monnaie globale.

En ayant survécu à ses premiers pas, Bitcoin continue de grandir chaque jour
dans plus d'un aspect. D'accord, sa technologie n'est pas encore mature. Il est
sans doute en pleine adolescence. Mais si la technologie est exponentielle,
passer de l'ombre à l'omniprésence sera un court chemin.

\begin{figure}
  \includegraphics{assets/images/mobile-phone.png}
  \caption{Le téléphone mobile, env. 1965 contre 2019.}
  \label{fig:mobile-phone}
\end{figure}

Dans sa conférence TED de 2003, Jeff Bezos a choisi l'électricité comme
métaphore au futur du web\footnote{\url{http://bit.ly/bezos-web}}. Ces trois
phénomènes --- l'électricité, Internet, Bitcoin --- sont des technologies
\textit{habilitantes}, des réseaux qui facilitent autre chose. Ce sont des
infrastructures fondamentales, qui permettent de bâtir.

Ça fait un moment que nous côtoyons l'électricité. On la prend pour acquise.
Internet est un peu plus jeune mais la plupart des gens le prennent aussi pour
acquis. Bitcoin a dix ans et n'a pénétré les consciences que durant le dernier
cycle d'engouement. Seuls les pionniers le prennent pour acquis. Plus le temps
passera, plus les gens verront Bitcoin comme une chose banale\footnote{Ceci est
connu sous le nom d'\textit{effet Lindy}. L'effet Lindy est la théorie selon
laquelle l'espérance de vie future d'une chose non périssable est
proportionnelle à son âge actuel, impliquant une espérance de vie restante plus
longue à chaque fois qu'elle survit à une période de temps.~\cite{wiki:lindy}}. 

En 1994, Internet était encore déroutant et contre-intuitif. Il suffit de
regarder ce vieil enregistrement du \textit{Today
Show}\footnote{\url{https://youtu.be/UlJku_CSyNg}} pour voir qu'à l'évidence, ce
qui nous paraît naturel et intuitif aujourd'hui ne l'était en fait pas à
l'époque. Pour la plupart, Bitcoin reste encore déroutant et étrange, mais tout
comme Internet est une seconde nature pour les natifs du numérique, dépenser et
accumuler des sats\footnote{\url{https://twitter.com/hashtag/stackingsats}} le
sera pour les natifs du Bitcoin dans le futur.

\begin{quotation}\begin{samepage}
\enquote{Le futur est déjà là --- il n’est simplement pas réparti
équitablement.}
\begin{flushright} -- William Gibson\footnote{William Gibson, \textit{La science
dans la science-fiction} \cite{william-gibson}}
\end{flushright}\end{samepage}\end{quotation}

En 1995, environ $15\%$ des adultes américains utilisaient Internet. Les données
historiques du centre de recherches Pew~\cite{pew-research} montrent à quel
point Internet s'est immiscé dans nos vies. Selon un sondage client de Kaspersky
Lab~\cite{web:kaspersky}, $13\%$ des personnes interrogées se sont servi de
Bitcoin ou de ses clones pour payer un bien en 2018. Bien que les paiements ne
soient pas l'unique cas d'usage de Bitcoin, cela donne une idée d'où nous en
sommes en temps Internet : dans la première moitié des années 90.

En 1997, Jeff Bezos écrivait à ses actionnaires~\cite{bezos-letter}
\enquote{c'est le premier jour d'Internet}, pressentant son gigantesque
potentiel inexploité et, par extension, celui de son entreprise. Peu importe à
quel jour se trouve Bitcoin, seul l'observateur inattentif ne voit pas
clairement les immenses volumes de potentiel inexploité.

\begin{figure}
  \includegraphics{assets/images/internet-evolution-white-dates.png}
  \caption{Internet, 1982 vs. 2005. Source : CC-BY Merit Network, Inc. et
  Barrett Lyon, Opte Project}
  \label{fig:internet-evolution-white-dates}
\end{figure}

Le premier nœud Bitcoin fut mis en ligne en 2009 après que Satoshi mina le
\textit{bloc de genèse}\footnote{Le bloc de genèse est le premier bloc de la
chaîne de blocs Bitcoin. Les versions modernes de Bitcoin le numérotent $0$,
alors que les anciennes versions le comptent comme le bloc $1$. Le bloc de
genèse est habituellement codé en dur dans les applications qui se servent de la
chaîne de blocs de Bitcoin. C'est un cas particulier car il ne référence pas de
bloc précédent et produit une récompense non-dépensable. Le paramètre
\textit{coinbase} contient, entres autres données normales, le texte suivant :
\textit{\enquote{The Times 03/Jan/2009 Chancellor on brink of second bailout for
banks}} \cite{btcwiki:genesis-block}} et libéra le logiciel dans la nature. Son
nœud ne fut pas seul très longtemps. Hal Finney fut l'un des premiers à
accrocher à l'idée et à rejoindre le réseau. Dix ans plus tard, au moment où
j'écris ceci, il y a plus de $75 000$ nœuds qui exécutent bitcoin.

\begin{figure}
  \centering
  \includegraphics[width=8cm]{assets/images/running-bitcoin.png}
  \caption{Hal Finney est l'auteur du premier tweet à mentionner bitcoin en
  janvier 2009.}
  \label{fig:running-bitcoin}
\end{figure}

La couche de base du protocole n'est pas la seule à croître de façon
exponentielle. Le réseau Lightning, une technologie de seconde couche, grandit
encore plus vite.

En janvier 2018, le réseau Lightning était composé de $40$ nœuds et $60$
canaux~\cite{web:lightning-nodes}. En avril 2019, le réseau s'était étendu à
plus de $4000$ nœuds et environ $40 000$ canaux. N'oubliez pas que ça reste une
technologie expérimentale où la perte de fonds peut arriver et arrive parfois.
Malgré ça, la tendance est limpide : des milliers de personnes téméraires sont
enthousiastes à l'idée de s'en servir. 

\begin{figure}
  \includegraphics{assets/images/lnd-growth-lopp-white.png}
  \caption{Le réseau Lightning, janvier 2018 vs. décembre 2018. Source : Jameson
  Lopp}
  \label{fig:lnd-growth-lopp-white.png}
\end{figure}

À mon sens, ayant vécu l'essor météorique du web, les analogies entre Internet
et Bitcoin sont évidentes. Ce sont tous deux des réseaux, des technologies
exponentielles et tous deux amènent de nouvelles possibilités, de nouvelles
industries, de nouveaux comportements. L'électricité est la meilleure métaphore
pour comprendre la direction que prend Internet, il est donc possible
qu'Internet soit la meilleure métaphore pour comprendre la direction que prend
Bitcoin. Ou alors, pour reprendre les mots d'Andreas Antonopoulos, Bitcoin est
l'\textit{Internet de l'argent}. Ces métaphores sont un très bon rappel d'une
Histoire qui ne se répète pas mais qui rime souvent.

Les technologies exponentielles sont difficiles à appréhender et sont souvent
sous-estimées. Même si je m'intéresse beaucoup à celles-ci, je suis sans cesse
surpris de l'allure du progrès et de l'innovation. Observer la croissance de
l'écosystème Bitcoin, ça ressemble à observer l'essor d'Internet, mais en
accéléré. C'est grisant.

Ma quête de sens envers Bitcoin m'a mené plus d'une fois sur les chemins de
l'Histoire. La compréhension des anciennes structures sociétales, des monnaies
du passé et de comment les réseaux de communication ont évolué ont toutes fait
partie du voyage. Du biface au smartphone, la technologie a sans nul doute
changé notre monde à de nombreuses reprises. Les technologies de réseaux ont un
caractère particulier de transformation : l'écriture, les routes, l'électricité,
Internet. Toutes ont changé le monde. Bitcoin a changé le mien et continuera de
transformer les esprits et les cœurs de ceux qui osent l'approcher.

\paragraph{Bitcoin m'a appris que la compréhension du passé était nécessaire à
la compréhension de son futur. Un futur qui ne fait que commencer\ldots}

% ---
%
% #### Down the Rabbit Hole
%
% - [The Rising Speed of Technological Adoption][the rising speed of technological adoption] by Jeff Desjardins
% - [The 7 Network Effects of Bitcoin][multiple network effects] by Trace Mayer
% - [The Electricity Metaphor for the Web's Future][TED talk] by Jeff Bezos
% - [How the internet has woven itself into American life][data from the Pew Research Center] by Susannah Fox and Lee Rainie
% - [Genesis Block][genesis block] on the Bitcoin Wiki
% - [Lindy Effect][more time] on Wikipedia
%
% [Our World in Data]: https://ourworldindata.org/
% [the rising speed of technological adoption]: https://www.visualcapitalist.com/rising-speed-technological-adoption/
% [multiple network effects]: https://www.thrivenotes.com/the-7-network-effects-of-bitcoin/
% [TED talk]: https://www.ted.com/talks/jeff_bezos_on_the_next_web_innovation
% [recording of the Today Show]: https://www.youtube.com/watch?v=UlJku_CSyNg
% [William Gibson]: https://www.npr.org/2018/10/22/1067220/the-science-in-science-fiction
% [data from the Pew Research Center]: https://www.pewinternet.org/2014/02/27/part-1-how-the-internet-has-woven-itself-into-american-life/
% [consumer survey]: https://www.kaspersky.com/blog/money-report-2018/
% [letter to shareholders]: http://media.corporate-ir.net/media_files/irol/97/97664/reports/Shareholderletter97.pdf
% [running bitcoin]: https://twitter.com/halfin/status/1110302988?lang=en
% [40 nodes]: https://bitcoinist.com/bitcoin-lightning-network-mainnet-nodes/
% [reckless]: https://twitter.com/hashtag/reckless
% [Jameson Lopp]: https://twitter.com/lopp/status/1077200836072296449
% [\textit{The Internet of Money}]: https://theinternetofmoney.info/
% [stacking]: https://twitter.com/hashtag/stackingsats
%
% <!-- Bitcoin Wiki -->
% [genesis block]: https://en.bitcoin.it/wiki/Genesis_block
%
% <!-- Wikipedia -->
% [more time]: https://en.wikipedia.org/wiki/Lindy_effect
% [alice]: https://en.wikipedia.org/wiki/Alice%27s_Adventures_in_Wonderland
% [carroll]: https://en.wikipedia.org/wiki/Lewis_Carroll

\addpart{Considérations finales}
\pdfbookmark{Conclusion}{conclusion}
\label{ch:conclusion}

\chapter*{Conclusion}

\begin{chapquote}{Lewis Carroll, \textit{Alice au pays des merveilles}}
\enquote{Commencez au commencement}  dit le Roi d’un ton grave, \enquote{et
continuez jusqu’à ce que vous arriviez à la fin ; ensuite, arrêtez-vous.}
\end{chapquote}

Comme je le disais en introduction, je pense que la réponse à la question
\textit{Qu'avez-vous appris de Bitcoin ?} sera toujours incomplète. La symbiose
de ce qu'on pourrait appeler de multiples systèmes vivants -- Bitcoin, la
technosphère et l'économie -- est trop entremêlée, les sujets sont trop nombreux
et les choses avancent trop vite pour être entièrement comprises par une seule
personne.

Même sans le comprendre dans son entièreté et même avec toutes ses excentricités
et ses défauts apparents, Bitcoin fonctionne indéniablement. Il continue à
produire des blocs à peu près toutes les dix minutes d'une manière admirable.
Plus Bitcoin continuera de fonctionner, plus les gens seront enclins à
l'utiliser.

\begin{quotation}\begin{samepage}
\enquote{C'est vrai que les choses sont belles lorsqu'elles fonctionnent. L'art
est fonctionnel.}
\begin{flushright} -- Giannina Braschi\footnote{Giannina Braschi,
\textit{L'Empire des rêves} \cite{braschi2011empire}}
\end{flushright}\end{samepage}\end{quotation}

\paragraph{} Bitcoin est un enfant d'Internet. Sa croissance est exponentielle,
gommant les frontières entre les disciplines. Par exemple, l'endroit où finit le
royaume purement technologique et où en commence un autre n'est pas clairement
établi. Bitcoin a effectivement besoin d'ordinateurs pour fonctionner mais
l'informatique ne suffit pas à le comprendre. Non seulement, par ses rouages,
Bitcoin ne connaît pas de frontières, mais il n'est pas non plus cloisonné
académiquement.

L'économie, la politique, la théorie des jeux, l'histoire monétaire, la théorie
des réseaux, la finance, la cryptographie, la théorie de l'information, la
censure, la loi et les régulations, les organisations humaines, la psychologie
-- tout ceci avec bien d'autres champs d'expertise qui pourraient faire avancer
la quête de la connaissance de Bitcoin et de son fonctionnement.

Aucune invention n'est seule responsable de son succès. C'est la combinaison
de multiples éléments auparavant sans lien qui, collés ensemble par l'incitation
de la théorie des jeux, constituent la révolution qu'est Bitcoin. Ce qui fait de
Satoshi un génie c'est le sublime mariage de nombreuses disciplines. 

\paragraph{} Comme tout système complexe, Bitcoin doit faire des compromis en
termes d'efficacité, de coût, de sécurité et de bien d'autres aspects. Tout
comme il n'y a pas de solution parfaite à la quadrature du cercle, toute
solution aux problèmes qu'entend résoudre Bitcoin sera toujours imparfaite.

\begin{quotation}\begin{samepage}
\enquote{Je ne crois pas que nous pourrons avoir à nouveau une bonne monnaie
avant que nous ne la retirions des mains du gouvernement, je veux dire, nous ne
pouvons pas la retirer violemment de leur contrôle, tout ce que nous pouvons
faire sera de manière rusée et détournée, introduire quelque chose qu'ils ne
pourront arrêter.}
\begin{flushright} -- Friedrich Hayek\footnote{Friedrich Hayek sur la politique
monétaire, l'étalon-or, les déficits, l'inflation et John Maynard Keynes
\url{https://youtu.be/EYhEDxFwFRU}}
\end{flushright}\end{samepage}\end{quotation}

Bitcoin est la manière rusée et détournée de réintroduire une bonne monnaie dans
notre monde. Il le fait en plaçant un individu souverain derrière chaque nœud,
tout comme De Vinci tentait de résoudre l'inextricable quadrature du cercle en
plaçant l'Homme de Vitruve en son centre. Les nœuds retirent en effet tout
concept de centre, créant un système incroyablement anti-fragile et extrêmement
difficile à stopper. Bitcoin est vivant et cœur battra probablement plus
longtemps que les nôtres.

J'espère que vous avez apprécié ces vingt et une leçons. La leçon la plus
importante est peut-être que Bitcoin mérite une approche holistique, sous
plusieurs angles, si l'on souhaite en dresser un tableau le moins inachevé
possible. De la même façon que l'on détruit un système complexe en retirant un
de ses éléments, examiner individuellement chaque partie de Bitcoin semble en
gâcher la compréhension. Si une seule personne efface \enquote{blockchain} de
son vocabulaire et le remplace par \enquote{une chaîne de blocs}, je pourrai
mourir tranquille.

Quoi qu'il en soit, mon voyage n'est pas terminé. J'ai l'intention de 
m'aventurer encore plus profondément dans le terrier du lapin et je vous invite
à me suivre sur le chemin\footnote{\url{https://twitter.com/dergigi}}.

% <!-- Twitter -->
% [dergigi]: https://twitter.com/dergigi
%
% <!-- Internal -->
% [sly roundabout way]: https://youtu.be/EYhEDxFwFRU?t=1124
% [Giannina Braschi]: https://en.wikipedia.org/wiki/Braschi%27s_Empire_of_Dreams


\cleardoublepage

\chapter*{Remerciements}
\pdfbookmark{Remerciements}{acknowledgments}

Je remercie les innombrables auteurs et créateurs de contenu qui ont
influencé ma pensée sur Bitcoin et les sujets qu'il touche. Il y en a beacoup
trop pour tous les citer, mais je vais faire de mon mieux pour les nommer.

\begin{itemize}
  \item Merci à Arjun Balaji pour le tweet qui m'a motivé à écrire ceci.
  \item Merci à Marty Bent d'avoir sans cesse fourni des éléments de réflexion
  et de l'amusement. Si vous ne suivez pas Marty Bent et ses Tales From The
  Crypt, vous ratez quelque chose. Bravo Matt et Marty de nous guider dans le
  terrier du lapin.
  \item Merci à Michael Goldstein et Pierre Rochard d'extraire et de fournir la
  meilleure littérature sur Bitcoin grâce au Nakamoto Institute. Et merci
  d'avoir créé le podcast Noded qui a grandement influencé ma vision
  philosophique de Bitcoin.
  \item Merci à Saifedean Ammous pour ses convictions, ses tweets sauvages et
  pour avoir écrit L'Étalon Bitcoin
  \item Merci à Francis Pouliot de partager son enthousiasme et d'avoir
  découvert les mentions de la timechain.
  \item Merci à Andreas M. Antonopoulos pour tout le contenu éducatif qu'il a
  créé année après année.
  \item Merci à Peter McCormack pour ses tweets honnêtes et le podcast What
  Bitcoin Did, qui continue encore à donner de bonnes perspectives des nombreux
  secteurs de l'écosystème.
  \item Merci à Jannik, Brandon, Matt, Camilo, Daniel, Michael, et Raphael
  d'avoir donné votre retour sur les ébauches de certaines leçons. Un merci tout
  particulier à Jannik qui a relu de multiples ébauches à de nombreuses
  reprises.
  \item Merci à Dhruv Bansal et Matt Odell d'avoir pris le temps de discuter de
  certaines de ces idées avec moi.
  \item Merci à Guy Swann d'avoir enregistré une version audio de 21lessons.com.
  \item Merci au Frère Hass pour son soutien moral et ses conseils et pour avoir
  pris le temps d'écrire l'avant-propos de ce livre.
  \item Merci à ma femme de m'avoir enduré, moi et ma nature obsessionnelle.
  \item Merci à ma famille de me supporter à la fois pendant les bons et les
  mauvais moments.
  \item Enfin, mais non des moindres, merci à tous les maximalistes Bitcoin, à
  tous les minimalistes des shitcoins, les porte-paroles, les bots et les
  shitposters qui vivent dans ce jardin magnifique qu'est Bitcoin Twitter.
\end{itemize}

Enfin, merci à vous d'avoir lu ceci. J'espère que ça vous a plu autant que j'ai
aimé l'écrire.


\listoffigures

\chapter*{À propos de la bibliographie}
\pdfbookmark{Bibliography}{bibliography}

Aujourd'hui, beaucoup de livres ont été publiés sur Bitcoin. Pourtant, la
plupart des conversations -- et donc la plupart des contenus dignes d'intérêt --
se passent en ligne.

\paragraph{}
La bibliographie qui va suivre établit indifféremment une liste de livres,
d'articles et de contenus en ligne. Si le contenu possède une URL associée, l'URL
était accessible en Octobre 2019, puisque j'ai pu m'y rendre. Si l'une ou
l'autre de ces URL venait à mener à une impasse, j'en suis désolé. Merci de me
le faire savoir\footnote{\url{https://dergigi.com/contact}} afin que je puisse
la mettre à jour.

\paragraph{}
P.S. : Bitcoin et \href{https://ipfs.io/}{IPFS} règlent ça.

\bibliography{main}

\end{document}
