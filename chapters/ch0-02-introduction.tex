\chapter*{Introduction}
\label{ch:introduction}

\begin{chapquote}{Lewis Carroll, \textit{Alice au pays des merveilles}}
\enquote{Mais je ne veux pas aller parmi les fous,} fit remarquer Alice.
\enquote{Impossible de faire autrement,} dit le Chat ; \enquote{nous sommes tous
fous ici. Je suis fou. Tu es folle.} \enquote{Comment savez-vous que je suis
folle ?} demanda Alice. \enquote{Tu dois l’être,} répondit le Chat,
\enquote{autrement tu ne serais pas venue ici.}
\end{chapquote}

En octobre 2018, Arjun Balaji posait cette question innocente :
\textit{Qu'avez-vous appris de Bitcoin ?} Après avoir essayé d'y répondre en un
court tweet, et avoir lamentablement échoué, j'ai compris que ce que j'avais
retenu était bien trop riche pour répondre en quelques mots, voire même répondre
tout court.

Ces connaissances que j'ai acquises, évidemment, concernent Bitcoin - ou tout du
moins lui sont liées. Cependant, bien que certains rouages de Bitcoin soient
expliqués ici, les leçons qui suivent ne sont pas une justification du
fonctionnement ou de la nature de Bitcoin. Elles pourront néanmoins aider à
explorer certains aspects satellites de Bitcoin comme les questions
philosophiques, les réalités économiques ou les innovations technologiques.

\begin{center}
  \includegraphics[width=7cm]{assets/images/the-tweet.png}
\end{center}

Les \textit{21 leçons} sont groupées par sept, formant ainsi trois chapitres.
Chacun de ces chapitres observe Bitcoin sous une lumière précise, récoltant les
enseignements à tirer de l'examen sous différents angles de cet étrange réseau.

\paragraph{Le \hyperref[ch:philosophy]{chapitre 1}}{explore les enseignements
philosophiques de Bitcoin. L'interaction entre immutabilité et changement, le
concept de rareté véritable, l'Immaculée Conception de Bitcoin, le problème de
l'identité, la contradiction entre réplication et localité, la force de la
liberté d'expression et les limites du savoir.}

\paragraph{Le \hyperref[ch:economics]{chapitre 2}}{s'intéresse aux enseignements
économiques de Bitcoin. Des leçons sur la méconnaissance financière,
l'inflation, la valeur, l'argent, son histoire, les réserves fractionnaires
des banques ainsi que la manière dont Bitcoin réintroduit la monnaie saine d'une
façon rusée et détournée.}

\paragraph{Le \hyperref[ch:technology]{chapitre 3}}{présente certaines leçons
acquises en examinant la technologie de Bitcoin. Pourquoi les nombres
renferment-ils une force, des remarques sur la confiance, pourquoi donner
l'heure demande du travail, comment une progression lente et prudente est une
fonctionnalité et pas un bug, ce que l'invention de Bitcoin peut nous apprendre
sur la vie privée, pourquoi les cypherpunks écrivent-ils du code (et non des
lois) et quelles métaphores pourraient être utiles pour imaginer l'avenir de
Bitcoin.}

~

Chaque leçon contient plusieurs citations et liens au fil du texte. Si une idée
vaut la peine d'être creusée, vous pouvez suivre les liens vers le contenu
pertinent dans les notes de bas de page ou la bibliographie.

Bien que quelques connaissances préalables sur Bitcoin puissent aider, j'ai bon
espoir que ces leçons pourront être assimilées par tout lecteur curieux. Malgré
les liens qui peuvent exister entre elles, chaque leçon devrait se suffire à
elle-même et pouvoir être lue indépendamment. J'ai accordé une attention
particulière à éviter le jargon technique, malgré cela quelques termes
spécifiques à certains domaines restent inévitables.

Je souhaite que mon récit puisse donner l'envie à d'autres personnes de gratter
le vernis et d'examiner certaines des questions les plus profondes qu'amène
Bitcoin. Ma propre inspiration émane d'une multitude d'auteurs et de créateurs
de contenu à qui je voue une éternelle gratitude.

Enfin, et surtout : en écrivant tout ceci mon but n'est pas de vous convaincre
de quoi que ce soit. Mon but est de vous amener à penser, de vous montrer que
Bitcoin représente bien plus que ce que l'on croit. Je ne peux même pas vous
dire ce qu'est Bitcoin ou ce qu'il va vous apprendre. Vous allez devoir le
découvrir par vous-même.

\begin{quotation}\begin{samepage}
\enquote{C'est ta dernière opportunité. Tu ne pourras pas rebrousser chemin. Si
tu choisis la bleue, tout s'arrête. Tu te réveilles dans ton lit et tu crois ce
que bon te semble. Si tu prends la rouge\footnote{la \textit{orange}}, tu restes
aux Pays des Merveilles et je t'emmène au tréfonds du terrier.}
\begin{flushright} -- Morpheus
\end{flushright}\end{samepage}\end{quotation}

\begin{figure}
  \includegraphics{assets/images/bitcoin-orange-pill.jpg}
  \caption*{Souvenez-vous : je n'offre que la vérité. Rien de plus.}
  \label{fig:bitcoin-orange-pill}
\end{figure}

%
% [Morpheus]: https://en.wikipedia.org/wiki/Red_pill_and_blue_pill#The_Matrix_(1999)
% [this question]: https://twitter.com/arjunblj/status/1050073234719293440
%
% <!-- Internal -->
% [chapter1]: {{ 'bitcoin/lessons/ch1-00-philosophy' | absolute_url }}
% [chapter2]: {{ 'bitcoin/lessons/ch2-00-economics' | absolute_url }}
% [chapter3]: {{ 'bitcoin/lessons/ch3-00-technology' | absolute_url }}
%
% <!-- Wikipedia -->
% [alice]: https://en.wikipedia.org/wiki/Alice%27s_Adventures_in_Wonderland
% [carroll]: https://en.wikipedia.org/wiki/Lewis_Carroll
