\part{Économie}
\label{ch:economics}
\chapter*{Économie}

\begin{chapquote}{Lewis Carroll, \textit{Alice au pays des merveilles}}
\enquote{Un grand rosier se dressait près de l’entrée du jardin ; il était tout
couvert de roses blanches, mais trois jardiniers s’affairaient à les peindre en
rouge. Ceci sembla très curieux à Alice\ldots}
\end{chapquote}

L'argent ne pousse pas sur les arbres. C'est idiot de croire ça et nos parents
ont fait en sorte de nous l'inculquer en le répétant comme un mantra. Nous
sommes encouragés à utiliser judicieusement l'argent, à ne pas le dépenser
inconsidérément et à l'épargner quand tout va bien pour les mauvais moments.
Après tout, l'argent ne pousse pas sur les arbres.

Bitcoin m'a plus appris sur l'argent que ce que j'aurais jamais cru devoir
savoir. Grâce à lui, j'ai été forcé d'explorer l'histoire de l'argent, du
secteur bancaire, diverses écoles de pensée économique et bien d'autres choses.
La quête pour la compréhension de Bitcoin m'a mené sur une multitude de chemins
et je tente d'en explorer certains au long de ce chapitre.

Dans les sept premières leçons j'ai abordé certaines questions philosophiques
qui entourent Bitcoin. Les sept suivantes se pencheront plutôt sur l'argent et
l'économie.

~

\begin{samepage}
Partie~\ref{ch:economics} -- Économie:

\begin{enumerate}
  \setcounter{enumi}{7}
  \item La méconnaissance financière
  \item L'inflation
  \item La valeur
  \item L'argent
  \item L'histoire et le déclin de la monnaie
  \item La folie des réserves fractionnaires
  \item Une monnaie saine
\end{enumerate}
\end{samepage}

À nouveau, je ne pourrai qu'effleurer la surface. Bitcoin est non seulement
ambitieux, mais il couvre aussi profondément un large spectre de domaines,
rendant impossibles à balayer tous les sujets pertinents en une seule leçon, un
seul essai, article ou livre. Je doute même que ce soit tout simplement
possible.

Bitcoin est une nouvelle forme de monnaie, qui rend l'étude de l'économie
primordiale à sa compréhension. S'agissant de la nature humaine et des
interactions entre agents économiques, l'économie est sans doute l'une des
pièces les plus grandes et les plus floues du puzzle Bitcoin.

À nouveau, ces leçons explorent diverses choses que Bitcoin m'a apprises. Elles
sont un reflet de ma chute dans le terrier du lapin. N'ayant pas de formation en
économie, je suis clairement en-dehors de ma zone de confort et je suis tout à
fait conscient que ma compréhension est potentiellement incomplète. Je ferai de
mon mieux pour présenter ce que j'ai retenu, au risque même de passer pour un
idiot. Après tout, je cherche toujours à répondre à la question :
\textit{\enquote{Qu'avez-vous appris de Bitcoin ?}}

Après sept leçons observées sous l'angle de la philosophie, passons à l'angle de
l'économie pour en examiner sept de plus. Tout ce que j'ai à vous offrir cette
fois, c'est un cours d'économie. Terminus : \textit{une monnaie saine}.

% [the question]: https://twitter.com/arjunblj/status/1050073234719293440
