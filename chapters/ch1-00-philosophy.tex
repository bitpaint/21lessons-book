\part{Philosophie}
\label{ch:philosophy}
\chapter*{Philosophie}

\begin{chapquote}{Lewis Carroll, \textit{Alice au pays des merveilles}}
  La Souris la regarda avec curiosité (Alice crut même la voir cligner l’un de
  ses petits yeux), mais elle ne répondit rien.
\end{chapquote}

Si l'on regarde Bitcoin en surface, on pourrait conclure qu'il est lent,
inefficace, inutilement redondant et excessivement paranoïaque. Si l'on observe
Bitcoin d'un esprit curieux, on pourrait bien découvrir que les choses ne sont
pas ce qu'elles paraissent au premier coup d'œil.

Bitcoin a le chic pour mettre vos présomptions sens dessus dessous.
Régulièrement, juste au moment où vous alliez retrouver votre zone de confort,
Bitcoin viendra à nouveau fracasser vos certitudes comme un éléphant dans un
magasin de porcelaine.

\begin{figure}
  \includegraphics{assets/images/blind-monks.jpg}
  \caption{Moines aveugles examinant le taureau Bitcoin}
  \label{fig:blind-monks}
\end{figure}

Bitcoin est l'enfant de nombreuses disciplines. Tout comme des moines aveugles
examinant un éléphant, chaque personne qui approche cette nouvelle technologie
le fait sous un angle particulier. Par conséquent, chacun arrivera à différentes
conclusions sur la nature de la bête.

Les leçons qui suivent présentent certaines idées préconçues que Bitcoin a
fracassées ainsi que les conclusions auxquelles je suis arrivé. Des questions
philosophiques à propos de l'immutabilité, de la rareté, de la localité et de
l'identité sont abordées au cours des quatre premières leçons. Chaque partie se
compose de sept leçons.

~

\begin{samepage}
Partie~\ref{ch:philosophy} -- Philosophie:

\begin{enumerate}
  \item Immutabilité et changement
  \item La rareté de la rareté
  \item Réplication et localité
  \item Le problème de l'identité
  \item L'Immaculée Conception
  \item La force de la liberté d'expression
  \item Les limites du savoir
\end{enumerate}
\end{samepage}

La leçon \ref{les:5} s'intéresse à la façon dont l'histoire de Bitcoin est non
seulement fascinante mais aussi absolument essentielle à un système sans
responsables. Les deux dernières leçons de ce chapitre couvriront la force de la
liberté d'expression et les limites de notre savoir individuel, auxquelles la
profondeur étonnante du terrier du lapin Bitcoin fait écho.

J'espère que vous trouverez l'univers Bitcoin aussi pédagogique, fascinant et
amusant que je l'ai trouvé et que je le trouve encore. Je vous invite à suivre
le lapin blanc et à explorer les tréfonds du terrier. Maintenant, accrochez-vous
à votre montre à gousset, sautez et profitez de la descente.
