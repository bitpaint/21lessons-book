\part{Technologie}
\label{ch:technology}
\chapter*{Technologie}

\begin{chapquote}{Lewis Carroll, \textit{Alice au pays des merveilles}}
\enquote{Cette fois-ci, je vais m’y prendre un peu mieux} se dit-elle, et elle
commença par s’emparer de la petite clé d’or et par ouvrir la porte qui donnait
sur le jardin.
\end{chapquote}

Des clés d'or, des horloges qui ne fonctionnent qu'au hasard, des courses pour
résoudre d'étranges énigmes et des bâtisseurs anonymes et sans visages. On
dirait des contes de fée tirés du pays des merveilles mais c'est pourtant la
routine dans l'univers Bitcoin.

Comme nous l'avons vu dans le Chapitre~\ref{ch:economics}, des pans entiers du
système financier actuel sont systématiquement défaillants. À la manière
d'Alice, nous ne pouvons qu'espérer faire mieux cette fois-ci. Pourtant, grâce à
un inventeur pseudonyme, nous disposons dorénavant d'une technologie
incroyablement sophistiquée pour nous aider : Bitcoin.

Résoudre des problèmes dans un environnement radicalement décentralisé et
hostile demande des solutions uniques. Des problèmes habituellement triviaux à
résoudre n'ont rien de trivial dans cet étrange monde fait de nœuds. Pour la
plupart de ses solutions, Bitcoin repose sur une cryptographie solide, tout du
moins du point de vue de la technologie. Nous verrons dans une des leçons qui
suivent à quel point cette cryptographie est solide.

Bitcoin se sert de la cryptographie pour se détacher de la confiance dans les
autorités. Au lieu d'être tributaire d'institutions centralisées, le système
s'appuie sur l'autorité ultime de notre univers : la physique. Cependant, il
reste tout de même quelques traces de confiance. Nous examinerons ces traces
dans la seconde leçon de ce chapitre.

~

\begin{samepage}
Partie~\ref{ch:technology} -- Technologie:

\begin{enumerate}
  \setcounter{enumi}{14}
  \item La force dans les nombres
  \item Remarques sur \enquote{Ne vous fiez pas, vérifiez}
  \item Donner l'heure demande du travail
  \item Avancer lentement sans rien casser
  \item La vie privée n'est pas morte
  \item Les cypherpunks écrivent du code
  \item Métaphores pour le futur de Bitcoin
\end{enumerate}
\end{samepage}

Les leçons suivantes traitent de l'éthique du développement technologique de
Bitcoin, un aspect potentiellement aussi important que la technologie elle-même.
Bitcoin, ce n'est pas la prochaine appli à la mode sur votre téléphone. C'est la
base d'une nouvelle réalité économique, ce qui explique qu'il devrait être
considéré comme un logiciel financier de qualité nucléaire.

Où en sommes-nous de cette révolution financière, sociétale et technologique ?
Les réseaux et technologies du passé peuvent figurer des métaphores au futur de
Bitcoin, que nous aborderons dans la toute dernière leçon de ce chapitre.

Une dernière fois, attachez vos ceintures et profitez de la descente. Comme
toutes les technologies exponentielles, nous allons devenir paraboliques.
