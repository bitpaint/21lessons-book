\addpart{Considérations finales}
\pdfbookmark{Conclusion}{conclusion}
\label{ch:conclusion}

\chapter*{Conclusion}

\begin{chapquote}{Lewis Carroll, \textit{Alice au pays des merveilles}}
\enquote{Commencez au commencement}  dit le Roi d’un ton grave, \enquote{et
continuez jusqu’à ce que vous arriviez à la fin ; ensuite, arrêtez-vous.}
\end{chapquote}

Comme je le disais en introduction, je pense que la réponse à la question
\textit{Qu'avez-vous appris de Bitcoin ?} sera toujours incomplète. La symbiose
de ce qu'on pourrait appeler de multiples systèmes vivants -- Bitcoin, la
technosphère et l'économie -- est trop entremêlée, les sujets sont trop nombreux
et les choses avancent trop vite pour être entièrement comprises par une seule
personne.

Même sans le comprendre dans son entièreté et même avec toutes ses excentricités
et ses défauts apparents, Bitcoin fonctionne indéniablement. Il continue à
produire des blocs à peu près toutes les dix minutes d'une manière admirable.
Plus Bitcoin continuera de fonctionner, plus les gens seront enclins à
l'utiliser.

\begin{quotation}\begin{samepage}
\enquote{C'est vrai que les choses sont belles lorsqu'elles fonctionnent. L'art
est fonctionnel.}
\begin{flushright} -- Giannina Braschi\footnote{Giannina Braschi,
\textit{L'Empire des rêves} \cite{braschi2011empire}}
\end{flushright}\end{samepage}\end{quotation}

\paragraph{} Bitcoin est un enfant d'Internet. Sa croissance est exponentielle,
gommant les frontières entre les disciplines. Par exemple, l'endroit où finit le
royaume purement technologique et où en commence un autre n'est pas clairement
établi. Bitcoin a effectivement besoin d'ordinateurs pour fonctionner mais
l'informatique ne suffit pas à le comprendre. Non seulement, par ses rouages,
Bitcoin ne connaît pas de frontières, mais il n'est pas non plus cloisonné
académiquement.

L'économie, la politique, la théorie des jeux, l'histoire monétaire, la théorie
des réseaux, la finance, la cryptographie, la théorie de l'information, la
censure, la loi et les régulations, les organisations humaines, la psychologie
-- tout ceci avec bien d'autres champs d'expertise qui pourraient faire avancer
la quête de la connaissance de Bitcoin et de son fonctionnement.

Aucune invention n'est seule responsable de son succès. C'est la combinaison
de multiples éléments auparavant sans lien qui, collés ensemble par l'incitation
de la théorie des jeux, constituent la révolution qu'est Bitcoin. Ce qui fait de
Satoshi un génie c'est le sublime mariage de nombreuses disciplines. 

\paragraph{} Comme tout système complexe, Bitcoin doit faire des compromis en
termes d'efficacité, de coût, de sécurité et de bien d'autres aspects. Tout
comme il n'y a pas de solution parfaite à la quadrature du cercle, toute
solution aux problèmes qu'entend résoudre Bitcoin sera toujours imparfaite.

\begin{quotation}\begin{samepage}
\enquote{Je ne crois pas que nous pourrons avoir à nouveau une bonne monnaie
avant que nous ne la retirions des mains du gouvernement, je veux dire, nous ne
pouvons pas la retirer violemment de leur contrôle, tout ce que nous pouvons
faire sera de manière rusée et détournée, introduire quelque chose qu'ils ne
pourront arrêter.}
\begin{flushright} -- Friedrich Hayek\footnote{Friedrich Hayek sur la politique
monétaire, l'étalon-or, les déficits, l'inflation et John Maynard Keynes
\url{https://youtu.be/EYhEDxFwFRU}}
\end{flushright}\end{samepage}\end{quotation}

Bitcoin est la manière rusée et détournée de réintroduire une bonne monnaie dans
notre monde. Il le fait en plaçant un individu souverain derrière chaque nœud,
tout comme De Vinci tentait de résoudre l'inextricable quadrature du cercle en
plaçant l'Homme de Vitruve en son centre. Les nœuds retirent en effet tout
concept de centre, créant un système incroyablement anti-fragile et extrêmement
difficile à stopper. Bitcoin est vivant et cœur battra probablement plus
longtemps que les nôtres.

J'espère que vous avez apprécié ces vingt et une leçons. La leçon la plus
importante est peut-être que Bitcoin mérite une approche holistique, sous
plusieurs angles, si l'on souhaite en dresser un tableau le moins inachevé
possible. De la même façon que l'on détruit un système complexe en retirant un
de ses éléments, examiner individuellement chaque partie de Bitcoin semble en
gâcher la compréhension. Si une seule personne efface \enquote{blockchain} de
son vocabulaire et le remplace par \enquote{une chaîne de blocs}, je pourrai
mourir tranquille.

Quoi qu'il en soit, mon voyage n'est pas terminé. J'ai l'intention de 
m'aventurer encore plus profondément dans le terrier du lapin et je vous invite
à me suivre sur le chemin\footnote{\url{https://twitter.com/dergigi}}.

% <!-- Twitter -->
% [dergigi]: https://twitter.com/dergigi
%
% <!-- Internal -->
% [sly roundabout way]: https://youtu.be/EYhEDxFwFRU?t=1124
% [Giannina Braschi]: https://en.wikipedia.org/wiki/Braschi%27s_Empire_of_Dreams
