
\def\bitcoinB{\leavevmode
  {\setbox0=\hbox{\textsf{B}}%
    \dimen0\ht0 \advance\dimen0 0.2ex
    \ooalign{\hfil \box0\hfil\cr
      \hfil\vrule height \dimen0 depth.2ex\hfil\cr
    }%
  }%
}

\chapter*{À propos de ce livre \\ (... et de son auteur)}
\pdfbookmark{À propos de ce livre (... et de son auteur)}{about}

Il s'agit d'un livre un peu particulier. Mais bon, Bitcoin est aussi une
technologie un peu particulière, donc un livre particulier à propos de Bitcoin
est sans doute adapté. Je ne sais pas vraiment si je suis un mec particulier
(j'aime bien penser que je suis un mec \textit{normal}) mais l'histoire de ce
livre, et de comment j'en suis venu à devenir auteur, mérite d'être racontée.

Premièrement, je ne suis pas auteur. Je suis ingénieur. Je n'ai pas étudié
les lettres. J'ai appris le code et comment coder. Deuxièmement, je n'ai jamais
eu l'intention d'écrire un livre, encore moins un livre sur Bitcoin. Bon sang,
ce n'est même pas ma langue maternelle.\footnote{La raison pour laquelle
j'écris ce livre en anglais, c'est que mon cerveau fonctionne d'une manière
bizarre. Dès que ça devient technique, il passe tout seul à l'anglais.} Je suis
juste un gars qui a attrapé le virus Bitcoin. Gravement.

Qui suis-\textit{je} alors pour écrire un livre sur Bitcoin ? Bonne question. En
bref, la réponse est simple : je suis Gigi et je suis un bitcoiner.

Mais le développement est un peu plus nuancé.

\paragraph{}
Je viens de l'informatique et du développement logiciel. Dans une vie
antérieure, j'étais dans une équipe de recherche qui tentait d'apprendre à
penser et à réfléchir à des ordinateurs, entre autres choses. Dans une vie
encore plus ancienne, j'écrivais des logiciels de traitement automatisé de
passeports et d'autres trucs du même style, ce qui est encore plus effrayant. Je
m'y connais un peu en informatique et en réseaux, donc j'imagine que j'ai
quelques longueurs d'avance pour comprendre l'aspect technique de Bitcoin. En
revanche, comme j'essaie de le souligner dans ce livre, cet aspect technique ne
représente qu'une petite partie de l'animal qu'est Bitcoin. Et chacune de ses
parties est importante.

Ce livre a vu le jour grâce à une seule question toute bête :
\textit{\enquote{Qu'avez-vous appris de Bitcoin ?}}. J'ai tenté d'y répondre
d'un simple tweet. Puis le tweet est devenu tempête de tweets. Cette tempête
s'est transformée en article. L'article a évolué en trois articles. Trois
articles sont devenus 21 leçons. Et 21 leçons ont engendré ce livre. Du coup, je
suppose que je suis juste nul pour résumer ma pensée en un seul tweet.

\paragraph{}
\textit{\enquote{Pourquoi écrire ce livre ?}}, me direz-vous. À nouveau, deux
réponses : une courte et une longue. La courte, c'est que je devais le faire.
J'étais (et suis toujours) \textit{possédé} par Bitcoin. Il ne cesse de me
fasciner. Je ne peux m'arrêter de penser à lui et aux implications qu'il aura
dans nos sociétés. La réponse longue, c'est que je crois que Bitcoin est
l'invention la plus importante de notre époque et que la nature de cette
invention doit être comprise par le plus grand nombre. Bitcoin reste l'un des
phénomènes les plus mal compris du monde actuel et ça m'a pris des années pour
réaliser pleinement le sérieux de cette technologie extraterrestre. Comprendre
ce qu'est Bitcoin et comment il va transformer nos sociétés est une expérience
marquante. J'ai l'espoir de faire germer dans votre tête les graines qui
pourraient vous conduire à cette prise de conscience.

Dans l'ordre des choses, bien que ce passage soit intitulé \enquote{\textit{À
propos de ce livre (... et de son auteur)}}, ce livre, qui je suis et ce que
j'ai fait n'ont pas vraiment d'importance. Je suis juste un nœud du réseau, à la
fois littéralement \textit{et} métaphoriquement. De toute façon, vous ne devriez
pas croire ce que je dis. Comme nous, bitcoiners, aimons le répéter : faites vos
propres recherches. Et par-dessus tout : ne vous fiez pas, vérifiez.

J'ai fait mes recherches au mieux afin de vous permettre, cher lecteur, de vous
plonger dans de nombreuses ressources. En plus des notes et des citations de ce
livre, j'essaie de garder à jour une liste de contenu sur
\href{https://21lessons.com/rabbithole}{21lessons.com/rabbithole} et sur
\href{https://bitcoin-resources.com}{bitcoin-resources.com}, qui recense
également plein d'autres morceaux choisis, livres, podcasts, qui vous aideront à
comprendre ce qu'est Bitcoin.

\paragraph{}
En résumé, c'est juste un livre qui parle de Bitcoin, écrit par un bitcoiner.
Bitcoin n'a pas besoin de ce livre, et vous n'avez sans doute pas besoin de ce
livre pour comprendre Bitcoin. Je pense que vous comprendrez Bitcoin dès que
\textit{vous} serez prêt et je crois aussi que vos premières fractions d'un
bitcoin vous trouveront dès que vous serez prêt à les recevoir. Par essence,
chacun obtiendra \bitcoinB{}itcoin exactement au bon moment. Dans l'intervalle,
Bitcoin existe et c'est bien suffisant.\footnote{Beautyon, \textit{Bitcoin is.
And that is enough.}~\cite{bitcoin-is}}
