\chapter*{Préface}

S'enfoncer dans le terrier du lapin Bitcoin est une expérience bizarre. Comme
tant d'autres, j'ai l'impression que ces deux dernières années passées à étudier
Bitcoin m'ont plus appris que deux décennies d'éducation classique.

Ces leçons forment la quintessence de ce que j'ai découvert. D'abord publié
comme une série d'articles intitulée \textit{« Ce que Bitcoin m'a enseigné »},
ce qui suit peut être vu comme la troisième édition de la série d'origine.

À l'instar de Bitcoin, ces leçons sont évolutives. Je compte revenir dessus
régulièrement, en publiant plus tard des mises à jour et du contenu
supplémentaire.

Contrairement à Bitcoin, les futures versions de ce projet ne seront pas
nécessairement rétro-compatibles. Certaines leçons pourront être complétées,
d'autres seront retravaillées voire même remplacées.

Bitcoin est un professeur intarissable, c'est pour cela que je ne considère pas
ces leçons comme exhaustives ou définitives. Elles sont le reflet de mon propre
périple au cœur du terrier. Il existe bien d'autres leçons à tirer, de fait
chaque personne qui entrera dans le monde de Bitcoin en retirera des
connaissances différentes.

J'espère que vous trouverez une utilité à ces leçons et que leur apprentissage
par la lecture vous paraîtra moins pénible et douloureux que je ne l'ai parfois
vécu par l'expérience.

% <!-- Internal -->
% [I]: 
%
% <!-- Twitter -->
% [dergigi]: https://twitter.com/dergigi
%
% <!-- Wikipedia -->
% [alice]: https://en.wikipedia.org/wiki/Alice%27s_Adventures_in_Wonderland
% [carroll]: https://en.wikipedia.org/wiki/Lewis_Carroll
